\documentclass{report}

\newcommand{\name}{Zayd Hammoudeh}
\newcommand{\course}{MATH562}
\newcommand{\assnName}{Homework~\#4}
\newcommand{\dueDate}{February~7,~2020}

\usepackage[margin=1in]{geometry}
\usepackage[skip=4pt]{caption}      % ``skip'' sets the spacing between the figure and the caption.
\usepackage{tikz}
\usetikzlibrary{arrows.meta,decorations.markings,shadows,positioning,calc}
\usepackage{pgfplots}               % Needed for plotting
\pgfplotsset{compat=newest}
\usepgfplotslibrary{fillbetween}    % Allow for highlighting under a curve
\usepackage{amsmath}                % Allows for piecewise functions using the ``cases'' construct
\usepackage{bbm}                    % Enables \mathbbm{1}
\usepackage{siunitx}                % Allows for ``S'' alignment in table to align by decimal point

\usepackage[obeyspaces,spaces]{url} % Used for typesetting with the ``path'' command
\usepackage[hidelinks]{hyperref}    % Make the cross references clickable hyperlinks
\usepackage[bottom]{footmisc}       % Prevents the table going below the footnote
\usepackage{nccmath}                % Needed in the workaround for the ``aligncustom'' environment
\usepackage{amssymb}                % Used for black QED symbol
\usepackage{bm}                     % Allows for bolding math symbols.
\usepackage{tabto}                  % Allows to tab to certain point on a line
\usepackage{float}
\usepackage{subcaption}             % Allows use of the ``subfigure'' environment
\usepackage{enumerate}              % Allow enumeration other than just numbers

\usepackage[noend]{algpseudocode}
\usepackage[Algorithm,ruled]{algorithm}
\algnewcommand\algorithmicforeach{\textbf{for each}}
\algdef{S}[FOR]{ForEach}[1]{\algorithmicforeach\ #1\ \algorithmicdo}

%---------------------------------------------------%
%     Define Distances Used for Document Margins    %
%---------------------------------------------------%

\newcommand{\hangindentdistance}{1cm}
\newcommand{\defaultleftmargin}{0.25in}
\newcommand{\questionleftmargin}{-.5in}

\setlength{\parskip}{1em}
\setlength{\oddsidemargin}{\defaultleftmargin}

%---------------------------------------------------%
%      Configure the Document Header and Footer     %
%---------------------------------------------------%

% Set up page formatting
\usepackage{todonotes}
\usepackage{fancyhdr}                   % Used for every page footer and title.
\pagestyle{fancy}
\fancyhf{}                              % Clears both the header and footer
\renewcommand{\headrulewidth}{0pt}      % Eliminates line at the top of the page.
\fancyfoot[LO]{\course\ -- \assnName}   % Left
\fancyfoot[CO]{\thepage}                % Center
\fancyfoot[RO]{\name}                   % Right

%---------------------------------------------------%
%           Define the Title Page Entries           %
%---------------------------------------------------%

\title{\textbf{\course\ -- \assnName}}
\author{\name}

%---------------------------------------------------%
% Define the Environments for the Problem Inclusion %
%---------------------------------------------------%

\usepackage{scrextend}
\newcounter{problemCount}
\setcounter{problemCount}{0}  % Reset the subproblem counter

\newcounter{subProbCount}[problemCount]   % Reset subProbCount any time problemCount changes.
\renewcommand{\thesubProbCount}{\alph{subProbCount}}  % Make it so the counter is referenced as a number

\newenvironment{problemshell}{
  \begin{addmargin}[\questionleftmargin]{0em}
    \par%
    \medskip
    \leftskip=0pt\rightskip=0pt%
    \setlength{\parindent}{0pt}
    \bfseries
  }
  {
    \par\medskip
  \end{addmargin}
}
\newenvironment{problem}
{%
  \refstepcounter{problemCount} % Increment the subproblem counter.  This must be before the exercise to ensure proper numbering of claims and lemmas.
  \begin{problemshell}
    \noindent \textit{Exercise~\#\arabic{problemCount}} \\
  }
  {
  \end{problemshell}
  %  \setcounter{subProbCount}{0} % Reset the subproblem counter
}
\newenvironment{subproblem}
{%
  \begin{problemshell}
    \refstepcounter{subProbCount} % Increment the subproblem counter
    \setlength{\leftskip}{\hangindentdistance}
    % Print the subproblem count and offset to the left
    \hspace{-\hangindentdistance}(\alph{subProbCount}) \tabto{0pt}
  }
  {
  \end{problemshell}
}

% Change interline spacing.
\renewcommand{\baselinestretch}{1.1}
\newenvironment{aligncustom}
{ \csname align*\endcsname % Need to do this instead of \begin{align*} because of LaTeX bug.
  \centering
}
{
  \csname endalign*\endcsname
}


%---------------------------------------------------%
% Define the Environments for the Problem Inclusion %
%---------------------------------------------------%

\usepackage{amsthm}       % Allows use of the ``proof'' environment.

% Number lemmas and claims using the problem count
\newtheorem{claim}{Claim}[problemCount]
\newtheorem{lemma}{Lemma}[problemCount]

%---------------------------------------------------%
%       Define commands related to managing         %
%    floats (e.g., images) across multiple pages    %
%---------------------------------------------------%

\usepackage{placeins}     % Allows \FloatBarrier

% Prevent preceding floats going to this page
\newcommand{\floatnewpage}{\FloatBarrier\newpage}

% Add the specified input file and prevent any floated figures/tables going onto the same page as new input
\newcommand{\problemFile}[1]{
  \floatnewpage
  \input{#1}
}

\newcommand{\probNum}[1]{(\textnormal{Problem: #1})}

\newcommand{\etal}{~et~al.}

% Used for including standalone docs
\usepackage{standalone}

% Imported via UltiSnips
% Unbreakable dash:
%  Words hyphened with these dashes can also be broken at other positions than the dash
%    \-/ hyphen
%    \-- en-dash
%    \--- em-dash
%    extdash unbreakable dashes
%
%  No line breaks possible at the hyphen
%    \=/ hyphen
%    \== en-dash
%    \=== em-dash
\usepackage[shortcuts]{extdash}

% Imported via UltiSnips
\usepackage{color}
\newcommand{\colortext}[2]{{\color{#1} #2}}
\newcommand{\red}[1]{\colortext{red}{#1}}
\newcommand{\blue}[1]{\colortext{blue}{#1}}
\newcommand{\green}[1]{\colortext{green}{#1}}

% Imported via UltiSnips
\usepackage{amsmath}
\DeclareMathOperator*{\argmax}{arg\,max}
\DeclareMathOperator*{\argmin}{arg\,min}
\usepackage{amsfonts}  % Used for \mathbb and \mathcal
\usepackage{amssymb}

% Imported via UltiSnips
\usepackage{mathtools} % for "\DeclarePairedDelimiter" macro
% \swapifbranches changes unstarred paired delimiters to starred and
% vice versa.  This means by default, paired delimiters have the star.
\usepackage{etoolbox}
\newcommand\swapifbranches[3]{#1{#3}{#2}}
\makeatletter
\MHInternalSyntaxOn
\patchcmd{\DeclarePairedDelimiter}{\@ifstar}{\swapifbranches\@ifstar}{}{}
\MHInternalSyntaxOff
\makeatother
% Place after swap to ensure swap star
\DeclarePairedDelimiter{\sbrack}{\lbrack}{\rbrack}
\DeclarePairedDelimiter{\floor}{\lfloor}{\rfloor}
\DeclarePairedDelimiter{\ceil}{\lceil}{\rceil}
\DeclarePairedDelimiter{\abs}{\lvert}{\rvert}
\DeclarePairedDelimiter{\norm}{\lVert}{\rVert}
\usepackage{bm}
\DeclarePairedDelimiterX\set[1]\lbrace\rbrace{#1}
\DeclarePairedDelimiterX\setbuild[2]\lbrace\rbrace{#1 \bm: #2}
\newcommand{\setint}[1]{{\sbrack{#1}}}
\newcommand{\func}[3]{{#1:#2\rightarrow#3}}
% \newcommand{\defeq}{\stackrel{\mathclap{\mbox{\tiny def}}}{=}}
\newcommand{\defeq}{\coloneqq}
\newcommand{\fedeq}{\eqqcolon}
\newcommand{\expect}[1]{\mathbb{E}\sbrack{#1}}
% Expectation with the subscript defining the distribution
\newcommand{\expectS}[2]{\mathbb{E}_{#1}\sbrack{#2}}

% Allow numbering in align*
\newcommand{\numberthis}{\addtocounter{equation}{1}\tag{\theequation}}

\newcommand{\ints}{\mathbb{Z}}
\newcommand{\nats}{\mathbb{N}}
\newcommand{\real}{\mathbb{R}}
\newcommand{\realnn}{\real_{{\geq}0}}  % Set of non-negative real numbers

\newcommand{\iidsim}{\stackrel{\mathclap{\mbox{\tiny i.i.d.}}}{\sim}}

\newcommand{\normaldist}[2]{{\mathcal{N}\mathopen{}\left(#1,#2\right)\mathclose{}}}

% Imported via UltiSnips
\usepackage{array}  % Provides a way add a \centering command to a p-column
\usepackage{arydshln}  % Introduces hdashline & cdashline
\usepackage{bigdelim}
\usepackage{booktabs}
\usepackage{multirow}
\usepackage{makecell}  % Needed for multirowcell

% % Imported via UltiSnips
\usepackage{amsthm}
\newtheorem{theorem}{Theorem}[section]
% \newtheorem{corollary}{Corollary}[theorem]  % Corollary number derives from theorem
% \newtheorem{lemma}[theorem]{Lemma}  % Lemma and theorem share same counter
% \newtheorem{claim}[theorem]{Claim}  % Same numbering as lemma and theorem
% \newtheorem*{remark}{Remark}
% \newtheorem*{note}{Note}
% \newtheoremstyle{definition}  % <name>
% {3pt}   % <Space above>
% {3pt}   % <Space below>
% % {\itshape}     % <Body font>
% {\normalfont}   % <Body font>
% {}      % <Indent amount>
% {\bfseries} % <Theorem head font>
% {:}     % <Punctuation after theorem head>
% {.5em}  % <Space after theorem head>
% {}      % <Theorem head spec (can be left empty, meaning `normal')>
% \theoremstyle{definition}
% \newtheorem{definition}{Def.}[section]

% % Imported via UltiSnips
% \usepackage[noend]{algpseudocode}
\usepackage[Algorithm,ruled]{algorithm}
% \algnewcommand\algorithmicforeach{\textbf{for each}}
% \algdef{S}[FOR]{ForEach}[1]{\algorithmicforeach\ #1\ \algorithmicdo}
% \newcommand{\algin}[1]{\hspace*{\algorithmicindent} \textbf{Input} #1\\}
% \newcommand{\algin}[1]{\textbf{Input} #1}
% \newcommand{\algout}[1]{\hspace*{\algorithmicindent} \textbf{Output} #1}

% Imported via UltiSnips
\usepackage{tikz}
\usetikzlibrary{arrows,decorations.markings,shadows,positioning,calc,backgrounds,shapes}

\usepackage{pgfplots}
\pgfplotsset{compat=1.13}
\usepackage{pgfplotstable}
% \usepackage{subcaption}  % Cannot be used with subfigure

% Handle empty parameters
\usepackage{xifthen}
\newcommand{\ifempty}[3]{%
  \ifthenelse{\isempty{#1}}{#2}{#3}%
}


\newcommand{\thhat}[1]{\hat{\theta}_{#1}}
\newcommand{\var}[1]{\text{Var}\left(#1\right)}

\begin{document}
  \maketitle

  \noindent
  \textbf{Name}: \name\\
  \textbf{Course}: \course\\
  \textbf{Assignment}: \assnName\\
  \textbf{Due Date}: \dueDate

  % \noindent
  % \textbf{Other Student Discussions}: I discussed the problems in this homework with the following student(s) below.  All write-ups were prepared independently.
  % \vspace{-1em}
  % \begin{itemize}
  %   \item Viet Lai
  % \end{itemize}

  \newpage
  \begin{problem}
  \probNum{5.6.3} If $\hat{\theta}$~is sufficient for~$\theta$, show that any one-to-one function of~$\hat{\theta}$ is also sufficient for~$\theta$.
\end{problem}

Define ${\hat{\theta}' \defeq f\left(\hat{\theta}\right)}$ where $f$~is one\=/to\=/one.

Since $\hat{\theta}$ is sufficient, it is possible to partition the likelihood function as:
\begin{equation}
  L(\theta) = \prod_{i=1}^{n} p_{X}(k_i;\theta) = p_{\hat{\theta}}(\theta_e;\theta)b(k_1,\ldots,k_n)\text{.}
\end{equation}

Since $f$ is one-to-one, its inverse $f^{-1}$, exists even if it is piecewise.  That means there is an equivalent formulation with substitution:
\begin{equation}
  L(\theta) =  p_{\hat{\theta}'}(f^{-1}(\theta_e);\theta)b(k_1,\ldots,k_n)\text{,}
\end{equation}
which by definition makes $\hat{\theta}'$ sufficient.

  \newpage
  \begin{problem}
  \probNum{5.6.4} Show that ${\hat{\sigma}^2 = \sum_{i=1}^{n} Y_i^2}$ is sufficient for $\sigma^2$ if ${Y_1,\ldots,Y_n}$ is a random sample from a normal pdf with ${\mu = 0}$.
\end{problem}

\begin{align}
  L(\theta) &= \left(2\pi\sigma^2\right)^{-\frac{n}{2}} e^{-\frac{\sum_{i=1}^{n} (x_i - \mu)^2}{2\sigma^2}} \\
            &= \left(2\pi\sigma^2\right)^{-\frac{n}{2}} e^{-\frac{\sum_{i=1}^{n} x_{i}^{2}}{2\sigma^2}} & \mu = 0 \\
            &= \underbrace{\left(2\pi\sigma^2\right)^{-\frac{n}{2}} e^{-\frac{\hat{\sigma}^2}{2\sigma^2}}}_{p_{\hat{\sigma}^2}(\sigma^{2}_e;\sigma^2)} \underbrace{1}_{b(Y_1,\ldots,Y_n)}
\end{align}

  \newpage
  \begin{problem}
  \probNum{5.6.7} Suppose a random sample of size~$n$ is drawn from the~pdf
  \begin{equation*}
    f_{Y}(y;\theta) = e^{-(y-\theta)}\text{,  } \theta \leq y
  \end{equation*}
\end{problem}

\begin{subproblem}
  Show that ${\hat{\theta} = Y_{\min}}$ is sufficient for the threshold parameter~$\theta$.
\end{subproblem}
\begin{align}
  L(\theta) &= \prod_{i=1}^{n} e^{-(y_i - \theta)} \mathbbm{1}\sbrack{y_i \geq \theta} \label{eq:HW04:P04:Indicator}\\
            &= \underbrace{\bigg(e^{n\theta} \mathbbm{1}\sbrack{Y_{\min} \geq \theta} \bigg)}_{p_{\hat{\theta}}(Y_{\min};\theta)} \underbrace{\left( e^{-\sum_{i=1}^{n} y_i}  \right)}_{b(Y_1,\ldots,Y_{n})}
\end{align}

\begin{subproblem}
  Show that $Y_{\max}$ is not sufficient for~$\theta$.
\end{subproblem}

Returning to Eq.~\ref{eq:HW04:P04:Indicator}, it is clear that the indicator function~$\mathbbm{1}\sbrack{\ldots}$ terms must be a component in~$p_{\hat{\theta}}$ since each indicator contains~$\theta$ and~$y_i$.  However, it is impossible to formulate the indicators in terms of~$Y_{\max}$ as it was possible for~${Y_{\min}}$.  Hence it is not sufficient by Theorem~5.6.1.

  \newpage
  \begin{problem}
  \probNum{5.7.01} How large a sample must be taken from a normal pdf where ${\expect{Y} = 18}$ in order to guarantee that ${\hat{\mu}_n = \bar{Y}_n = \frac{1}{n} \sum_{i=1}^{n} Y_i}$ has a 90\%~probability of lying somewhere in the interval~${\sbrack{16,20}}$?  Assume that ${\sigma = 5.0}$.
\end{problem}

To have 90\% confidence, each of the tails should have 5\%~of the probability mass.  Using the $Z$\=/table, this corresponds to ${Z \approx 1.645}$.

\begin{align}
  Z \approx 1.645 &= \frac{X - \mu}{\sigma} \\
                  &= \frac{2\sqrt{n}}{5.0} \\
                n &= \ceil{16.91} \\
                  &= \boxed{17} \text{.}
\end{align}



  \newpage
  \begin{problem}
  Let ${Y_1,Y_2,\ldots,Y_n}$ be a random sample of size~$n$ from a normal pdf having~${\mu = 0}$.  Show that ${S^2 = \frac{1}{n} \sum_{i=1}^{n} Y_{i}^2}$ is a consistent estimator for ${\sigma^{2} = \text{Var}(Y)}$.
\end{problem}

\begin{align}
  \expect{S^2} &= \expect{\frac{1}{n}\sum_{i=1}^{n} Y_{i}^{2}} \\
               &= \frac{1}{n} \sum_{i=1}^{n} \expect{Y_{i}^{2}} \\
               &= \frac{1}{n} \sum_{i=1}^{n} \left( \sigma^2 - \mu^2 \right) \\
               &= \frac{n\sigma^{2}}{n} & \mu = 0\\
               &= \boxed{\sigma^2}
\end{align}

  \newpage
  \begin{problem}
  Let ${Y_1,Y_2,\ldots,Y_n}$ be a random sample of size~$n$ from from the exponential pdf, ${f_Y(y;\lambda) = \lambda e^{-\lambda y} \text{, } y > 0}$.
\end{problem}

\begin{subproblem}
  Show that ${\hat{\lambda}_n = Y_1}$ is not consistent for~$\lambda$.
\end{subproblem}

\begin{subproblem}
  Show that ${\hat{\lambda}_n = \sum_{i=1}^{n} Y_i}$ is not a consistent for~$\lambda$.
\end{subproblem}

  \newpage
  \begin{problem}
  Suppose that~$X$ is a geometric random variable, where ${p_{X}(k\vert\theta) = (1-\theta)^{k-1}\theta\text{,  }k=1,2,\ldots}$. Assume that the prior distribution for~$\theta$ is the beta pdf with parameters~$r$ and~$s$.  Find the posterior distribution for~$\theta$.
\end{problem}

\begin{align}
  g(\theta;k) &= \frac{\left(\frac{\Gamma(r+s)}{\Gamma(r)\Gamma(s)}\right) \theta^{r} (1 - \theta)^{s+k-2}}{\int_{0}^{1} \left(\frac{\Gamma(r+s)}{\Gamma(r)\Gamma(s)}\right) \theta^{r} (1 - \theta)^{s+k-2} d\theta} \\
              &= \frac{\left(\frac{\Gamma(r+s+k)}{\Gamma(r+1)\Gamma(s+k-1)}\right) \theta^{r} (1 - \theta)^{s+k-2}}{\int_{0}^{1} \frac{\Gamma(r+s+k)}{\Gamma(r+1)\Gamma(s+k-1)} \theta^{r} (1 - \theta)^{s+k-2} d\theta} \\
              &= \left(\frac{\Gamma(r+s+k)}{\Gamma(r+1)\Gamma(s+k-1)}\right) \theta^{r} (1 - \theta)^{s+k-2}
\end{align}

  \newpage
  \begin{problem}
  \probNum{5.8.2} Find the squared loss ${\sbrack{L(\hat{\theta},\theta) = \left(\hat{\theta} - \theta\right)^2}}$ Bayes estimate for the~$\theta$ in Example~5.8.2 and express it as a weighted average of the maximum likelihood estimate for~$\theta$ and the mean of the prior pdf.
\end{problem}

The squared loss Bayes estimate for~$\theta$ is the mean of the penultimate equation on page~333.  This is found from the Beta Distribution with~${r = k + 4}$ and~${s = n - k + 102}$.  Therefore, the mean is~$\frac{k+4}{n+106}$.

\noindent
It is easy to show that the MLE of the Binomial Distribution is~${\frac{k}{n}}$.

\noindent
The mean of the prior is from the Beta Distribution with~${r=4}$ and ${s=102}$ making it~${\frac{4}{106}}$.

\noindent
Specifying it as the weighted sum as required in the problem makes it:
\begin{equation}
  \boxed{\frac{n}{n+106}\left(\frac{k}{n}\right) + \frac{106}{n + 106} \left(\frac{4}{106}\right)}
\end{equation}

  \newpage
  \begin{problem}
  \probNum{5.8.3} Suppose that the binomial pdf described in Example~5.8.2 refers to the number of votes a candidate might receive in a poll conducted before the general election. Moreover, suppose a beta prior distribution has been assigned to~$\theta$, and every indicator suggests the election will be close.  The pollster, then, has good reason for concentrating the bulk of the prior distribution around the value~${\theta = \frac{1}{2}}$.  Setting the two beta parameters $r$~and $s$~both equal to~135 will accomplish that objective (in the event ${r = s = 135}$, the probability of~$\theta$ being between~$0.45$ and~$0.55$ is approximately~0.90).
\end{problem}

\begin{subproblem}
  Find the corresponding posterior distribution
\end{subproblem}
\begin{align}
  g(\theta;k) &= \frac{\binom{n}{k} \left(\frac{\Gamma(r+s)}{\Gamma(r)\Gamma(s)}\right) \theta^{r + k - 1} (1 - \theta)^{s+n-1-k}}{\int_{0}^{1} \binom{n}{k} \left(\frac{\Gamma(r+s)}{\Gamma(r)\Gamma(s)}\right) \theta^{r+k-1} (1 - \theta)^{s+n-k-1} d\theta} \\
  g(\theta;k) &= \frac{\left(\frac{\Gamma(r+s+n)}{\Gamma(r+k)\Gamma(s + n - k)}\right) \theta^{r + k - 1} (1 - \theta)^{s+n-1-k}}{\int_{0}^{1} \left(\frac{\Gamma(r+s+n)}{\Gamma(r+k)\Gamma(s + n -k)}\right) \theta^{r+k-1} (1 - \theta)^{s+n-k-1} d\theta} \\
  g(\theta;k) &= \left(\frac{\Gamma(r+s+n)}{\Gamma(r+k)\Gamma(s + n - k)}\right) \theta^{r + k - 1} (1 - \theta)^{s+n-1-k} \\
  g(\theta;k) &= \left(\frac{\Gamma(270+n)}{\Gamma(135+k)\Gamma(135 + n - k)}\right) \theta^{134 + k} (1 - \theta)^{134+n-k}
\end{align}

\begin{subproblem}
  Find the squared-error loss Bayes estimate for~$\theta$ and express it as a weighted average of the maximum likelihood estimate for~$\theta$ and the mean of the prior pdf.
\end{subproblem}
The Beta distribution's mean is~${\frac{r}{r+s}}$.  This is well-known and used without proof.

\noindent
The squared-error loss of~${g(\theta;k)}$ is its mean which for ${r=135+k}$ and ${s = 135 + n - k}$.  That makes the squared-loss Bayes estimate: ${\frac{135 + k}{270 + n}}$. For the prior, ${r=135}$ and~${s=135}$.   Therefore, the prior's mean is~${\frac{1}{2}}$.

\noindent
The MLE of~$\theta$ is found from the Binomial Distribution.  It is easy to show it is~${\frac{\bar{k}}{n}}$.  For a single sample, it is~$\frac{k}{n}$.

\noindent
Redefining the squared-loss of the Bayes' estimate we find:
\begin{equation}
  \boxed{\frac{n}{270 + n} \left(\frac{k}{n}\right) + \frac{270}{270+n} \left(\frac{1}{2}\right)}
\end{equation}

  % \newpage
  % \begin{problem}
  \probNum{5.6.4} Show that ${\hat{\sigma}^2 = \sum_{i=1}^{n} Y_i^2}$ is sufficient for $\sigma^2$ if ${Y_1,\ldots,Y_n}$ is a random sample from a normal pdf with ${\mu = 0}$.
\end{problem}

\begin{align}
  L(\theta) &= \left(2\pi\sigma^2\right)^{-\frac{n}{2}} e^{-\frac{\sum_{i=1}^{n} (x_i - \mu)^2}{2\sigma^2}} \\
            &= \left(2\pi\sigma^2\right)^{-\frac{n}{2}} e^{-\frac{\sum_{i=1}^{n} x_{i}^{2}}{2\sigma^2}} & \mu = 0 \\
            &= \underbrace{\left(2\pi\sigma^2\right)^{-\frac{n}{2}} e^{-\frac{\hat{\sigma}^2}{2\sigma^2}}}_{p_{\hat{\sigma}^2}(\sigma^{2}_e;\sigma^2)} \underbrace{1}_{b(Y_1,\ldots,Y_n)}
\end{align}

  % \newpage
  % \begin{problem}
  \probNum{5.6.7} Suppose a random sample of size~$n$ is drawn from the~pdf
  \begin{equation*}
    f_{Y}(y;\theta) = e^{-(y-\theta)}\text{,  } \theta \leq y
  \end{equation*}
\end{problem}

\begin{subproblem}
  Show that ${\hat{\theta} = Y_{\min}}$ is sufficient for the threshold parameter~$\theta$.
\end{subproblem}
\begin{align}
  L(\theta) &= \prod_{i=1}^{n} e^{-(y_i - \theta)} \mathbbm{1}\sbrack{y_i \geq \theta} \label{eq:HW04:P04:Indicator}\\
            &= \underbrace{\bigg(e^{n\theta} \mathbbm{1}\sbrack{Y_{\min} \geq \theta} \bigg)}_{p_{\hat{\theta}}(Y_{\min};\theta)} \underbrace{\left( e^{-\sum_{i=1}^{n} y_i}  \right)}_{b(Y_1,\ldots,Y_{n})}
\end{align}

\begin{subproblem}
  Show that $Y_{\max}$ is not sufficient for~$\theta$.
\end{subproblem}

Returning to Eq.~\ref{eq:HW04:P04:Indicator}, it is clear that the indicator function~$\mathbbm{1}\sbrack{\ldots}$ terms must be a component in~$p_{\hat{\theta}}$ since each indicator contains~$\theta$ and~$y_i$.  However, it is impossible to formulate the indicators in terms of~$Y_{\max}$ as it was possible for~${Y_{\min}}$.  Hence it is not sufficient by Theorem~5.6.1.

  % \newpage
  % \begin{problem}
  \probNum{5.7.01} How large a sample must be taken from a normal pdf where ${\expect{Y} = 18}$ in order to guarantee that ${\hat{\mu}_n = \bar{Y}_n = \frac{1}{n} \sum_{i=1}^{n} Y_i}$ has a 90\%~probability of lying somewhere in the interval~${\sbrack{16,20}}$?  Assume that ${\sigma = 5.0}$.
\end{problem}

To have 90\% confidence, each of the tails should have 5\%~of the probability mass.  Using the $Z$\=/table, this corresponds to ${Z \approx 1.645}$.

\begin{align}
  Z \approx 1.645 &= \frac{X - \mu}{\sigma} \\
                  &= \frac{2\sqrt{n}}{5.0} \\
                n &= \ceil{16.91} \\
                  &= \boxed{17} \text{.}
\end{align}



  % \newpage
  % \begin{problem}
  Let ${Y_1,Y_2,\ldots,Y_n}$ be a random sample of size~$n$ from a normal pdf having~${\mu = 0}$.  Show that ${S^2 = \frac{1}{n} \sum_{i=1}^{n} Y_{i}^2}$ is a consistent estimator for ${\sigma^{2} = \text{Var}(Y)}$.
\end{problem}

\begin{align}
  \expect{S^2} &= \expect{\frac{1}{n}\sum_{i=1}^{n} Y_{i}^{2}} \\
               &= \frac{1}{n} \sum_{i=1}^{n} \expect{Y_{i}^{2}} \\
               &= \frac{1}{n} \sum_{i=1}^{n} \left( \sigma^2 - \mu^2 \right) \\
               &= \frac{n\sigma^{2}}{n} & \mu = 0\\
               &= \boxed{\sigma^2}
\end{align}

  % \newpage
  % \begin{problem}
  Let ${Y_1,Y_2,\ldots,Y_n}$ be a random sample of size~$n$ from from the exponential pdf, ${f_Y(y;\lambda) = \lambda e^{-\lambda y} \text{, } y > 0}$.
\end{problem}

\begin{subproblem}
  Show that ${\hat{\lambda}_n = Y_1}$ is not consistent for~$\lambda$.
\end{subproblem}

\begin{subproblem}
  Show that ${\hat{\lambda}_n = \sum_{i=1}^{n} Y_i}$ is not a consistent for~$\lambda$.
\end{subproblem}

\end{document}
