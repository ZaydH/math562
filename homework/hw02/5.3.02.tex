\begin{problem}
  \probNum{5.3.2} The production of a nationally marketed detergent results in certain workers receiving prolonged exposures to a \textit{Bacillus subtilis} enzyme. Nineteen workers were tested to determine the effects of those exposures, if any, on various respiratory functions. One such function, air-flow rate, is measured by computing the ratio of a person’s forced expiratory volume (FEV\textsubscript{1}) to his or her vital capacity (VC). (Vital capacity is the maximum volume of air a person can exhale after taking as deep a breath as possible; FEV\textsubscript{1} is the maximum volume of air a person can exhale in one second.) In persons with no lung dysfunction, the “norm” for FEV\textsubscript{1}/VC ratios is~0.80. Based on the following data (\textsubscript{1}75), is it believable that exposure to the \textit{Bacillus subtilis} enzyme has no effect on the FEV\textsubscript{1} /VC ratio? Answer the question by constructing a 95\%~confidence interval. Assume that FEV\textsubscript{1}/VC ratios are normally distributed with ${\sigma = 0.09}$.
\end{problem}

The sample mean of this population is:

\begin{align*}
  \bar{y} = \frac{1}{n} \sum_{i=1}^{19}y_i \approx 0.766 \text{.}
\end{align*}

The 95\%~confidence interval is approximately ${0.766 \pm 1.96 * \frac{\sigma}{\sqrt{n}}}$, which equals ${0.766 \pm 0.04}$.  Therefore the upper bound of the confidence interval is approximately~0.806 meaning it is believable the enzyme has no effect.
