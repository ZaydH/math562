\documentclass{report}

\newcommand{\name}{Zayd Hammoudeh}
\newcommand{\course}{MATH562}
\newcommand{\assnName}{Homework~\#2}
\newcommand{\dueDate}{January~24,~2020}

\usepackage[margin=1in]{geometry}
\usepackage[skip=4pt]{caption}      % ``skip'' sets the spacing between the figure and the caption.
\usepackage{tikz}
\usetikzlibrary{arrows.meta,decorations.markings,shadows,positioning,calc}
\usepackage{pgfplots}               % Needed for plotting
\pgfplotsset{compat=newest}
\usepgfplotslibrary{fillbetween}    % Allow for highlighting under a curve
\usepackage{amsmath}                % Allows for piecewise functions using the ``cases'' construct
\usepackage{siunitx}                % Allows for ``S'' alignment in table to align by decimal point

\usepackage[obeyspaces,spaces]{url} % Used for typesetting with the ``path'' command
\usepackage[hidelinks]{hyperref}    % Make the cross references clickable hyperlinks
\usepackage[bottom]{footmisc}       % Prevents the table going below the footnote
\usepackage{nccmath}                % Needed in the workaround for the ``aligncustom'' environment
\usepackage{amssymb}                % Used for black QED symbol
\usepackage{bm}                     % Allows for bolding math symbols.
\usepackage{tabto}                  % Allows to tab to certain point on a line
\usepackage{float}
\usepackage{subcaption}             % Allows use of the ``subfigure'' environment
\usepackage{enumerate}              % Allow enumeration other than just numbers

\usepackage[noend]{algpseudocode}
\usepackage[Algorithm,ruled]{algorithm}
\algnewcommand\algorithmicforeach{\textbf{for each}}
\algdef{S}[FOR]{ForEach}[1]{\algorithmicforeach\ #1\ \algorithmicdo}

%---------------------------------------------------%
%     Define Distances Used for Document Margins    %
%---------------------------------------------------%

\newcommand{\hangindentdistance}{1cm}
\newcommand{\defaultleftmargin}{0.25in}
\newcommand{\questionleftmargin}{-.5in}

\setlength{\parskip}{1em}
\setlength{\oddsidemargin}{\defaultleftmargin}

%---------------------------------------------------%
%      Configure the Document Header and Footer     %
%---------------------------------------------------%

% Set up page formatting
\usepackage{todonotes}
\usepackage{fancyhdr}                   % Used for every page footer and title.
\pagestyle{fancy}
\fancyhf{}                              % Clears both the header and footer
\renewcommand{\headrulewidth}{0pt}      % Eliminates line at the top of the page.
\fancyfoot[LO]{\course\ -- \assnName}   % Left
\fancyfoot[CO]{\thepage}                % Center
\fancyfoot[RO]{\name}                   % Right

%---------------------------------------------------%
%           Define the Title Page Entries           %
%---------------------------------------------------%

\title{\textbf{\course\ -- \assnName}}
\author{\name}

%---------------------------------------------------%
% Define the Environments for the Problem Inclusion %
%---------------------------------------------------%

\usepackage{scrextend}
\newcounter{problemCount}
\setcounter{problemCount}{0}  % Reset the subproblem counter

\newcounter{subProbCount}[problemCount]   % Reset subProbCount any time problemCount changes.
\renewcommand{\thesubProbCount}{\alph{subProbCount}}  % Make it so the counter is referenced as a number

\newenvironment{problemshell}{
  \begin{addmargin}[\questionleftmargin]{0em}
    \par%
    \medskip
    \leftskip=0pt\rightskip=0pt%
    \setlength{\parindent}{0pt}
    \bfseries
  }
  {
    \par\medskip
  \end{addmargin}
}
\newenvironment{problem}
{%
  \refstepcounter{problemCount} % Increment the subproblem counter.  This must be before the exercise to ensure proper numbering of claims and lemmas.
  \begin{problemshell}
    \noindent \textit{Exercise~\#\arabic{problemCount}} \\
  }
  {
  \end{problemshell}
  %  \setcounter{subProbCount}{0} % Reset the subproblem counter
}
\newenvironment{subproblem}
{%
  \begin{problemshell}
    \refstepcounter{subProbCount} % Increment the subproblem counter
    \setlength{\leftskip}{\hangindentdistance}
    % Print the subproblem count and offset to the left
    \hspace{-\hangindentdistance}(\alph{subProbCount}) \tabto{0pt}
  }
  {
  \end{problemshell}
}

% Change interline spacing.
\renewcommand{\baselinestretch}{1.1}
\newenvironment{aligncustom}
{ \csname align*\endcsname % Need to do this instead of \begin{align*} because of LaTeX bug.
  \centering
}
{
  \csname endalign*\endcsname
}


%---------------------------------------------------%
% Define the Environments for the Problem Inclusion %
%---------------------------------------------------%

\usepackage{amsthm}       % Allows use of the ``proof'' environment.

% Number lemmas and claims using the problem count
\newtheorem{claim}{Claim}[problemCount]
\newtheorem{lemma}{Lemma}[problemCount]

%---------------------------------------------------%
%       Define commands related to managing         %
%    floats (e.g., images) across multiple pages    %
%---------------------------------------------------%

\usepackage{placeins}     % Allows \FloatBarrier

% Prevent preceding floats going to this page
\newcommand{\floatnewpage}{\FloatBarrier\newpage}

% Add the specified input file and prevent any floated figures/tables going onto the same page as new input
\newcommand{\problemFile}[1]{
  \floatnewpage
  \input{#1}
}

\newcommand{\probNum}[1]{(\textnormal{Problem: #1})}

\newcommand{\etal}{~et~al.}

% Used for including standalone docs
\usepackage{standalone}

% Imported via UltiSnips
% Unbreakable dash:
%  Words hyphened with these dashes can also be broken at other positions than the dash
%    \-/ hyphen
%    \-- en-dash
%    \--- em-dash
%    extdash unbreakable dashes
%
%  No line breaks possible at the hyphen
%    \=/ hyphen
%    \== en-dash
%    \=== em-dash
\usepackage[shortcuts]{extdash}

% Imported via UltiSnips
\usepackage{color}
\newcommand{\colortext}[2]{{\color{#1} #2}}
\newcommand{\red}[1]{\colortext{red}{#1}}
\newcommand{\blue}[1]{\colortext{blue}{#1}}
\newcommand{\green}[1]{\colortext{green}{#1}}

% Imported via UltiSnips
\usepackage{amsmath}
\DeclareMathOperator*{\argmax}{arg\,max}
\DeclareMathOperator*{\argmin}{arg\,min}
\usepackage{amsfonts}  % Used for \mathbb and \mathcal
\usepackage{amssymb}

% Imported via UltiSnips
\usepackage{mathtools} % for "\DeclarePairedDelimiter" macro
% \swapifbranches changes unstarred paired delimiters to starred and
% vice versa.  This means by default, paired delimiters have the star.
\usepackage{etoolbox}
\newcommand\swapifbranches[3]{#1{#3}{#2}}
\makeatletter
\MHInternalSyntaxOn
\patchcmd{\DeclarePairedDelimiter}{\@ifstar}{\swapifbranches\@ifstar}{}{}
\MHInternalSyntaxOff
\makeatother
% Place after swap to ensure swap star
\DeclarePairedDelimiter{\sbrack}{\lbrack}{\rbrack}
\DeclarePairedDelimiter{\floor}{\lfloor}{\rfloor}
\DeclarePairedDelimiter{\ceil}{\lceil}{\rceil}
\DeclarePairedDelimiter{\abs}{\lvert}{\rvert}
\DeclarePairedDelimiter{\norm}{\lVert}{\rVert}
\usepackage{bm}
\DeclarePairedDelimiterX\set[1]\lbrace\rbrace{#1}
\DeclarePairedDelimiterX\setbuild[2]\lbrace\rbrace{#1 \bm: #2}
\newcommand{\setint}[1]{{\sbrack{#1}}}
\newcommand{\func}[3]{{#1:#2\rightarrow#3}}
% \newcommand{\defeq}{\stackrel{\mathclap{\mbox{\tiny def}}}{=}}
\newcommand{\defeq}{\coloneqq}
\newcommand{\fedeq}{\eqqcolon}
\newcommand{\expect}[1]{\mathbb{E}\sbrack{#1}}
% Expectation with the subscript defining the distribution
\newcommand{\expectS}[2]{\mathbb{E}_{#1}\sbrack{#2}}

% Allow numbering in align*
\newcommand{\numberthis}{\addtocounter{equation}{1}\tag{\theequation}}

\newcommand{\ints}{\mathbb{Z}}
\newcommand{\nats}{\mathbb{N}}
\newcommand{\real}{\mathbb{R}}
\newcommand{\realnn}{\real_{{\geq}0}}  % Set of non-negative real numbers

\newcommand{\iidsim}{\stackrel{\mathclap{\mbox{\tiny i.i.d.}}}{\sim}}

\newcommand{\normaldist}[2]{{\mathcal{N}\mathopen{}\left(#1,#2\right)\mathclose{}}}

% Imported via UltiSnips
\usepackage{array}  % Provides a way add a \centering command to a p-column
\usepackage{arydshln}  % Introduces hdashline & cdashline
\usepackage{bigdelim}
\usepackage{booktabs}
\usepackage{multirow}
\usepackage{makecell}  % Needed for multirowcell

% % Imported via UltiSnips
\usepackage{amsthm}
\newtheorem{theorem}{Theorem}[section]
% \newtheorem{corollary}{Corollary}[theorem]  % Corollary number derives from theorem
% \newtheorem{lemma}[theorem]{Lemma}  % Lemma and theorem share same counter
% \newtheorem{claim}[theorem]{Claim}  % Same numbering as lemma and theorem
% \newtheorem*{remark}{Remark}
% \newtheorem*{note}{Note}
% \newtheoremstyle{definition}  % <name>
% {3pt}   % <Space above>
% {3pt}   % <Space below>
% % {\itshape}     % <Body font>
% {\normalfont}   % <Body font>
% {}      % <Indent amount>
% {\bfseries} % <Theorem head font>
% {:}     % <Punctuation after theorem head>
% {.5em}  % <Space after theorem head>
% {}      % <Theorem head spec (can be left empty, meaning `normal')>
% \theoremstyle{definition}
% \newtheorem{definition}{Def.}[section]

% % Imported via UltiSnips
% \usepackage[noend]{algpseudocode}
\usepackage[Algorithm,ruled]{algorithm}
% \algnewcommand\algorithmicforeach{\textbf{for each}}
% \algdef{S}[FOR]{ForEach}[1]{\algorithmicforeach\ #1\ \algorithmicdo}
% \newcommand{\algin}[1]{\hspace*{\algorithmicindent} \textbf{Input} #1\\}
% \newcommand{\algin}[1]{\textbf{Input} #1}
% \newcommand{\algout}[1]{\hspace*{\algorithmicindent} \textbf{Output} #1}

% Imported via UltiSnips
\usepackage{tikz}
\usetikzlibrary{arrows,decorations.markings,shadows,positioning,calc,backgrounds,shapes}

\usepackage{pgfplots}
\pgfplotsset{compat=1.13}
\usepackage{pgfplotstable}
% \usepackage{subcaption}  % Cannot be used with subfigure

% Handle empty parameters
\usepackage{xifthen}
\newcommand{\ifempty}[3]{%
  \ifthenelse{\isempty{#1}}{#2}{#3}%
}


\begin{document}
  \maketitle

  \noindent
  \textbf{Name}: \name\\
  \textbf{Course}: \course\\
  \textbf{Assignment}: \assnName\\
  \textbf{Due Date}: \dueDate

  \noindent
  \textbf{Other Student Discussions}: I discussed the problems in this homework with the following student(s) below.  All write-ups were prepared independently.
  \vspace{-1em}
  \begin{itemize}
    \item Viet Lai
  \end{itemize}

  \newpage
  \begin{problem}
  \probNum{5.2.18} Use the method of moments to estimate~$\theta$ in the pdf:

  \begin{equation*}
    f_{Y}(y;\theta) = (\theta^{2} + \theta) y^{\theta - 1} (1 - y)\text{,} \hspace{0.5cm} 0 \leq y \leq 1
  \end{equation*}

  \noindent
  Assume that a random sample of size~$n$ has been collected.
\end{problem}

\begin{align*}
  \sum_{i=1}^{n} y_i = \bar{y} &= \expectS{Y}{f(y;\theta)} \\
                     &= \int  y(\theta^{2} + \theta) y^{\theta - 1} (1 - y) dy \\
                     &= (\theta^{2} + \theta) \int y^{\theta} - y^{\theta + 1} dy \\
                     &= (\theta^{2} + \theta) \left( \frac{y^{\theta + 1}}{\theta + 1} -\frac{y^{\theta + 2}}{\theta + 2} \right)\bigg\vert_{0}^{1} \\
                     &= (\theta^{2} + \theta) \left( \frac{1}{\theta + 1} - \frac{1}{\theta + 2} \right)\\
                     &= \left( \frac{\theta^{2} + 2\theta}{\theta + 2} - \frac{\theta^{2} + \theta}{\theta + 2} \right)\\
                     &= \frac{\theta}{\theta + 2}
\end{align*}

Rearranging the above we get:

\begin{equation}
  \theta = \boxed{\frac{2\bar{y}}{1 - \bar{y}}}.
\end{equation}

  \newpage
  \begin{problem}
  \probNum{5.2.20} Find the method of moments estimate for~$\lambda$ if a random sample of size~$n$ is taken from the exponential~pdf, ${f_Y(y;\lambda) = \lambda \exp(-\lambda y), y \geq 0}$.
\end{problem}

\begin{align*}
  \sum_{i=1}^{n} y_i = \bar{y} &= \expectS{Y}{f(y)} \\
                               &= \int \lambda y \exp\left( -\lambda y  \right) dy \\
                               &= \lambda \left(-\frac{y}{\lambda} - \frac{1}{\lambda^2}\right) \exp\left(-\lambda y\right) \bigg\vert_{0}^{\infty} \\
                               &= \frac{1}{\lambda}
\end{align*}

Therefore, ${\lambda = \boxed{\frac{1}{\lambda}}}$.

  \newpage
  \begin{problem}
  \probNum{5.2.25} Use the method of moments to derive estimates for the parameters~$r$ and~$\lambda$ in the gamma pdf:

  \begin{equation}
    f_{Y}(y;r,\lambda) = \frac{\lambda^{r}}{\Gamma(r)} y^{r-1} e^{-\lambda y}\text{,} \hspace{0.5cm}y \geq 0
  \end{equation}
\end{problem}

% \begin{align*}
%   \sum_{i=1}^{n} y_i = \bar{y} &= \int \frac{\lambda^{r}}{\Gamma(r)} y^{r} e^{-\lambda y}\\
%                                &= \frac{\lambda^{r}}{\Gamma(r)} \left(  \right) \bigg\vert_{0}^{\infty}
% \end{align*}

The mean and variance of the gamma distribution as formulated above are~$\frac{r}{\lambda}$ and~$\frac{r}{\lambda^2}$ respectively.  Therefore,

\begin{align*}
  \sum_{i=1}^{n} y_i = \bar{y}_1 &= \frac{r}{\lambda} \\
                         \lambda &= \frac{r}{\bar{y}_{1}}
\end{align*}

\begin{align*}
  \sum_{i=1}^{n} y_{i}^{2} = \bar{y}_2  &= \frac{r}{\lambda^{2}} + \frac{r^2}{\lambda^2} \\
                                        &= \frac{r\bar{y}_{1}^2}{r^2} + \frac{r^2\bar{y}_{1}^2}{r^2} \\
                                        &= \frac{\bar{y}_{1}^2}{r} + \bar{y}_{1}^2
\end{align*}

\begin{equation}
  r = \boxed{\frac{\bar{y}_{1}^{2}}{\bar{y}_{2} - \bar{y}_{1}^{2}}}
\end{equation}

\begin{equation*}
  \lambda = \boxed{\frac{\bar{y}_{1}}{\bar{y}_{2} - \bar{y}_{1}^{2}}}
\end{equation*}

  \newpage
  \begin{problem}
  \probNum{5.3.2} The production of a nationally marketed detergent results in certain workers receiving prolonged exposures to a \textit{Bacillus subtilis} enzyme. Nineteen workers were tested to determine the effects of those exposures, if any, on various respiratory functions. One such function, air-flow rate, is measured by computing the ratio of a person’s forced expiratory volume (FEV\textsubscript{1}) to his or her vital capacity (VC). (Vital capacity is the maximum volume of air a person can exhale after taking as deep a breath as possible; FEV\textsubscript{1} is the maximum volume of air a person can exhale in one second.) In persons with no lung dysfunction, the “norm” for FEV\textsubscript{1}/VC ratios is~0.80. Based on the following data (\textsubscript{1}75), is it believable that exposure to the \textit{Bacillus subtilis} enzyme has no effect on the FEV\textsubscript{1} /VC ratio? Answer the question by constructing a 95\%~confidence interval. Assume that FEV\textsubscript{1}/VC ratios are normally distributed with ${\sigma = 0.09}$.
\end{problem}

The sample mean of this population is:

\begin{align*}
  \bar{y} = \frac{1}{n} \sum_{i=1}^{19}y_i \approx 0.766 \text{.}
\end{align*}

The 95\%~confidence interval is approximately ${0.766 \pm 1.96 * \frac{\sigma}{\sqrt{n}}}$, which equals ${0.766 \pm 0.04}$.  Therefore the upper bound of the confidence interval is approximately~0.806 mean it is believable the enzyme has no effect.

  \newpage
  \begin{problem}
  \probNum{5.3.4} A physician who has a group of thirty-eight female patients aged 18 to 24 on a special diet wishes to estimate the effect of the diet on total serum cholesterol. For this group, their average serum cholesterol is 188.4 (measured in mg/100mL). Because of a large-scale government study, the physician is willing to assume that the total serum cholesterol measurements are normally distributed with standard deviation of ${\sigma = 40.7}$. Find a 95\%~confidence interval of the mean serum cholesterol of patients on the special diet. Does the diet seem to have any effect on their serum cholesterol, given that the national average for women aged 18 to 24 is 192.0?
\end{problem}

The 95\%~confidence interval is ${\bar{x} \pm 1.96\frac{\sigma}{\sqrt{n}}}$.  Substituting in this problem's values, we get:

\begin{equation*}
  188.4 \pm 1.96 \frac{40.7}{\sqrt{38}} = 188.4 \pm 12.9 \text{.}
\end{equation*}

The upper bound is therefore~201.3 so this diet does not seem to have an effect.

  \newpage
  \begin{problem}
  \probNum{5.3.12} During one of the first ``beer wars'' in the early 1980s, a taste test between Schlitz and Budweiser was the focus of a nationally broadcast TV commercial. One hundred people agreed to drink from two unmarked mugs and indicate which of the two beers they liked better; fifty-four said, ``Bud.'' Construct and interpret the corresponding 95\%~confidence interval for~$p$, the true proportion of beer drinkers who preferred Budweiser to Schlitz. How would Budweiser and Schlitz executives each have put these results in the best possible light for their respective companies?
\end{problem}

The mean value is ${\bar{p}=0.54}$.  The Bernoulli variance is ${\sigma^2 = p(1-p)}$.  Using the sample proportion we get the approximate sigma is: ${\hat{\sigma} \approx 0.50}$.  The 95\%~confidence interval is

\begin{equation*}
  \bar{x} \pm 1.96 * \frac{\sigma}{\sqrt{n}} \approx 0.54 \pm 1.96 * 0.05 \approx 0.54 \pm 0.098 \text{.}
\end{equation*}

If I worked for Budweiser marketing, I would say my beer was preferred in a nationwide taste test (essentially just reporting $\bar{x}$).  If I worked for Schlitz, I would say its flavor is as popular as Budweiser's since 50\% is within the confidence interval.

  \newpage
  \begin{problem}
  \probNum{5.3.13} The Pew Research Center did a survey of 2,253~adults and discovered that 63\%~of them had broadband Internet connections in their homes. The survey report noted that this figure represented a ``significant jump'' from the similar figure of 54\%~from two years earlier. One way to define ``significant jump'' is to show that the earlier number does not lie in the 95\%~confidence interval. Was the increase significant by this definition?
\end{problem}

The Bernoulli variance equals ${p(1-p)}$.  Using the sample proportion we see that for this problem, the estimated standard deviation is~0.48.  Consider the sample size and the confidence level, the confidence interval's width is ${\pm1.99\%}$.  Since the difference in proportion equals~0.09, the change is significant.

  \newpage
  \begin{problem}
  \probNum{5.4.3} Five hundred adults are asked whether they favor a bipartisan campaign finance reform bill. If the true proportion of the electorate is 52\%~in favor of the legislation, what are the chances that fewer than half of those in the sample support the proposal? Use a $Z$~transformation to approximate the answer.
\end{problem}

Given ${p=0.52}$, the standard deviation can be estimated as ${\sqrt{0.52 \cdot 0.48} \approx 0.50}$.  Normalizing for sample size yields a confidence width of ${\pm z * \frac{0.5}{\sqrt{500}} \approx \pm 0.023}$.  For a 2\%~shift, ${z=-0.894}$.  Linearly interpolating from the $Z$\-/transformation table, the chances the sample is less than~50\% is approximately~$\boxed{0.1851}$.

  \newpage
  \begin{problem}
  \probNum{5.4.5} Suppose $X_1,X_2,\ldots,X_n$ is a random sample of size~$n$ drawn from a Poisson pdf where $\lambda$ is an unknown parameter. Show that ${\hat{\lambda} = X}$ is unbiased for~$\lambda$. For what type of parameter, in general, will the sample mean necessarily be an unbiased estimator? (Hint: The answer is implicit in the derivation showing that $X$ is unbiased for the Poisson~$\lambda$.)
\end{problem}

  \newpage
  \begin{problem}
  \probNum{5.4.6} Let $Y_{\min}$ be the smallest order statistic in a random sample of size~$n$ drawn from the uniform pdf,
  \begin{equation*}
    f_Y(y;\theta) = \frac{1}{\theta}\text{, } 0 \leq y \leq \theta \text{, } \theta > 0\text{.}
  \end{equation*}
  Find an unbiased estimator for~$\theta$ based on~$Y_{\min}$.
\end{problem}

Given

\begin{equation}
  f_{Y_{\min}}(y) = n f_{Y}(y) (1 - F_{Y}(y))^{n-1}\text{,}
\end{equation}

\noindent
makes the pdf for this problem

\begin{equation}
  f_{Y_{\min}}(y) = \frac{n}{\theta} \left(1 - \frac{y}{\theta} \right)^{n-1}\text{.}
\end{equation}

Recall integration by parts specifies that

\begin{equation}
  \int uv dx = u\int v dx - \int u' \left( \int v dx \right) dx \text{.}
\end{equation}

\noindent
The expected value of $Y_{\min}$ is:

\begin{align*}
  \expect{Y_{\min}} &= \int y f_{Y_{\min}}(y) dy \\
                    &= y\int f_{Y_{\min}}(y) dy - \int \int f_{Y_{\min}}(y) dy dy \\
                    &= -y\left(1 - \frac{y}{\theta}\right)^{n}- \frac{\theta \left(1 - \frac{y}{\theta}\right)^{n+1}}{n+1} \bigg\vert_{0}^{\theta} \\
                    &= \frac{\theta}{n+1}\text{.}
 \end{align*}

 Therefore, $\boxed{(n+1)Y_{\min}}$ is an unbiased estimator of~$\theta$.

  \newpage
  \begin{problem}
  \probNum{5.4.7} Let $Y$ be the random variable described in Example 5.2.4, where ${f_{Y} (y; \theta) = e^{-(y-\theta)}\text{, }y \geq \theta\text{, }\theta > 0}$. Show that ${Y_{\min} - \frac{1}{n}}$ is an unbiased estimator of~$\theta$.
\end{problem}

Given

\begin{equation}
  f_{Y_{\min}}(y) = n f_{Y}(y) (1 - F_{Y}(y))^{n-1}\text{,}
\end{equation}

\noindent
makes the pdf for this problem

\begin{align*}
  f_{Y_{\min}}(y) &= n\exp\left(-(y-\theta)\right) \bigg(\exp\left(-(y-\theta)\right)\bigg)^{n-1} \\
                  &= n \exp\left(-n(y-\theta)\right)^{n}
\end{align*}

The expected value~$Y_{\min}$ is

\begin{align*}
  \expect{Y_{\min}} &= \int y f_{Y_{\min}}(y) dy \\
                    &= -\frac{y}{\exp\left(n(y-\theta)\right)} - \frac{\exp\left(-n(y-\theta)\right)}{n} \bigg\vert_{\theta}^{\infty} & \text{Integration by parts} \\
                    &= \theta + \frac{1}{n} & \text{L'H\^{o}pital's Rule} \text{.}
\end{align*}

\noindent
Therefore, $\boxed{Y_{\min} - \frac{1}{n}}$ is an unbiased estimator for~$\theta$.

  \newpage
  \begin{problem}
  \probNum{5.4.9} A random sample of size~2, $Y_1$ and~$Y_2$, is drawn from the pdf

  \begin{equation}
    f_Y(y;\theta) = 2y\theta^2, \hspace{0.5cm}0 < y < \frac{1}{\theta} \text{.}
  \end{equation}

  \noindent
  What must $c$~equal if the statistic ${c(Y_1 + 2Y_2)}$ is to be an unbiased estimator for~$\frac{1}{\theta}$?
\end{problem}

The expected value of $Y$ is

\begin{align}
  \expect{Y} &= \int_{0}^{\frac{1}{\theta}} 2y^2 \theta^2 dy \\
             &= \frac{2y^3 \theta^2}{3} \bigg\vert_{0}^{\frac{1}{\theta}} \\
             &= \frac{2}{3\theta} \text{.}
\end{align}

The statistic's estimated value is:

\begin{align}
  \expect{cY_1 + 2cY_2} &= c\expect{Y_1} + 2c\expect{Y_2} \\
                        &= 3c\expect{Y}\text{.}
\end{align}

For this quantity to equal $\frac{1}{\theta}$, $\boxed{c=\frac{1}{2}}$.

  \newpage
  \begin{problem}
  \probNum{5.4.17}~Let ${X_1,X_2,\ldots,X_n}$  denote the outcomes of a series of $n$~independent trials, where
  \begin{equation*}
    X_i =   \begin{cases}
              1 & \text{with probability }~p \\
              0 & \text{with probability }~1 - p
            \end{cases}
  \end{equation*}
  \noindent
  for ${i=1,2,\ldots,n}$. Let ${X = X_1 + X_2 + \cdots + X_n}$.
\end{problem}

\begin{subproblem}
  Show that $\hat{p}_1 = X_1$ and ${\hat{p}_2 = \frac{X}{n}}$ are unbiased estimators for~$p$.
\end{subproblem}

\begin{align*}
  \hat{p}_1 = \expect{X_i} &= \sum_{x \in \mathcal{X}} p_x \cdot x \\
                            &= 1 * p + 0 * (1 - p) =\boxed{p}
\end{align*}

\begin{align*}
  \hat{p}_2 = \expect{\frac{\bar{x}}{n}} &= \expect{\frac{1}{n} \left(X_1 + \cdots + X_n\right)} \\
                                          &= \frac{1}{n} \sum_{i=1}^{n} \expect{X_i} \\
                                          &= \frac{1}{n} np = \boxed{p}
\end{align*}

\begin{subproblem}
  Intuitively, $\hat{p}_2$ is a better estimator than $\hat{p}_1$ because $\hat{p}_1$ fails to include any of the information about the parameter contained in trials~2 through~$n$. Verify that speculation by comparing the variances of~$\hat{p}_1$ and~$\hat{p}_2$.
\end{subproblem}

$\var{X_1} = p(1-p)$ (Variance of Bernoulli)

\begin{align}
  \var{\frac{\bar{x}}{n}} &= n \var{\frac{X_i}{n}} \\
                          &= \frac{1}{n} \text{Var}\left(X_i\right) \\
                          &= \frac{p(1-p)}{n} \text{.}
\end{align}

For ${n > 1}$, ${\frac{p(1-p)}{n} < p(1-p)}$.

  \newpage
  \begin{problem}
  \probNum{5.4.19}~Let ${Y_1,Y_2,\ldots,Y_n}$ be a random sample size of size~$n$ from the pdf
  \begin{equation}
    f_Y(y;\theta) = \frac{1}{\theta}e^{-y\theta}\text{, }y > 0 \text{.}
  \end{equation}
\end{problem}

\begin{subproblem}
  Show that ${\thhat{1} = Y_1}$, ${\thhat{2} = \bar{Y}}$, and ${\thhat{3} = nY_{\min}}$ are all unbiased estimators for~$\theta$.
\end{subproblem}

\begin{align}
  \expect{Y_1} &= \int_{0}^{\infty} \frac{y}{\theta} e^{-y/\theta}dy \\
               &= e^{-y/\theta} \bigg\vert_{0}^{\theta} & \text{Integration by Parts}\\
               &= \theta
\end{align}

\begin{align}
  \expect{\bar{y}} &= \frac{1}{n} \sum_{i=1}^{n} \expect{Y_i} \\
                   &= \frac{1}{n} n\theta = \theta
\end{align}

\begin{align}
  f_{Y_{\min}}(y) &= \frac{n}{\theta}e^{-y/\theta}\left(e^{-y/\theta}\right)^{n-1} \\
  \expect{Y_{\min}} &= \int \frac{yn}{\theta} e^{-yn/\theta} dy \\
                    &= \int e^{-yn/\theta} dy & \text{Integration by parts} \\
                    &= \frac{\theta}{n}
\end{align}
Therefore, ${\expect{\thhat{3}} = n\expect{Y_{\min}} = \boxed{\theta}}$.

\begin{subproblem}
  Find the variances of~$\thhat{1}$, $\thhat{2}$, and~$\thhat{3}$.
\end{subproblem}

\begin{align}
  \expect{Y^2} &= \int_{0}^{\infty} \frac{y^2}{\theta} e^{-y/\theta}dy \\
               &= \int_{0}^{\infty} 2y e^{-y/\theta}dy & \text{Integration by parts}\\
               &= \int_{0}^{\infty} 2\theta e^{-y/\theta}dy & \text{Integration by parts}\\
               &= 2\theta^{2}
\end{align}
Therefore, ${\boxed{\thhat{1} = \theta^2}}$.

\begin{align}
  \var{\thhat{2}} &= \sum_{i=1}^{n} \var{Y_i /n} \\
                  &= \boxed{\frac{1}{n}\theta^{2}}\text{.}
\end{align}

\begin{align}
  f_{Y_{\min}}(y) &= \frac{1}{n\theta}e^{-y/\theta}\left(e^{-y/\theta}\right)^{n-1} \\
  \expect{Y_{\min}^2} &= \int \frac{y^2}{n\theta} e^{-yn/\theta} dy \\
                    &= \int 2y e^{-yn/\theta} dy & \text{Integration by parts} \\
                    &= \int \frac{2\theta}{n} e^{-yn/\theta} dy & \text{Integration by parts} \\
                    &=  \frac{2\theta^{2}}{n^{2}} & \text{Integration by parts}
\end{align}
Therefore, ${\var{\thhat{3}} = n^2 \var{Y_{\min}} = \boxed{\theta^2}}$.

\begin{subproblem}
  Calculate the relative efficiencies of~$\thhat{1}$ to~$\thhat{3}$ and $\thhat{2}$ to~$\thhat{3}$.
\end{subproblem}

\begin{align}
  \text{Efficiency }\thhat{1} \text{ to }\thhat{3} &= \frac{\var{\thhat{3}}}{\var{\thhat{1}}} \\
                                                   &= \frac{\theta^2}{\theta^2} = \boxed{1} \\
  \text{Efficiency }\thhat{2} \text{ to }\thhat{3} &= \frac{\var{\thhat{3}}}{\var{\thhat{2}}} \\
                                                   &= \frac{\theta^2}{\theta^2 /n} = \boxed{n}
\end{align}

\end{document}
