\documentclass{report}

\newcommand{\name}{Zayd Hammoudeh}
\newcommand{\course}{MATH562}
\newcommand{\assnName}{Homework~\#8}
\newcommand{\dueDate}{March~13,~2020}

\usepackage[margin=1in]{geometry}
\usepackage[skip=4pt]{caption}      % ``skip'' sets the spacing between the figure and the caption.
\usepackage{tikz}
\usetikzlibrary{arrows.meta,decorations.markings,shadows,positioning,calc}
\usepackage{pgfplots}               % Needed for plotting
\pgfplotsset{compat=newest}
\usepgfplotslibrary{fillbetween}    % Allow for highlighting under a curve
\usepackage{amsmath}                % Allows for piecewise functions using the ``cases'' construct
\usepackage{bbm}                    % Enables \mathbbm{1}
\usepackage{siunitx}                % Allows for ``S'' alignment in table to align by decimal point

\usepackage[obeyspaces,spaces]{url} % Used for typesetting with the ``path'' command
\usepackage[hidelinks]{hyperref}    % Make the cross references clickable hyperlinks
\usepackage[bottom]{footmisc}       % Prevents the table going below the footnote
\usepackage{nccmath}                % Needed in the workaround for the ``aligncustom'' environment
\usepackage{amssymb}                % Used for black QED symbol
\usepackage{bm}                     % Allows for bolding math symbols.
\usepackage{tabto}                  % Allows to tab to certain point on a line
\usepackage{float}
\usepackage{subcaption}             % Allows use of the ``subfigure'' environment
\usepackage{enumerate}              % Allow enumeration other than just numbers
\usepackage{booktabs}

\usepackage[noend]{algpseudocode}
\usepackage[Algorithm,ruled]{algorithm}
\algnewcommand\algorithmicforeach{\textbf{for each}}
\algdef{S}[FOR]{ForEach}[1]{\algorithmicforeach\ #1\ \algorithmicdo}

%---------------------------------------------------%
%     Define Distances Used for Document Margins    %
%---------------------------------------------------%

\newcommand{\hangindentdistance}{1cm}
\newcommand{\defaultleftmargin}{0.25in}
\newcommand{\questionleftmargin}{-.5in}

\setlength{\parskip}{1em}
\setlength{\oddsidemargin}{\defaultleftmargin}

%---------------------------------------------------%
%      Configure the Document Header and Footer     %
%---------------------------------------------------%

% Set up page formatting
\usepackage{todonotes}
\usepackage{fancyhdr}                   % Used for every page footer and title.
\pagestyle{fancy}
\fancyhf{}                              % Clears both the header and footer
\renewcommand{\headrulewidth}{0pt}      % Eliminates line at the top of the page.
\fancyfoot[LO]{\course\ -- \assnName}   % Left
\fancyfoot[CO]{\thepage}                % Center
\fancyfoot[RO]{\name}                   % Right

%---------------------------------------------------%
%           Define the Title Page Entries           %
%---------------------------------------------------%

\title{\textbf{\course\ -- \assnName}}
\author{\name}

%---------------------------------------------------%
% Define the Environments for the Problem Inclusion %
%---------------------------------------------------%

\usepackage{scrextend}
\newcounter{problemCount}
\setcounter{problemCount}{0}  % Reset the subproblem counter

\newcounter{subProbCount}[problemCount]   % Reset subProbCount any time problemCount changes.
\renewcommand{\thesubProbCount}{\alph{subProbCount}}  % Make it so the counter is referenced as a number

\newenvironment{problemshell}{
  \begin{addmargin}[\questionleftmargin]{0em}
    \par%
    \medskip
    \leftskip=0pt\rightskip=0pt%
    \setlength{\parindent}{0pt}
    \bfseries
  }
  {
    \par\medskip
  \end{addmargin}
}
\newenvironment{problem}
{%
  \refstepcounter{problemCount} % Increment the subproblem counter.  This must be before the exercise to ensure proper numbering of claims and lemmas.
  \begin{problemshell}
    \noindent \textit{Exercise~\#\arabic{problemCount}} \\
  }
  {
  \end{problemshell}
  %  \setcounter{subProbCount}{0} % Reset the subproblem counter
}
\newenvironment{subproblem}
{%
  \begin{problemshell}
    \refstepcounter{subProbCount} % Increment the subproblem counter
    \setlength{\leftskip}{\hangindentdistance}
    % Print the subproblem count and offset to the left
    \hspace{-\hangindentdistance}(\alph{subProbCount}) \tabto{0pt}
  }
  {
  \end{problemshell}
}

% Change interline spacing.
\renewcommand{\baselinestretch}{1.1}
\newenvironment{aligncustom}
{ \csname align*\endcsname % Need to do this instead of \begin{align*} because of LaTeX bug.
  \centering
}
{
  \csname endalign*\endcsname
}


%---------------------------------------------------%
% Define the Environments for the Problem Inclusion %
%---------------------------------------------------%

\usepackage{amsthm}       % Allows use of the ``proof'' environment.

% Number lemmas and claims using the problem count
\newtheorem{claim}{Claim}[problemCount]
\newtheorem{lemma}{Lemma}[problemCount]

%---------------------------------------------------%
%       Define commands related to managing         %
%    floats (e.g., images) across multiple pages    %
%---------------------------------------------------%

\usepackage{placeins}     % Allows \FloatBarrier

% Prevent preceding floats going to this page
\newcommand{\floatnewpage}{\FloatBarrier\newpage}

% Add the specified input file and prevent any floated figures/tables going onto the same page as new input
\newcommand{\problemFile}[1]{
  \floatnewpage
  \input{#1}
}

\newcommand{\probNum}[1]{(\textnormal{Problem: #1})}

\newcommand{\etal}{~et~al.}

% Used for including standalone docs
\usepackage{standalone}

% Imported via UltiSnips
% Unbreakable dash:
%  Words hyphened with these dashes can also be broken at other positions than the dash
%    \-/ hyphen
%    \-- en-dash
%    \--- em-dash
%    extdash unbreakable dashes
%
%  No line breaks possible at the hyphen
%    \=/ hyphen
%    \== en-dash
%    \=== em-dash
\usepackage[shortcuts]{extdash}

% Imported via UltiSnips
\usepackage{color}
\newcommand{\colortext}[2]{{\color{#1} #2}}
\newcommand{\red}[1]{\colortext{red}{#1}}
\newcommand{\blue}[1]{\colortext{blue}{#1}}
\newcommand{\green}[1]{\colortext{green}{#1}}

% Imported via UltiSnips
\usepackage{amsmath}
\DeclareMathOperator*{\argmax}{arg\,max}
\DeclareMathOperator*{\argmin}{arg\,min}
\usepackage{amsfonts}  % Used for \mathbb and \mathcal
\usepackage{amssymb}

% Imported via UltiSnips
\usepackage{mathtools} % for "\DeclarePairedDelimiter" macro
% \swapifbranches changes unstarred paired delimiters to starred and
% vice versa.  This means by default, paired delimiters have the star.
\usepackage{etoolbox}
\newcommand\swapifbranches[3]{#1{#3}{#2}}
\makeatletter
\MHInternalSyntaxOn
\patchcmd{\DeclarePairedDelimiter}{\@ifstar}{\swapifbranches\@ifstar}{}{}
\MHInternalSyntaxOff
\makeatother
% Place after swap to ensure swap star
\DeclarePairedDelimiter{\sbrack}{\lbrack}{\rbrack}
\DeclarePairedDelimiter{\floor}{\lfloor}{\rfloor}
\DeclarePairedDelimiter{\ceil}{\lceil}{\rceil}
\DeclarePairedDelimiter{\abs}{\lvert}{\rvert}
\DeclarePairedDelimiter{\norm}{\lVert}{\rVert}
\usepackage{bm}
\DeclarePairedDelimiterX\set[1]\lbrace\rbrace{#1}
\DeclarePairedDelimiterX\setbuild[2]\lbrace\rbrace{#1 \bm: #2}
\newcommand{\setint}[1]{{\sbrack{#1}}}
\newcommand{\func}[3]{{#1:#2\rightarrow#3}}
% \newcommand{\defeq}{\stackrel{\mathclap{\mbox{\tiny def}}}{=}}
\newcommand{\defeq}{\coloneqq}
\newcommand{\fedeq}{\eqqcolon}
\newcommand{\expect}[1]{\mathbb{E}\sbrack{#1}}
% Expectation with the subscript defining the distribution
\newcommand{\expectS}[2]{\mathbb{E}_{#1}\sbrack{#2}}

% Allow numbering in align*
\newcommand{\numberthis}{\addtocounter{equation}{1}\tag{\theequation}}

\newcommand{\ints}{\mathbb{Z}}
\newcommand{\nats}{\mathbb{N}}
\newcommand{\real}{\mathbb{R}}
\newcommand{\realnn}{\real_{{\geq}0}}  % Set of non-negative real numbers

\newcommand{\iidsim}{\stackrel{\mathclap{\mbox{\tiny i.i.d.}}}{\sim}}

\newcommand{\normaldist}[2]{{\mathcal{N}\mathopen{}\left(#1,#2\right)\mathclose{}}}

% Imported via UltiSnips
\usepackage{array}  % Provides a way add a \centering command to a p-column
\usepackage{arydshln}  % Introduces hdashline & cdashline
\usepackage{bigdelim}
\usepackage{booktabs}
\usepackage{multirow}
\usepackage{makecell}  % Needed for multirowcell

% % Imported via UltiSnips
\usepackage{amsthm}
\newtheorem{theorem}{Theorem}[section]
% \newtheorem{corollary}{Corollary}[theorem]  % Corollary number derives from theorem
% \newtheorem{lemma}[theorem]{Lemma}  % Lemma and theorem share same counter
% \newtheorem{claim}[theorem]{Claim}  % Same numbering as lemma and theorem
% \newtheorem*{remark}{Remark}
% \newtheorem*{note}{Note}
% \newtheoremstyle{definition}  % <name>
% {3pt}   % <Space above>
% {3pt}   % <Space below>
% % {\itshape}     % <Body font>
% {\normalfont}   % <Body font>
% {}      % <Indent amount>
% {\bfseries} % <Theorem head font>
% {:}     % <Punctuation after theorem head>
% {.5em}  % <Space after theorem head>
% {}      % <Theorem head spec (can be left empty, meaning `normal')>
% \theoremstyle{definition}
% \newtheorem{definition}{Def.}[section]

% % Imported via UltiSnips
% \usepackage[noend]{algpseudocode}
\usepackage[Algorithm,ruled]{algorithm}
% \algnewcommand\algorithmicforeach{\textbf{for each}}
% \algdef{S}[FOR]{ForEach}[1]{\algorithmicforeach\ #1\ \algorithmicdo}
% \newcommand{\algin}[1]{\hspace*{\algorithmicindent} \textbf{Input} #1\\}
% \newcommand{\algin}[1]{\textbf{Input} #1}
% \newcommand{\algout}[1]{\hspace*{\algorithmicindent} \textbf{Output} #1}

% Imported via UltiSnips
\usepackage{tikz}
\usetikzlibrary{arrows,decorations.markings,shadows,positioning,calc,backgrounds,shapes}

\usepackage{pgfplots}
\pgfplotsset{compat=1.13}
\usepackage{pgfplotstable}
% \usepackage{subcaption}  % Cannot be used with subfigure

% Handle empty parameters
\usepackage{xifthen}
\newcommand{\ifempty}[3]{%
  \ifthenelse{\isempty{#1}}{#2}{#3}%
}


\newcommand{\thhat}[1]{\hat{\theta}_{#1}}
\newcommand{\var}[1]{\text{Var}\left(#1\right)}

\begin{document}
  \maketitle

  \noindent
  \textbf{Name}: \name\\
  \textbf{Course}: \course\\
  \textbf{Assignment}: \assnName\\
  \textbf{Due Date}: \dueDate

  \noindent
  \textbf{Other Student Discussions}: I discussed the problems in this homework with the following student(s) below.  All write-ups were prepared independently.
  \vspace{-1em}
  \begin{itemize}
    \item Viet Lai
  \end{itemize}

  \newpage
  \begin{problem}
  \probNum{9.2.1}~Some states that operate a lottery believe that restricting the use of lottery profits to supporting education makes the lottery more profitable. Other states permit general use of lottery income.  The profitability of the lottery for a group of states in each category is given below.

  Test at the ${\alpha = 0.01}$~level whether the mean profit of states using the lottery for education is higher than that of states permitting general use. Assume that the variances of the two random variables are equal.
\end{problem}

Define $X$~as the education states, and $Y$~is the general use states.

\noindent
${H_0: \mu_{X} = \mu_{Y}}$ \\
${H_0: \mu_{X} > \mu_{Y}}$

Since ${\sigma_{X} = \sigma_{Y}}$, we can use the property:
\begin{equation}
  t = \frac{\bar{X} - \bar{Y}}{S_{p} \sqrt{\frac{1}{n} + \frac{1}{m}}}
\end{equation}
\noindent
where
\begin{equation}
  S^{2}_{p} = \frac{(n-1)S^{2}_{X} + (m-1)S^{2}_{Y}}{n + m - 2}\text{.}
\end{equation}

${\bar{X} = 29.83}$ and ${\bar{Y} = 26.89}$. ${S^{2}_{X} = 13.78}$ and ${S^{2}_{Y} = 15.61}$.

This makes the sample variance:
\begin{align}
  S^{2}_{p} &= \frac{(12 - 1) * 13.78 + (9 - 1) * 15.61}{12 + 9 - 2} \\
            &= \frac{276.46}{19}
            &= 14.55 \text{.}
\end{align}
This makes:
\begin{align}
  t &=  \frac{29.83 - 26.89}{3.81 \sqrt{\frac{1}{12} + \frac{1}{9}}} \\
    &= 1.748\text{.}
\end{align}
\noindent
${t_{0.01,19} = 2.539}$ so the null hypothesis is \underline{not rejected}.

  \newpage
  \begin{problem}
  \probNum{9.2.2}~As the United States has struggled with the growing obesity of its citizens, diets have become big business. Among the many competing regimens for those seeking weight reduction are the Atkins and Zone diets. In a comparison of these two diets for one-year weight loss, a study found that seventy\=/seven subjects on the Atkins diet had an average weight loss of ${\bar{x} = -4.7\text{kg}}$ and a sample standard deviation of~${s_X =7.05\text{kg}}$. Similar figures for the seventy-nine people on the Zone diet were~${\bar{y} = -1.6\text{kg}}$ and ${s_Y = 5.36\text{kg}}$. Is the greater reduction with the Atkins diet statistically significant? Test for ${\alpha = 0.05}$.
\end{problem}

\noindent
${H_0: \mu_{X} = \mu_{Y}}$ \\
${H_1: \mu_{X} < \mu_{Y}}$

\begin{align}
  W &= \frac{\bar{X} - \bar{Y} - (\mu_{X} - \mu_{Y})}{\sqrt{\frac{S_{X}^{2}}{n} + \frac{S_{Y}^{2}}{m}}} \\
    &= \frac{-4.7 - (-1.6) - (0)}{\sqrt{\frac{7.05^2}{77}+\frac{5.36^2}{79}}} \\
    &= -3.086\text{.}
\end{align}

${\hat{\theta} = \frac{7.05^2}{5.36^2} = 1.730}$.  This makes the degrees of freedom estimate
\begin{align}
  \nu &= \frac{(\hat{\theta} + \frac{n}{m})^2}{\frac{1}{n-1}\hat{\theta}^2 + \frac{1}{m-1}\left(\frac{n}{m}\right)^2} \\
      &= \frac{\left(1.730 + \frac{77}{79}\right)^2}{\frac{1}{76} 1.730^2 + \frac{1}{78}\left(\frac{77}{79}\right)^2} \\
      &= 141.88 \text{.}
\end{align}

${t_{0.05,100} = 1.6602}$ (with the value even smaller for more degrees of freedom.  ${3.086 > 1.6602}$~so  the result \underline{is statistically significant}.

  \newpage
  \begin{problem}
  \probNum{9.2.4}~Among a number of beliefs concerning the so\=/called \textit{lunar effect} is that more children are born during the full moon.  The table below gives the average number of births per day during the full moon (lunar faction~${{\geq}0.95}$) versus the average number during a period of lunar fraction~${{\leq} 0.75}$. The table shows the opposite effect.  The average is smaller during the full moon. But is this a significant difference? Test the equality of means at the 0.05~level of significance.  Assume the variances are equal.
\end{problem}

\noindent
${H_0: \mu_{X} = \mu_{Y}}$ \\
${H_1: \mu_{X} \ne \mu_{Y}}$

\begin{align}
  S_{p} &= \sqrt{\frac{(n-1)S^{2}_{X} + (m-1)S^{2}_{Y}}{n + m - 2}} \\
        &= \sqrt{\frac{108 \cdot 2017^2 + 493 \cdot 1897^2}{109 + 494 - 2}} \\
        &= 1919.12
\end{align}

\begin{align}
  t &= \frac{\bar{X} - \bar{Y}}{S_{p} \sqrt{\frac{1}{n} + \frac{1}{m}}} \\
    &= \frac{10732 - 10970}{1919.12 \sqrt{\frac{1}{109} + \frac{1}{494}}} \\
    &= -1.17
\end{align}
Since the sample size is large, the $Z$\=/score can be used making ${Z_{0.025} = 1.96}$.  Since ${1.17 < 1.96}$, the null hypothesis is \underline{not rejected}.

  \newpage
  \begin{problem}
  \probNum{9.2.17}~Poverty Point is the name given to a number of widely scattered archaeological sites throughout Louisiana, Mississippi, and Arkansas.  These are the remains of a society thought to have flourished during the period from~1700 to~500~B.C.\ Among their characeristic artifacts are ornaments that were fashioned out of clay and then baked.  The following table shows the dates (in years~BC) associated with four of these baked clay ornaments found in two different Poverty Point sites, Terral Lewis and Jaketown.  The averages for the two samples are~1133.0 and~1013.5, respectively.  Is it believable that these two settlements developed the technology to manufacture baked clay ornaments at the same time?  Set up and test an appropriate~$H_0$ against a two-side~$H_1$ at the ${\alpha = 0.05}$\%~level of significance.  For these data,~${s_{X} = 266.9}$ and~${s_Y = 224.3}$.
\end{problem}

\noindent
${H_0: \mu_{X} = \mu_{Y}}$ \\
${H_1: \mu_{X} \ne \mu_{Y}}$

${\mu_{X} = 1133}$ and

  \newpage
  \begin{problem}
  \probNum{9.3.3}~Among the standard personality inventories used by psychologists is the thematic apperception test (TAT) in which a subject is shown a series of pictures and is asked to make up a story about each one. Interpreted properly, the content of the stories can provide valuable insights into the subject’s mental well-being. The following data show the TAT results for forty women, twenty of whom were the mothers of normal children and twenty the mothers of schizophrenic children. In each case the subject was shown the same set of ten pictures. The figures recorded were the numbers of stories (out of ten) that revealed a positive parent-child relationship, one where the mother was clearly capable of interacting with her child in a flexible, open-minded way.
\end{problem}

\begin{subproblem}
  Test~${H_0 : \sigma_{X}^{2} = \sigma_{Y}^{2}}$ versus~${H_1 : \sigma_{X}^{2} \ne \sigma_{Y}^{2}}$, where~$\sigma_{X}^2$ and~$\sigma_{Y}^{2}$ are the variances of the scores of mothers of normal children and scores of mothers of schizophrenic children, respectively. Let~${\alpha = 0.05}$.
\end{subproblem}

${S^{2}_{X} = 3.524}$ and~${S^{2}_Y = 2.411}$.  ${m = n = 20}$.

The $F$\=/table does not have values for ${m = n = 20}$ so ${m = n = 21}$ is used. ${F_{0.025,20,20} = 0.406}$ and ${F_{0.975,20,20} = 2.46}$\text{.}  ${\frac{S^{2}_{X}}{S^{2}_{Y}} = \frac{3.524}{2.411} = 1.416}$.  Since ${F_{0.025,20,20} < \frac{S^{2}_{X}}{S^{2}_{Y}} < F_{0.975,20,20}}$ so the hypothesis is \underline{not rejected}.

\begin{subproblem}
  If~${H_0: \sigma_{X}^{2} = \sigma_{Y}^{2}}$ is accepted in part~(a), test ${\mu_{X} = \mu_{Y}}$ versus~${H_{1}: \mu_{X} \ne \mu_{Y}}$. Set $\alpha$~equal to~0.05.
\end{subproblem}

${\bar{X} = 3.55}$ and ${\bar{Y} = 2.1}$.  ${m = n = 20}$.  We proceed with the assumption of identical variance so:
\begin{align}
  S^{2}_{p} &= \frac{(n-1)S^{2}_{X} + (m-1)S^{2}_{Y}}{n + m - 2} \\
            &= \frac{(20 - 1) * 3.524+ (20 - 1) * 2.411}{20 + 20 - 2} \\
            &= 2.967 \text{.}
\end{align}

This makes:
\begin{align}
  t &= \frac{\bar{X} - \bar{Y}}{S_{p} \sqrt{\frac{1}{n} + \frac{1}{m}}} \\
    &=  \frac{3.55 - 2.1}{\sqrt{2.967} \sqrt{\frac{1}{20} + \frac{1}{20}}} \\
    &= 2.662\text{.}
\end{align}

${t_{0.025,38} = 2.0244}$.  The hypothesis is \underline{rejected}.

  \newpage
  \begin{problem}
  \probNum{9.3.4}~In a study designed to investigate the effects of a strong magnetic field on the early development of mice, ten cages, each containing three 30-day-old albino female mice, were subjected for a period of 12~days to a magnetic field having an average strength of 80~Oe/cm. Thirty other mice, housed in ten similar cages, were not put in the magnetic field and served as controls. Listed in the table are the weight gains, in grams, for each of the twenty sets of mice.

Test whether the variances of the two sets of weight gains are significantly different. Let~${\alpha = 0.05}$. For the mice in the magnetic field,~${s_X = 5.67}$; for the other mice,~${s_Y = 3.18}$.
\end{problem}

\noindent
${H_0: \mu_{X} = \mu_{Y}}$ \\
${H_1:  \mu_{X} \ne \mu_{Y}}$

\noindent
${n = m = 10}$  ${\frac{s^{2}_{X}}{s^{2}_{Y}} = \frac{5.67^2}{3.18^2} = 3.179}$.

\noindent
${F_{0.025,9,9} = 0.248}$ and~${F_{0.975,9,9} = 4.03}$.  ${0.248 \leq 3.179 \leq 4.03}$ so the hypothesis is \underline{not rejected}.

  \newpage
  \begin{problem}
  \probNum{9.5.1}~Historically, fluctuations in the amount of rare metals found in coins are not uncommon (82). The following data may be a case in point. Listed are the silver percentages found in samples of a Byzantine coin minted on two separate occasions during the reign of Manuel~I~(1143\=/1180). Construct a 90\%~confidence interval for ${\mu_{X} - \mu_{Y}}$, the true average difference in the coin’s silver content (= ``early'' - ``late''). What does the interval imply about the outcome of testing ${H_0: \mu_{X} = \mu_{Y}}$? For these data, ${s_{X} = 0.54}$ and~${s_{Y} = 0.36}$.
\end{problem}

\noindent
Assuming equal variance. ${n = 9}$ and~${m = 7}$.
\begin{align}
  S_{p} &= \sqrt{\frac{(n-1)S^{2}_{X} + (m-1)S^{2}_{Y}}{n + m - 2}} \\
        &= \sqrt{\frac{8 \cdot 0.54^2 + 6 \cdot 0.36^2}{9 + 7 - 2}} \\
        &= 0.4471\text{.}
\end{align}

${t_{0.05,13} = 1.7709}$.  ${\mu_{X} = 6.7}$ and~${\mu_{Y} = 5.6}$. The confidence interval is then:
\begin{align}
  \sbrack{\bar{x} - \bar{y} - t_{\alpha/2,n+m-2} \cdot s_p\sqrt{\frac{1}{n} + \frac{1}{m}},\bar{x} - \bar{y} + t_{\alpha/2,n+m-2} \cdot s_p\sqrt{\frac{1}{n} + \frac{1}{m}}} \\
  \sbrack{6.7 - 5.6 - 1.7709 \cdot 0.471 \sqrt{\frac{1}{9} + \frac{1}{7}}, 6.7 - 5.6+ 1.7709 \cdot 0.471 \sqrt{\frac{1}{9} + \frac{1}{7}}} \\
  \sbrack{1.1 - 0.4203, 1.1 + 0.4203} \\
  \sbrack{0.680, 1.520}
\end{align}

This indicates with at least 95\% certainty the means are different.

  \newpage
  \begin{problem}
  \probNum{9.5.2}~Male fiddler crabs solicit attention from the opposite sex by standing in front of their burrows and waving their claws at the females who walk by. If a female likes what she sees, she pays the male a brief visit in his burrow. If everything goes well and the crustacean chemistry clicks, she will stay a little longer and mate. In what may be a ploy to lessen the risk of spending the night alone, some of the males build elaborate mud domes over their burrows. Do the following data suggest that a male's time spent waving to females is influenced by whether his burrow has a dome? Answer the question by constructing and interpreting a 95\%~confidence interval for~${\mu_{X} - \mu_{Y}}$. Use the value~${s_p = 11.2}$.
\end{problem}

\noindent
${\bar{X} = 83.96}$ and~${\bar{Y} = 84.84}$. ${n = 5}$ and~${m = 7}$.

\noindent
${t_{0.025,10} = 2.2281}$.  This makes the confidence interval:
\begin{align}
  \sbrack{\bar{x} - \bar{y} - t_{\alpha/2,n+m-2} \cdot s_p\sqrt{\frac{1}{n} + \frac{1}{m}},\bar{x} - \bar{y} + t_{\alpha/2,n+m-2} \cdot s_p\sqrt{\frac{1}{n} + \frac{1}{m}}} \\
  \sbrack{83.96 - 84.84 - 2.2281 \cdot 11.2 \sqrt{\frac{1}{5} + \frac{1}{7}}, 83.96 - 84.84 + 2.2281 \cdot 11.2 \sqrt{\frac{1}{5} + \frac{1}{7}}} \\
  \sbrack{-0.88 - 14.612, -0.88 + 14.612} \\
  \sbrack{-15.492, 13.732}
\end{align}

\noindent
${\mu_{X} - \mu_{Y} = 0}$ is in the confidence interval so the hypothesis is \underline{not rejected}.

  \newpage
  \begin{problem}
  \probNum{9.5.6}~Construct a 95\%~confidence interval for ${\sigma_{X}^2 / \sigma_{Y}^2}$ based on the data in Case Study~9.2.1.  The hypothesis test referred to tacitly assumed that the variances were equal.  Does that agree with your confidence interval? Explain.
\end{problem}

\noindent
Per the case study, the two sample variances are: ${s_{X}^2 = 2.103 \cdot 10^{-4}}$ and~${s_{Y}^2 = 9.55 \cdot 10^{-5}}$. ${n = 8}$ and~${m = 10}$.

${F_{0.025,7,9} = 0.238}$ and~${F_{0.975,7,9} = 4.82}$. The confidence interval of the ratio~$\frac{\sigma_{X}^2}{\sigma_{Y}^2}$ is defined as:
\begin{align}
  \left(\frac{s_{X}^2}{s_{Y}^2} F_{\alpha/2,m-1,n-1}, \frac{s_{X}^2}{s_{Y}^2} F_{1-\alpha/2,m-1,n-1} \right) \\
  \left(2.202 \cdot 0.238, 2.202 \cdot 4.82 \right) \\
  \left(0.524,10.614\right) \text{.}
\end{align}

If the variances are equal then the ratio equals~1, which is within the confidence interval.  Therefore, the null hypothesis that the variances are equal is reasonable.

\end{document}
