\begin{problem}
  \probNum{9.2.17}~Poverty Point is the name given to a number of widely scattered archaeological sites throughout Louisiana, Mississippi, and Arkansas.  These are the remains of a society thought to have flourished during the period from~1700 to~500~B.C.\ Among their characeristic artifacts are ornaments that were fashioned out of clay and then baked.  The following table shows the dates (in years~BC) associated with four of these baked clay ornaments found in two different Poverty Point sites, Terral Lewis and Jaketown.  The averages for the two samples are~1133.0 and~1013.5, respectively.  Is it believable that these two settlements developed the technology to manufacture baked clay ornaments at the same time?  Set up and test an appropriate~$H_0$ against a two-side~$H_1$ at the ${\alpha = 0.05}$\%~level of significance.  For these data,~${s_{X} = 266.9}$ and~${s_Y = 224.3}$.
\end{problem}

\noindent
${H_0: \mu_{X} = \mu_{Y}}$ \\
${H_1: \mu_{X} \ne \mu_{Y}}$

${\mu_{X} = 1133}$ and
