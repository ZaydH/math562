\begin{problem}
  \probNum{9.2.17}~A person exposed to an infectious agent, either by contact or by vaccination, normally develops antibodies to that agent. Presumably, the severity of an infection is related to the number of antibodies produced. The degree of antibody response is indicated by saying that the person’s blood serum has a certain titer, with higher titers indicating greater concentrations of antibodies. The following table gives the titers of twenty-two persons involved in a tularemia epidemic in Vermont. Eleven were quite ill; the other eleven were asymptomatic. Use Theorem~9.2.3 to test ${H_0: \mu_X = \mu_Y}$ against a one\=/sided~${H_1}$ at the 0.05~level of significance.

  The sample standard deviations for the “Severely Ill” and “Asymptomatic” groups are 428 and 183, respectively.
\end{problem}

\noindent
${H_0: \mu_{X} = \mu_{Y}}$ \\
${H_0: \mu_{X} > \mu_{Y}}$

\noindent
${m = n = 11}$. ${\mu_{X} = 545.45}$ and~${\mu_{Y} = 241.82}$.
\begin{align}
  W &= \frac{\bar{X} - \bar{Y} - (\mu_{X} - \mu_{Y})}{\sqrt{\frac{S_{X}^{2}}{n} + \frac{S_{Y}^{2}}{m}}} \\
    &= \frac{545.45 - 241.82 - (0)}{\sqrt{\frac{428^2}{11}+\frac{183}{11}}} \\
    &= 2.163\text{.}
\end{align}

\noindent
${\hat{\theta} = \frac{428^2}{183^2} \approx 5.470}$.  This makes
\begin{align}
  \nu &= \frac{(\hat{\theta} + \frac{n}{m})^2}{\frac{1}{n-1}\hat{\theta}^2 + \frac{1}{m-1}\left(\frac{n}{m}\right)^2} \\
      &= \frac{\left(5.470 + \frac{11}{11}\right)^2}{\frac{1}{10} 5.470^2 + \frac{1}{10}\left(\frac{11}{11}\right)^2} \\
      &= 13.538 \text{.}
\end{align}

\noindent
${t_{0.05,14} = 1.7613}$.  Therefore, the null hypothesis is \underline{rejected}.
