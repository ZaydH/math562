\documentclass{report}

\newcommand{\name}{Zayd Hammoudeh}
\newcommand{\course}{MATH562}
\newcommand{\assnName}{Homework~\#6}
\newcommand{\dueDate}{February~28,~2020}

\usepackage[margin=1in]{geometry}
\usepackage[skip=4pt]{caption}      % ``skip'' sets the spacing between the figure and the caption.
\usepackage{tikz}
\usetikzlibrary{arrows.meta,decorations.markings,shadows,positioning,calc}
\usepackage{pgfplots}               % Needed for plotting
\pgfplotsset{compat=newest}
\usepgfplotslibrary{fillbetween}    % Allow for highlighting under a curve
\usepackage{amsmath}                % Allows for piecewise functions using the ``cases'' construct
\usepackage{bbm}                    % Enables \mathbbm{1}
\usepackage{siunitx}                % Allows for ``S'' alignment in table to align by decimal point

\usepackage[obeyspaces,spaces]{url} % Used for typesetting with the ``path'' command
\usepackage[hidelinks]{hyperref}    % Make the cross references clickable hyperlinks
\usepackage[bottom]{footmisc}       % Prevents the table going below the footnote
\usepackage{nccmath}                % Needed in the workaround for the ``aligncustom'' environment
\usepackage{amssymb}                % Used for black QED symbol
\usepackage{bm}                     % Allows for bolding math symbols.
\usepackage{tabto}                  % Allows to tab to certain point on a line
\usepackage{float}
\usepackage{subcaption}             % Allows use of the ``subfigure'' environment
\usepackage{enumerate}              % Allow enumeration other than just numbers

\usepackage[noend]{algpseudocode}
\usepackage[Algorithm,ruled]{algorithm}
\algnewcommand\algorithmicforeach{\textbf{for each}}
\algdef{S}[FOR]{ForEach}[1]{\algorithmicforeach\ #1\ \algorithmicdo}

%---------------------------------------------------%
%     Define Distances Used for Document Margins    %
%---------------------------------------------------%

\newcommand{\hangindentdistance}{1cm}
\newcommand{\defaultleftmargin}{0.25in}
\newcommand{\questionleftmargin}{-.5in}

\setlength{\parskip}{1em}
\setlength{\oddsidemargin}{\defaultleftmargin}

%---------------------------------------------------%
%      Configure the Document Header and Footer     %
%---------------------------------------------------%

% Set up page formatting
\usepackage{todonotes}
\usepackage{fancyhdr}                   % Used for every page footer and title.
\pagestyle{fancy}
\fancyhf{}                              % Clears both the header and footer
\renewcommand{\headrulewidth}{0pt}      % Eliminates line at the top of the page.
\fancyfoot[LO]{\course\ -- \assnName}   % Left
\fancyfoot[CO]{\thepage}                % Center
\fancyfoot[RO]{\name}                   % Right

%---------------------------------------------------%
%           Define the Title Page Entries           %
%---------------------------------------------------%

\title{\textbf{\course\ -- \assnName}}
\author{\name}

%---------------------------------------------------%
% Define the Environments for the Problem Inclusion %
%---------------------------------------------------%

\usepackage{scrextend}
\newcounter{problemCount}
\setcounter{problemCount}{0}  % Reset the subproblem counter

\newcounter{subProbCount}[problemCount]   % Reset subProbCount any time problemCount changes.
\renewcommand{\thesubProbCount}{\alph{subProbCount}}  % Make it so the counter is referenced as a number

\newenvironment{problemshell}{
  \begin{addmargin}[\questionleftmargin]{0em}
    \par%
    \medskip
    \leftskip=0pt\rightskip=0pt%
    \setlength{\parindent}{0pt}
    \bfseries
  }
  {
    \par\medskip
  \end{addmargin}
}
\newenvironment{problem}
{%
  \refstepcounter{problemCount} % Increment the subproblem counter.  This must be before the exercise to ensure proper numbering of claims and lemmas.
  \begin{problemshell}
    \noindent \textit{Exercise~\#\arabic{problemCount}} \\
  }
  {
  \end{problemshell}
  %  \setcounter{subProbCount}{0} % Reset the subproblem counter
}
\newenvironment{subproblem}
{%
  \begin{problemshell}
    \refstepcounter{subProbCount} % Increment the subproblem counter
    \setlength{\leftskip}{\hangindentdistance}
    % Print the subproblem count and offset to the left
    \hspace{-\hangindentdistance}(\alph{subProbCount}) \tabto{0pt}
  }
  {
  \end{problemshell}
}

% Change interline spacing.
\renewcommand{\baselinestretch}{1.1}
\newenvironment{aligncustom}
{ \csname align*\endcsname % Need to do this instead of \begin{align*} because of LaTeX bug.
  \centering
}
{
  \csname endalign*\endcsname
}


%---------------------------------------------------%
% Define the Environments for the Problem Inclusion %
%---------------------------------------------------%

\usepackage{amsthm}       % Allows use of the ``proof'' environment.

% Number lemmas and claims using the problem count
\newtheorem{claim}{Claim}[problemCount]
\newtheorem{lemma}{Lemma}[problemCount]

%---------------------------------------------------%
%       Define commands related to managing         %
%    floats (e.g., images) across multiple pages    %
%---------------------------------------------------%

\usepackage{placeins}     % Allows \FloatBarrier

% Prevent preceding floats going to this page
\newcommand{\floatnewpage}{\FloatBarrier\newpage}

% Add the specified input file and prevent any floated figures/tables going onto the same page as new input
\newcommand{\problemFile}[1]{
  \floatnewpage
  \input{#1}
}

\newcommand{\probNum}[1]{(\textnormal{Problem: #1})}

\newcommand{\etal}{~et~al.}

% Used for including standalone docs
\usepackage{standalone}

% Imported via UltiSnips
% Unbreakable dash:
%  Words hyphened with these dashes can also be broken at other positions than the dash
%    \-/ hyphen
%    \-- en-dash
%    \--- em-dash
%    extdash unbreakable dashes
%
%  No line breaks possible at the hyphen
%    \=/ hyphen
%    \== en-dash
%    \=== em-dash
\usepackage[shortcuts]{extdash}

% Imported via UltiSnips
\usepackage{color}
\newcommand{\colortext}[2]{{\color{#1} #2}}
\newcommand{\red}[1]{\colortext{red}{#1}}
\newcommand{\blue}[1]{\colortext{blue}{#1}}
\newcommand{\green}[1]{\colortext{green}{#1}}

% Imported via UltiSnips
\usepackage{amsmath}
\DeclareMathOperator*{\argmax}{arg\,max}
\DeclareMathOperator*{\argmin}{arg\,min}
\usepackage{amsfonts}  % Used for \mathbb and \mathcal
\usepackage{amssymb}

% Imported via UltiSnips
\usepackage{mathtools} % for "\DeclarePairedDelimiter" macro
% \swapifbranches changes unstarred paired delimiters to starred and
% vice versa.  This means by default, paired delimiters have the star.
\usepackage{etoolbox}
\newcommand\swapifbranches[3]{#1{#3}{#2}}
\makeatletter
\MHInternalSyntaxOn
\patchcmd{\DeclarePairedDelimiter}{\@ifstar}{\swapifbranches\@ifstar}{}{}
\MHInternalSyntaxOff
\makeatother
% Place after swap to ensure swap star
\DeclarePairedDelimiter{\sbrack}{\lbrack}{\rbrack}
\DeclarePairedDelimiter{\floor}{\lfloor}{\rfloor}
\DeclarePairedDelimiter{\ceil}{\lceil}{\rceil}
\DeclarePairedDelimiter{\abs}{\lvert}{\rvert}
\DeclarePairedDelimiter{\norm}{\lVert}{\rVert}
\usepackage{bm}
\DeclarePairedDelimiterX\set[1]\lbrace\rbrace{#1}
\DeclarePairedDelimiterX\setbuild[2]\lbrace\rbrace{#1 \bm: #2}
\newcommand{\setint}[1]{{\sbrack{#1}}}
\newcommand{\func}[3]{{#1:#2\rightarrow#3}}
% \newcommand{\defeq}{\stackrel{\mathclap{\mbox{\tiny def}}}{=}}
\newcommand{\defeq}{\coloneqq}
\newcommand{\fedeq}{\eqqcolon}
\newcommand{\expect}[1]{\mathbb{E}\sbrack{#1}}
% Expectation with the subscript defining the distribution
\newcommand{\expectS}[2]{\mathbb{E}_{#1}\sbrack{#2}}

% Allow numbering in align*
\newcommand{\numberthis}{\addtocounter{equation}{1}\tag{\theequation}}

\newcommand{\ints}{\mathbb{Z}}
\newcommand{\nats}{\mathbb{N}}
\newcommand{\real}{\mathbb{R}}
\newcommand{\realnn}{\real_{{\geq}0}}  % Set of non-negative real numbers

\newcommand{\iidsim}{\stackrel{\mathclap{\mbox{\tiny i.i.d.}}}{\sim}}

\newcommand{\normaldist}[2]{{\mathcal{N}\mathopen{}\left(#1,#2\right)\mathclose{}}}

% Imported via UltiSnips
\usepackage{array}  % Provides a way add a \centering command to a p-column
\usepackage{arydshln}  % Introduces hdashline & cdashline
\usepackage{bigdelim}
\usepackage{booktabs}
\usepackage{multirow}
\usepackage{makecell}  % Needed for multirowcell

% % Imported via UltiSnips
\usepackage{amsthm}
\newtheorem{theorem}{Theorem}[section]
% \newtheorem{corollary}{Corollary}[theorem]  % Corollary number derives from theorem
% \newtheorem{lemma}[theorem]{Lemma}  % Lemma and theorem share same counter
% \newtheorem{claim}[theorem]{Claim}  % Same numbering as lemma and theorem
% \newtheorem*{remark}{Remark}
% \newtheorem*{note}{Note}
% \newtheoremstyle{definition}  % <name>
% {3pt}   % <Space above>
% {3pt}   % <Space below>
% % {\itshape}     % <Body font>
% {\normalfont}   % <Body font>
% {}      % <Indent amount>
% {\bfseries} % <Theorem head font>
% {:}     % <Punctuation after theorem head>
% {.5em}  % <Space after theorem head>
% {}      % <Theorem head spec (can be left empty, meaning `normal')>
% \theoremstyle{definition}
% \newtheorem{definition}{Def.}[section]

% % Imported via UltiSnips
% \usepackage[noend]{algpseudocode}
\usepackage[Algorithm,ruled]{algorithm}
% \algnewcommand\algorithmicforeach{\textbf{for each}}
% \algdef{S}[FOR]{ForEach}[1]{\algorithmicforeach\ #1\ \algorithmicdo}
% \newcommand{\algin}[1]{\hspace*{\algorithmicindent} \textbf{Input} #1\\}
% \newcommand{\algin}[1]{\textbf{Input} #1}
% \newcommand{\algout}[1]{\hspace*{\algorithmicindent} \textbf{Output} #1}

% Imported via UltiSnips
\usepackage{tikz}
\usetikzlibrary{arrows,decorations.markings,shadows,positioning,calc,backgrounds,shapes}

\usepackage{pgfplots}
\pgfplotsset{compat=1.13}
\usepackage{pgfplotstable}
% \usepackage{subcaption}  % Cannot be used with subfigure

% Handle empty parameters
\usepackage{xifthen}
\newcommand{\ifempty}[3]{%
  \ifthenelse{\isempty{#1}}{#2}{#3}%
}


\newcommand{\thhat}[1]{\hat{\theta}_{#1}}
\newcommand{\var}[1]{\text{Var}\left(#1\right)}

\begin{document}
  \maketitle

  \noindent
  \textbf{Name}: \name\\
  \textbf{Course}: \course\\
  \textbf{Assignment}: \assnName\\
  \textbf{Due Date}: \dueDate

  \noindent
  \textbf{Other Student Discussions}: I discussed the problems in this homework with the following student(s) below.  All write-ups were prepared independently.
  \vspace{-1em}
  \begin{itemize}
    \item Viet Lai
  \end{itemize}

  \newpage
  \begin{problem}
  \probNum{6.5.2} Let ${y_1,y_2,\ldots,y_{10}}$ be a random sample from an exponential pdf with unknown parameter~$\lambda$. Find the form of the GLRT for $H_0$: ${\lambda = \lambda_0}$ versus ${H_1}$: ${\lambda \ne \lambda_0}$.  What integral would have to be evaluated to determine the critical value if~$\alpha$ were equal to~0.05.
\end{problem}

For ${n=10}$ the likelihood is:
\begin{align}
  L_{n}(\lambda) &= \prod_{i=1}^{n} \lambda e^{-\lambda y_i} \\
                 &= \lambda^{n} e^{-\lambda \sum_{i=1}^{n} y_i} \text{.}
\end{align}

If ${H_0}$ is true, then:
\begin{equation}
  L_{n}(\omega) = \lambda_{0}^{n}  e^{-\lambda_0 \sum_{i=1}^n y_i} \text{.}
\end{equation}

The maximum likelihood is found via:
\begin{align}
  \ell_{n}(\lambda)   &= n \ln (\lambda) - \lambda \sum_{i=1}^{n} y_i \\
  \ell'_{n}(\lambda)  &= \frac{n}{\lambda} - \sum_{i=1}^{n} y_i = 0 \\
          \lambda_{e} &= \frac{n}{\sum_{i=1}^{n} y_i} \text{.}
\end{align}

Therefore:
\begin{align}
  \lambda &= \frac{L(\omega)}{L(\Omega)} \\
          &= \frac{\lambda_{0}^{n}  e^{-\lambda_0 \sum_{i=1}^n y_i}}{\lambda_{e}^{n}  e^{-n}} \\
          &= \frac{\lambda_{0}^{n} \sum_{i=1}^n y_i e^{n -\lambda_0 \sum_{i=1}^n y_i}}{n} \\
          &= \lambda_{0}^{n} \bar{y} e^{n -n \lambda_0 \bar{y}} \text{,}
\end{align}
\noindent
where ${\bar{y} = \frac{1}{n} \sum_{i=1}^{n} y_i}$\text{.}

The null hypothesis should be rejected when ${\lambda_{0}^{n} \bar{y} e^{n -n \lambda_0 \bar{y}} \leq \lambda^{*}}$.  $\lambda^{*}$ is found via:
\begin{equation}
  \int_{0}^{\lambda^*} f_{\Lambda} (\lambda \vert H_0 \text{ is true}) d\lambda = \alpha \text{.}
\end{equation}

  \newpage
  \begin{problem}
  \probNum{6.5.3}~Let ${y_1,y_2,\ldots,y_n}$ be a random sample from a normal pdf with unknown mean~$\mu$ and variance~1.  Find the form of the GLRT for ${H_0: \mu = \mu_0}$ versus ${H_1: \mu \ne \mu_0}$.
\end{problem}

The likelihood is:
\begin{equation}
  L_{n}(\mu) = \left( 2 \pi \right)^{-n/2} e^{-\frac{\sum_{i=1}^{n}(y_i - \mu)^2}{2}} \text{.}
\end{equation}

The maximum likelihood is:
\begin{align}
  \ell_{n}(\mu)  &= -\frac{n}{2} \ln(2\pi) - \frac{\sum_{i=1}^{n} (y_i - \mu)^2}{2} \\
  \ell'_{n}(\mu) &=  \sum_{i=1}^{n} (y_i - \mu) = 0 \\
           \mu_e &= \frac{\sum_{i=1}^{n} y_i}{n} = \bar{y} \text{.}
\end{align}

The GLR is:
\begin{align}
  \mu = \frac{\exp\left( -\frac{\sum_{i=1}^{n} (y_i - \mu_0)}{2} \right)}{\exp\left( -\frac{\sum_{i=1}^{n} (y_i - \bar{y})}{2} \right)} \\
  \mu = \exp\left( \frac{ - \sum_{i=1}^{n} (y_i - \mu_0) + \sum_{i=1}^{n} (y_i - \bar{y})}{2} \right) \\
  \mu = \exp\left( \frac{\sum_{i=1}^{n} (-y_{i}^{2} + 2y_i \mu_0  - \mu^2_0 + y^2_i - 2y_i \bar{y}  + \bar{y}^2)}{2} \right) \\
  \mu = \exp\left( \frac{2n \bar{y} \mu_0  - n\mu^2_0 - 2n\bar{y}^2 + n\bar{y}^2}{2} \right) \\
  \mu = \exp\left(\frac{-n}{2} (\mu_0 - \bar{y})^2 \right)
\end{align}

The GLRT is ${\exp\left(\frac{-n}{2} (\mu_0 - \bar{y})^2 \right) \leq \mu^{*}}$.

  \newpage
  \begin{problem}
  \probNum{6.5.04} In the scenario of Question~6.5.3, suppose the alternative hypothesis is ${H_1: \mu = \mu_1}$ for some particular value of~$\mu_1$. How does the likelihood ratio test change in this case? In what way does the critical region depend on the particular value of~$\mu_1$?
\end{problem}

The value of~$\lambda$ becomes:
\begin{align}
  \lambda &= \frac{\exp\left( - \frac{\sum_{i=1}^{n} (y_i - \mu_0)^{2}}{2} \right) }{\exp\left( - \frac{\sum_{i=1}^{n} (y_i - \mu_1)^{2}}{2} \right) } \\
          &= \exp\left( \frac{ -\sum_{i=1}^{n} (y_i - \mu_0)^{2} + (y_i - \mu_1)^{2} }{2} \right) \\
          &= \exp\left( \frac{ -n\bar{y}^2 +2n\bar{y}\mu_0 - n\mu_0^{2} + n\bar{y}^2 - 2n\bar{y}\mu_1 + n\mu_1^{2} }{2} \right) \\
          &= \exp\left( \frac{ 2n\bar{y}\mu_0 - n\mu_0^{2} - 2n\bar{y}\mu_1 + n\mu_1^{2} }{2} \right) \\
\end{align}

Therefore,
\begin{equation}
  \exp\left( \frac{ 2n\bar{y}\mu_0 - n\mu_0^{2} - 2n\bar{y}\mu_1 + n\mu_1^{2} }{2} \right) \leq \lambda^* \text{.}
\end{equation}

The null hypothesis will be rejected (i.e.,~$\lambda^{*}$ is in the critical region when)
\begin{align}
  2n\bar{y}\mu_0 - n\mu_0^{2} - 2n\bar{y}\mu_1 + n\mu_1^{2} &\leq 2\ln (\lambda^*) \\
  \bar{y} (2n \mu_0 - 2n\mu_1) &\leq n\mu_0^{2} - n\mu_1^{2} + 2\ln (\lambda^*) \\
  \bar{y} &\leq \frac{n\mu_0^{2} - n\mu_1^{2} + 2\ln (\lambda^*)}{2n \mu_0 - 2n\mu_1} \text{.}
\end{align}

  \newpage
  \begin{problem}
  \probNum{6.5.6}~Suppose a sufficient statistic exists for the parameter~$\theta$. Use Theorem~5.6.1 to show that the critical region of a likelihood ratio test will depend on the sufficient statistic.
\end{problem}

By Theorem~5.6.1, the likelihood can be decomposed into the product of two functions namely:
\begin{enumerate}
  \item $g$, which is a function of the sufficient statistic and $\theta$
  \item $b$, which is exclusively a function of the sample data.
\end{enumerate}
Since~$b$ and~$\theta$ are unrelated, the~$\theta$ maximizing~$\g$ is solely based on the sufficient statistic.  Therefore, the GLR value,~$\lambda$ which is a ratio of likelihoods must also be a function of the sufficient statistic.

  \newpage
  \begin{problem}
 \probNum{7.3.3} Is it believable that the numbers 65, 30, and~55 are a random sample of size~3 from a normal distribution with ${\mu = 50}$ and ${\sigma = 10}$ Answer the question by using a chi-square distribution. \textnormal{Hint: Let ${Z_i = (Y_1 - 50)/10}$ and use Theorem~7.3.1}.
\end{problem}

Using the specified transformation, we can solve for~$U$ as
\begin{align}
  U &= \sum_{j=1}^{m} Z_i^2 \\
    &= 1.5^2 + (-2)^2 + 0.5^2
    &= 2.25 + 4 + 0.25 \\
    &= 6.5 \text{,}
\end{align}
\noindent
with ${m=3}$.  Reviewing the table in the back of the book with ${\alpha = 0.05}$, the range of expected values is ${[0.216, 9.348]}$.  Therefore, there is insufficient evidence to reject the null hypothesis.

  \newpage
  \begin{problem}
  Use Appendix Table~A.4 to find:
\end{problem}

\begin{subproblem}
  $F_{.50,6,7}$
\end{subproblem}

\begin{subproblem}
  $F_{.001,15,5}$
\end{subproblem}

\begin{subproblem}
  $F_{.9,2,2}$
\end{subproblem}

  \newpage
  \begin{problem}
  \probNum{7.3.09} Use Appendix Table~A.4 to find the values of~$x$ that satisfy the following equations:
\end{problem}

\begin{subproblem}
  $\Pr\sbrack{0.109 < F_{4,6} < x} = 0.95$
\end{subproblem}

${x = 6.23}$

\begin{subproblem}
  $\Pr\sbrack{0.427 < F_{11,7} < 1.69} = x$
\end{subproblem}

${x = 0.75 - 0.1 = 0.65}$

\begin{subproblem}
  $\Pr\sbrack{F_{x,x} > 5.35} = 0.01$
\end{subproblem}

${m = 9}$, ${n = 9}$

\begin{subproblem}
  $\Pr\sbrack{0.115 < F_{3,x} < 3.29} = 0.90$
\end{subproblem}

${n = 15}$

\begin{subproblem}
  $\Pr\sbrack{x < \frac{V/2}{U/3}} = 0.25$ where~$V$ is a chi square random variable with~2~df and~$U$ is an independent chi square random variable with~3~df.
\end{subproblem}

${x = 2.28}$

  \newpage
  \begin{problem}
  \probNum{7.4.01}~Use Appendix Table~A.2 to find the following probabilities:
\end{problem}

\begin{subproblem}
  $\Pr\sbrack{T_6 \geq 1.134}$
\end{subproblem}

0.15

\begin{subproblem}
  $\Pr\sbrack{T_{15} \leq 0.866}$
\end{subproblem}

0.80

\begin{subproblem}
  $\Pr\sbrack{T_3 \geq -1.250}$
\end{subproblem}

0.85

\begin{subproblem}
  $\Pr\sbrack{-1.055 < T_{29} < 2.462}$
\end{subproblem}

${0.99 - 0.15 = 0.84}$

  \newpage
  \begin{problem}
  \probNum{7.4.2}~What values of~$x$ satisfy the following equations?
\end{problem}

\begin{subproblem}
  ${\Pr\sbrack{-x \leq T_{22} \leq x} = 0.98}$
\end{subproblem}

$x = 2.508$

\begin{subproblem}
  ${\Pr\sbrack{T_{13} \geq x} = 0.85}$
\end{subproblem}

$x = -1.079$

\begin{subproblem}
  ${\Pr\sbrack{T_{26} < x} = 0.95}$
\end{subproblem}

$x = 1.7056$

\begin{subproblem}
  ${\Pr\sbrack{T_{2} \geq x} = 0.025}$
\end{subproblem}

$x = 4.3027$

  \newpage
  \begin{problem}
  \probNum{7.4.5}~Suppose a random sample of size~${n=11}$ is drawn from a normal distribution with ${\mu = 15.0}$.  For what value of~$k$ is the following true?
  \begin{equation}
    \Pr\sbrack{\abs{\frac{\bar{Y} - 15.0}{S/\sqrt{11}}} \geq k} = 0.05
  \end{equation}
\end{problem}

The degrees of freedom is~${n - 1 = 10}$. ${k = 2.2281}$

  \newpage
  \begin{problem}
  \probNum{7.4.7}~Cell phones emit radio frequency energy that is absorbed by the body when the phone is next to the ear and may be harmful.  The table in the next column gives the absorption rate for a sample of twenty high-radio cell phones.  Construct a 90\%~confidence interval for the true average cell phone absorption rate.
\end{problem}

The number of samples is ${n=20}$ so there are 19~degrees of freedom.  The mean is~1.4255.  The sample standard deviation is:
\begin{equation}\label{eq:SampleStdDev}
  S = \sqrt{\frac{\sum_{i=1}^{n} (X_i - \bar{X})}{n - 1}}\text{.}
\end{equation}
\noindent
${S = 0.056}$. For~${\alpha = 0.9}$, ${T_{\alpha / 2} = 1.7291}$.

The confidence interval is ${1.4255 \pm \frac{T \cdot S}{\sqrt{n}}}$.  The final range is ${[1.404,1.447]}$.

\end{document}
