\documentclass{report}

\newcommand{\name}{Zayd Hammoudeh}
\newcommand{\course}{MATH562}
\newcommand{\assnName}{Homework~\#3}
\newcommand{\dueDate}{January~31,~2020}

\usepackage[margin=1in]{geometry}
\usepackage[skip=4pt]{caption}      % ``skip'' sets the spacing between the figure and the caption.
\usepackage{tikz}
\usetikzlibrary{arrows.meta,decorations.markings,shadows,positioning,calc}
\usepackage{pgfplots}               % Needed for plotting
\pgfplotsset{compat=newest}
\usepgfplotslibrary{fillbetween}    % Allow for highlighting under a curve
\usepackage{amsmath}                % Allows for piecewise functions using the ``cases'' construct
\usepackage{siunitx}                % Allows for ``S'' alignment in table to align by decimal point

\usepackage[obeyspaces,spaces]{url} % Used for typesetting with the ``path'' command
\usepackage[hidelinks]{hyperref}    % Make the cross references clickable hyperlinks
\usepackage[bottom]{footmisc}       % Prevents the table going below the footnote
\usepackage{nccmath}                % Needed in the workaround for the ``aligncustom'' environment
\usepackage{amssymb}                % Used for black QED symbol
\usepackage{bm}                     % Allows for bolding math symbols.
\usepackage{tabto}                  % Allows to tab to certain point on a line
\usepackage{float}
\usepackage{subcaption}             % Allows use of the ``subfigure'' environment
\usepackage{enumerate}              % Allow enumeration other than just numbers

\usepackage[noend]{algpseudocode}
\usepackage[Algorithm,ruled]{algorithm}
\algnewcommand\algorithmicforeach{\textbf{for each}}
\algdef{S}[FOR]{ForEach}[1]{\algorithmicforeach\ #1\ \algorithmicdo}

%---------------------------------------------------%
%     Define Distances Used for Document Margins    %
%---------------------------------------------------%

\newcommand{\hangindentdistance}{1cm}
\newcommand{\defaultleftmargin}{0.25in}
\newcommand{\questionleftmargin}{-.5in}

\setlength{\parskip}{1em}
\setlength{\oddsidemargin}{\defaultleftmargin}

%---------------------------------------------------%
%      Configure the Document Header and Footer     %
%---------------------------------------------------%

% Set up page formatting
\usepackage{todonotes}
\usepackage{fancyhdr}                   % Used for every page footer and title.
\pagestyle{fancy}
\fancyhf{}                              % Clears both the header and footer
\renewcommand{\headrulewidth}{0pt}      % Eliminates line at the top of the page.
\fancyfoot[LO]{\course\ -- \assnName}   % Left
\fancyfoot[CO]{\thepage}                % Center
\fancyfoot[RO]{\name}                   % Right

%---------------------------------------------------%
%           Define the Title Page Entries           %
%---------------------------------------------------%

\title{\textbf{\course\ -- \assnName}}
\author{\name}

%---------------------------------------------------%
% Define the Environments for the Problem Inclusion %
%---------------------------------------------------%

\usepackage{scrextend}
\newcounter{problemCount}
\setcounter{problemCount}{0}  % Reset the subproblem counter

\newcounter{subProbCount}[problemCount]   % Reset subProbCount any time problemCount changes.
\renewcommand{\thesubProbCount}{\alph{subProbCount}}  % Make it so the counter is referenced as a number

\newenvironment{problemshell}{
  \begin{addmargin}[\questionleftmargin]{0em}
    \par%
    \medskip
    \leftskip=0pt\rightskip=0pt%
    \setlength{\parindent}{0pt}
    \bfseries
  }
  {
    \par\medskip
  \end{addmargin}
}
\newenvironment{problem}
{%
  \refstepcounter{problemCount} % Increment the subproblem counter.  This must be before the exercise to ensure proper numbering of claims and lemmas.
  \begin{problemshell}
    \noindent \textit{Exercise~\#\arabic{problemCount}} \\
  }
  {
  \end{problemshell}
  %  \setcounter{subProbCount}{0} % Reset the subproblem counter
}
\newenvironment{subproblem}
{%
  \begin{problemshell}
    \refstepcounter{subProbCount} % Increment the subproblem counter
    \setlength{\leftskip}{\hangindentdistance}
    % Print the subproblem count and offset to the left
    \hspace{-\hangindentdistance}(\alph{subProbCount}) \tabto{0pt}
  }
  {
  \end{problemshell}
}

% Change interline spacing.
\renewcommand{\baselinestretch}{1.1}
\newenvironment{aligncustom}
{ \csname align*\endcsname % Need to do this instead of \begin{align*} because of LaTeX bug.
  \centering
}
{
  \csname endalign*\endcsname
}


%---------------------------------------------------%
% Define the Environments for the Problem Inclusion %
%---------------------------------------------------%

\usepackage{amsthm}       % Allows use of the ``proof'' environment.

% Number lemmas and claims using the problem count
\newtheorem{claim}{Claim}[problemCount]
\newtheorem{lemma}{Lemma}[problemCount]

%---------------------------------------------------%
%       Define commands related to managing         %
%    floats (e.g., images) across multiple pages    %
%---------------------------------------------------%

\usepackage{placeins}     % Allows \FloatBarrier

% Prevent preceding floats going to this page
\newcommand{\floatnewpage}{\FloatBarrier\newpage}

% Add the specified input file and prevent any floated figures/tables going onto the same page as new input
\newcommand{\problemFile}[1]{
  \floatnewpage
  \input{#1}
}

\newcommand{\probNum}[1]{(\textnormal{Problem: #1})}

\newcommand{\etal}{~et~al.}

% Used for including standalone docs
\usepackage{standalone}

% Imported via UltiSnips
% Unbreakable dash:
%  Words hyphened with these dashes can also be broken at other positions than the dash
%    \-/ hyphen
%    \-- en-dash
%    \--- em-dash
%    extdash unbreakable dashes
%
%  No line breaks possible at the hyphen
%    \=/ hyphen
%    \== en-dash
%    \=== em-dash
\usepackage[shortcuts]{extdash}

% Imported via UltiSnips
\usepackage{color}
\newcommand{\colortext}[2]{{\color{#1} #2}}
\newcommand{\red}[1]{\colortext{red}{#1}}
\newcommand{\blue}[1]{\colortext{blue}{#1}}
\newcommand{\green}[1]{\colortext{green}{#1}}

% Imported via UltiSnips
\usepackage{amsmath}
\DeclareMathOperator*{\argmax}{arg\,max}
\DeclareMathOperator*{\argmin}{arg\,min}
\usepackage{amsfonts}  % Used for \mathbb and \mathcal
\usepackage{amssymb}

% Imported via UltiSnips
\usepackage{mathtools} % for "\DeclarePairedDelimiter" macro
% \swapifbranches changes unstarred paired delimiters to starred and
% vice versa.  This means by default, paired delimiters have the star.
\usepackage{etoolbox}
\newcommand\swapifbranches[3]{#1{#3}{#2}}
\makeatletter
\MHInternalSyntaxOn
\patchcmd{\DeclarePairedDelimiter}{\@ifstar}{\swapifbranches\@ifstar}{}{}
\MHInternalSyntaxOff
\makeatother
% Place after swap to ensure swap star
\DeclarePairedDelimiter{\sbrack}{\lbrack}{\rbrack}
\DeclarePairedDelimiter{\floor}{\lfloor}{\rfloor}
\DeclarePairedDelimiter{\ceil}{\lceil}{\rceil}
\DeclarePairedDelimiter{\abs}{\lvert}{\rvert}
\DeclarePairedDelimiter{\norm}{\lVert}{\rVert}
\usepackage{bm}
\DeclarePairedDelimiterX\set[1]\lbrace\rbrace{#1}
\DeclarePairedDelimiterX\setbuild[2]\lbrace\rbrace{#1 \bm: #2}
\newcommand{\setint}[1]{{\sbrack{#1}}}
\newcommand{\func}[3]{{#1:#2\rightarrow#3}}
% \newcommand{\defeq}{\stackrel{\mathclap{\mbox{\tiny def}}}{=}}
\newcommand{\defeq}{\coloneqq}
\newcommand{\fedeq}{\eqqcolon}
\newcommand{\expect}[1]{\mathbb{E}\sbrack{#1}}
% Expectation with the subscript defining the distribution
\newcommand{\expectS}[2]{\mathbb{E}_{#1}\sbrack{#2}}

% Allow numbering in align*
\newcommand{\numberthis}{\addtocounter{equation}{1}\tag{\theequation}}

\newcommand{\ints}{\mathbb{Z}}
\newcommand{\nats}{\mathbb{N}}
\newcommand{\real}{\mathbb{R}}
\newcommand{\realnn}{\real_{{\geq}0}}  % Set of non-negative real numbers

\newcommand{\iidsim}{\stackrel{\mathclap{\mbox{\tiny i.i.d.}}}{\sim}}

\newcommand{\normaldist}[2]{{\mathcal{N}\mathopen{}\left(#1,#2\right)\mathclose{}}}

% Imported via UltiSnips
\usepackage{array}  % Provides a way add a \centering command to a p-column
\usepackage{arydshln}  % Introduces hdashline & cdashline
\usepackage{bigdelim}
\usepackage{booktabs}
\usepackage{multirow}
\usepackage{makecell}  % Needed for multirowcell

% % Imported via UltiSnips
\usepackage{amsthm}
\newtheorem{theorem}{Theorem}[section]
% \newtheorem{corollary}{Corollary}[theorem]  % Corollary number derives from theorem
% \newtheorem{lemma}[theorem]{Lemma}  % Lemma and theorem share same counter
% \newtheorem{claim}[theorem]{Claim}  % Same numbering as lemma and theorem
% \newtheorem*{remark}{Remark}
% \newtheorem*{note}{Note}
% \newtheoremstyle{definition}  % <name>
% {3pt}   % <Space above>
% {3pt}   % <Space below>
% % {\itshape}     % <Body font>
% {\normalfont}   % <Body font>
% {}      % <Indent amount>
% {\bfseries} % <Theorem head font>
% {:}     % <Punctuation after theorem head>
% {.5em}  % <Space after theorem head>
% {}      % <Theorem head spec (can be left empty, meaning `normal')>
% \theoremstyle{definition}
% \newtheorem{definition}{Def.}[section]

% % Imported via UltiSnips
% \usepackage[noend]{algpseudocode}
\usepackage[Algorithm,ruled]{algorithm}
% \algnewcommand\algorithmicforeach{\textbf{for each}}
% \algdef{S}[FOR]{ForEach}[1]{\algorithmicforeach\ #1\ \algorithmicdo}
% \newcommand{\algin}[1]{\hspace*{\algorithmicindent} \textbf{Input} #1\\}
% \newcommand{\algin}[1]{\textbf{Input} #1}
% \newcommand{\algout}[1]{\hspace*{\algorithmicindent} \textbf{Output} #1}

% Imported via UltiSnips
\usepackage{tikz}
\usetikzlibrary{arrows,decorations.markings,shadows,positioning,calc,backgrounds,shapes}

\usepackage{pgfplots}
\pgfplotsset{compat=1.13}
\usepackage{pgfplotstable}
% \usepackage{subcaption}  % Cannot be used with subfigure

% Handle empty parameters
\usepackage{xifthen}
\newcommand{\ifempty}[3]{%
  \ifthenelse{\isempty{#1}}{#2}{#3}%
}


\newcommand{\thhat}[1]{\hat{\theta}_{#1}}
\newcommand{\var}[1]{\text{Var}\left(#1\right)}

\begin{document}
  \maketitle

  \noindent
  \textbf{Name}: \name\\
  \textbf{Course}: \course\\
  \textbf{Assignment}: \assnName\\
  \textbf{Due Date}: \dueDate

  \noindent
  \textbf{Other Student Discussions}: I discussed the problems in this homework with the following student(s) below.  All write-ups were prepared independently.
  \vspace{-1em}
  \begin{itemize}
    \item Viet Lai
  \end{itemize}

  \newpage
  \begin{problem}
  \probNum{5.4.17}~Let ${X_1,X_2,\ldots,X_n}$  denote the outcomes of a series of $n$~independent trials, where
  \begin{equation*}
    X_i =   \begin{cases}
              1 & \text{with probability }~p \\
              0 & \text{with probability }~1 - p
            \end{cases}
  \end{equation*}
  \noindent
  for ${i=1,2,\ldots,n}$. Let ${X = X_1 + X_2 + \cdots + X_n}$.
\end{problem}

\begin{subproblem}
  Show that $\hat{p}_1 = X_1$ and ${\hat{p}_2 = \frac{X}{n}}$ are unbiased estimators for~$p$.
\end{subproblem}

\begin{align*}
  \hat{p}_1 = \expect{X_i} &= \sum_{x \in \mathcal{X}} p_x \cdot x \\
                            &= 1 * p + 0 * (1 - p) =\boxed{p}
\end{align*}

\begin{align*}
  \hat{p}_2 = \expect{\frac{\bar{x}}{n}} &= \expect{\frac{1}{n} \left(X_1 + \cdots + X_n\right)} \\
                                          &= \frac{1}{n} \sum_{i=1}^{n} \expect{X_i} \\
                                          &= \frac{1}{n} np = \boxed{p}
\end{align*}

\begin{subproblem}
  Intuitively, $\hat{p}_2$ is a better estimator than $\hat{p}_1$ because $\hat{p}_1$ fails to include any of the information about the parameter contained in trials~2 through~$n$. Verify that speculation by comparing the variances of~$\hat{p}_1$ and~$\hat{p}_2$.
\end{subproblem}

$\var{X_1} = p(1-p)$ (Variance of Bernoulli)

\begin{align}
  \var{\frac{\bar{x}}{n}} &= n \var{\frac{X_i}{n}} \\
                          &= \frac{1}{n} \text{Var}\left(X_i\right) \\
                          &= \frac{p(1-p)}{n} \text{.}
\end{align}

For ${n > 1}$, ${\frac{p(1-p)}{n} < p(1-p)}$.

  \newpage
  \begin{problem}
  \probNum{5.4.18} Suppose that ${n = 5}$ observations are taken from the uniform pdf ${f_{Y}(y;\theta) = 1/\theta, 0 \leq y \leq \theta}$ where $\theta$~is unnown.  Two biased estimators for $\theta$~are:
  \begin{equation*}
    \hat{\theta}_1 = \frac{6}{5} \cdot Y_{\max} \text{ and } \hat{\theta}_2 = 6\cdot Y_{\min}\text{.}
  \end{equation*}
\end{problem}

\begin{align*}
  \expect{Y_{\max}^2} &= \int_{0}^{\theta} \frac{n y^{n+1}}{\theta^{n}} dy \\
                                &= \frac{n\theta^2}{n+2}
\end{align*}

\begin{align*}
  \text{Var}(Y_{\max}) &= \expect{Y_{\max}^2} - \expect{Y_{\max}}^{2} \\
                       &= \frac{n}{n+2}\theta^2 - \frac{n^2}{(n+1)^{2}}\theta^2 \\
                       &= \frac{n}{(n+2)(n+1)^{2}}\theta^{2}
\end{align*}

\begin{align*}
  \expect{Y_{\min}} &= \int_{0}^{\theta} y f_{Y_{\min}}(y;\theta)dy \\
                    &= \int_{0}^{\theta} \frac{ny}{\theta}\left(\frac{\theta -y}{\theta}\right)^{n-1} dy \\
                    &= \int_{0}^{\theta} \left(\frac{\theta -y}{\theta}\right)^{n} dy & \text{Integration by parts} \\
                    &= \frac{\theta}{n+1} & \text{Integration by Parts}
\end{align*}

\begin{align*}
  \expect{Y_{\min}^{2}} &= \int_{0}^{\theta} y^2 f_{Y_{\min}}(y;\theta)dy \\
                    &= \int_{0}^{\theta} \frac{ny^2}{\theta}\left(\frac{\theta -y}{\theta}\right)^{n-1} dy \\
                    &= \int_{0}^{\theta} 2y\left(\frac{\theta -y}{\theta}\right)^{n} dy & \text{Integration by parts} \\
                    &= \int_{0}^{\theta} \frac{2\theta}{n+1} \left(\frac{\theta -y}{\theta}\right)^{n+1} dy & \text{Integration by parts} \\
                    &= \frac{2\theta^2}{(n+1)(n+2)} & \text{Integration by parts}
\end{align*}

\begin{align*}
  \text{Var}(Y_{\min}) &= \expect{Y_{\min}^2} - \expect{Y_{\min}}^{2} \\
                       &= \frac{2\theta^2}{(n+1)(n+2)}\theta^2 - \frac{\theta^2}{(n+1)^{2}} \\
                       &=  \frac{n}{(n+1)^{2}(n+2)}\theta^2
\end{align*}

\begin{align*}
  \var{\hat{\theta}_1} &= \frac{\theta^{2}}{n(n+2)}\\
  \var{\hat{\theta}_2} &= \frac{n\theta^{2}}{n+2}
\end{align*}

${\var{\hat{\theta}_1} < \var{\hat{\theta}_2}}$.  This is expected since multiplicative scalar on $Y_{\min}$ is larger and amplifying the error.

  \newpage
  \begin{problem}
  \probNum{5.4.19}~Let ${Y_1,Y_2,\ldots,Y_n}$ be a random sample size of size~$n$ from the pdf
  \begin{equation}
    f_Y(y;\theta) = \frac{1}{\theta}e^{-y\theta}\text{, }y > 0 \text{.}
  \end{equation}
\end{problem}

\begin{subproblem}
  Show that ${\thhat{1} = Y_1}$, ${\thhat{2} = \bar{Y}}$, and ${\thhat{3} = nY_{\min}}$ are all unbiased estimators for~$\theta$.
\end{subproblem}

\begin{align}
  \expect{Y_1} &= \int_{0}^{\infty} \frac{y}{\theta} e^{-y/\theta}dy \\
               &= e^{-y/\theta} \bigg\vert_{0}^{\theta} & \text{Integration by Parts}\\
               &= \theta
\end{align}

\begin{align}
  \expect{\bar{y}} &= \frac{1}{n} \sum_{i=1}^{n} \expect{Y_i} \\
                   &= \frac{1}{n} n\theta = \theta
\end{align}

\begin{align}
  f_{Y_{\min}}(y) &= \frac{n}{\theta}e^{-y/\theta}\left(e^{-y/\theta}\right)^{n-1} \\
  \expect{Y_{\min}} &= \int \frac{yn}{\theta} e^{-yn/\theta} dy \\
                    &= \int e^{-yn/\theta} dy & \text{Integration by parts} \\
                    &= \frac{\theta}{n}
\end{align}
Therefore, ${\expect{\thhat{3}} = n\expect{Y_{\min}} = \boxed{\theta}}$.

\begin{subproblem}
  Find the variances of~$\thhat{1}$, $\thhat{2}$, and~$\thhat{3}$.
\end{subproblem}

\begin{align}
  \expect{Y^2} &= \int_{0}^{\infty} \frac{y^2}{\theta} e^{-y/\theta}dy \\
               &= \int_{0}^{\infty} 2y e^{-y/\theta}dy & \text{Integration by parts}\\
               &= \int_{0}^{\infty} 2\theta e^{-y/\theta}dy & \text{Integration by parts}\\
               &= 2\theta^{2}
\end{align}
Therefore, ${\boxed{\thhat{1} = \theta^2}}$.

\begin{align}
  \var{\thhat{2}} &= \sum_{i=1}^{n} \var{Y_i /n} \\
                  &= \boxed{\frac{1}{n}\theta^{2}}\text{.}
\end{align}

\begin{align}
  f_{Y_{\min}}(y) &= \frac{1}{n\theta}e^{-y/\theta}\left(e^{-y/\theta}\right)^{n-1} \\
  \expect{Y_{\min}^2} &= \int \frac{y^2}{n\theta} e^{-yn/\theta} dy \\
                    &= \int 2y e^{-yn/\theta} dy & \text{Integration by parts} \\
                    &= \int \frac{2\theta}{n} e^{-yn/\theta} dy & \text{Integration by parts} \\
                    &=  \frac{2\theta^{2}}{n^{2}} & \text{Integration by parts}
\end{align}
Therefore, ${\var{\thhat{3}} = n^2 \var{Y_{\min}} = \boxed{\theta^2}}$.

\begin{subproblem}
  Calculate the relative efficiencies of~$\thhat{1}$ to~$\thhat{3}$ and $\thhat{2}$ to~$\thhat{3}$.
\end{subproblem}

\begin{align}
  \text{Efficiency }\thhat{1} \text{ to }\thhat{3} &= \frac{\var{\thhat{3}}}{\var{\thhat{1}}} \\
                                                   &= \frac{\theta^2}{\theta^2} = \boxed{1} \\
  \text{Efficiency }\thhat{2} \text{ to }\thhat{3} &= \frac{\var{\thhat{3}}}{\var{\thhat{2}}} \\
                                                   &= \frac{\theta^2}{\theta^2 /n} = \boxed{n}
\end{align}

  \newpage
  \begin{problem}
  \probNum{5.4.21} If ${Y_1,Y_2,\ldots,Y_n}$ are random observations from a uniform pdf over~${[0,\theta]}$, bother ${\hat{\theta}_1 \cdot Y_{\max}}$ and ${\hat{\theta}_2 = (n+1) \cdot Y_{\min}}$ are unbiased estimators for~$\theta$.  Show that ${\var{\hat{\theta}_2} / \var{\hat{\theta}_1} = n^2}$.
\end{problem}

From my solution to~5.4.18
\begin{align}
  \var{\hat{\theta}_1} &= \frac{\theta^{2}}{n(n+2)}\\
  \var{\hat{\theta}_2} &= \frac{n\theta^{2}}{n+2}
\end{align}

Therefore:
\begin{equation}
  \frac{\var{\hat{\theta}_2}}{\var{\hat{\theta}_1}} = n^2
\end{equation}

  \newpage
  \begin{problem}
  \probNum{5.5.1} Let ${Y_1,Y_2,\ldots,Y_n}$ be a random sample from ${f_{Y}(y;\theta) = \frac{1}{\theta} e^{-y/\theta}\text{, } y > 0}$.  Compare the Cramer-Rao lower bound for ${f_{Y}(y;\theta)}$ to the variance of the maximum likelihood estimator for~$\theta$, ${\hat{\theta} = \frac{1}{n}\sum_{i=1}^{n}Y_i}$. Is~$\bar{Y}$ a best estimator for~$\theta$?
\end{problem}

\begin{align}
  \ln f_Y &= -\ln\theta - \frac{y}{\theta} \\
  \frac{\partial \ln f_Y}{\partial \theta} &= -\frac{1}{\theta} + \frac{y}{\theta^2} \\
  \frac{\partial^{2} \ln f_Y}{\partial \theta^{2}} &= \frac{1}{\theta^2} - \frac{2y}{\theta^3}
\end{align}

\begin{align}
  \expect{\frac{1}{\theta^2} - \frac{2y}{\theta^3}} &= \frac{1}{\theta^2} - \frac{2}{\theta^4} \int_{0}^{\infty} ye^{-y/\theta} dy \\
                                                    &= \frac{1}{\theta^2} - \frac{2}{\theta^3} \int_{0}^{\infty} e^{-y/\theta} dy  & \text{Integration by Parts} \\
                                                    &= \frac{1}{\theta^2} - \frac{2}{\theta^2} \\
                                                    &= - \frac{1}{\theta^2}
\end{align}

The Cramer-Rao lower bound is therefore ${\boxed{\frac{\theta^2}{n}}}$.

The exponential distribution's variance is~$\theta^2$.  Therefore, the sample mean's variance is~$\frac{\theta}{n}$ making $\bar{Y}$~a best estimator.

  \newpage
  \begin{problem}
  \probNum{5.5.2} Let ${X_1,X_2,\ldots,X_n}$ be a random sample of size~$n$ from the Poisson distribution, ${p_{X}(k;\lambda) = \frac{e^{-\lambda} \lambda^{k}}{k!}}$, ${k=0,1,\ldots}$. Show that ${\hat{\lambda} = \frac{1}{n} \sum_{i=1}^{n} X_i}$ is an efficient estimator for~$\lambda$.
\end{problem}

\begin{align}
  \ln p &= \ln \left( \frac{e^{-\lambda} \lambda^{k}}{k!} \right) \\
        &= -\lambda + k\ln(\lambda) - \ln(k!) \\
  \frac{\partial \ln p}{\partial \lambda} &= -1 +\frac{k}{\lambda} \\
  \frac{\partial^2 \ln p}{\partial \lambda^2} &= -\frac{k}{\lambda^2}
\end{align}

\begin{align}
  \expect{\frac{\partial^2 \ln p}{\partial \lambda^2}} &= -\frac{1}{\lambda^2} \expect{k} \\
                                                       &= -\frac{1}{\lambda}
\end{align}

The Cramer-Rao lower bound is ${\boxed{\frac{\lambda}{n}}}$.

The Poisson distribution's variance is~$\lambda$.  Therefore, the mean has variance~${\frac{\lambda}{n}}$ making it efficient.

  \newpage
  \begin{problem}
  \probNum{5.5.4}~Let ${Y_1,Y_2,\ldots,Y_n}$ be a random sample from the uniform pdf ${f_Y(y;\theta) = 1 / \theta \text{, } 0 \leq y \leq \theta}$. Compare the Cramer-Rao lower bound for~${f_Y(y;\theta)}$ with the variance of the unbiased estimator ${\hat{\theta} = \frac{n+1}{n} Y_{\max}}$. Discuss.
\end{problem}

\begin{align}
  \ln f_Y &= \ln \left(\frac{1}{\theta}\right) \\
  \frac{\partial \ln f_Y}{\partial \theta} &= -\frac{1}{\theta} %\\
  % \frac{\partial^2 \ln f_Y}{\partial \theta^2} &= \frac{1}{\theta^2}
\end{align}

\begin{align}
  \expect{\left(\frac{\partial \ln f_Y}{\partial \theta} \right)^2} = \expect{\frac{1}{\theta^2}} &= \int_{0}^{\theta}\frac{1}{\theta^3}dy \\
                              &= \frac{1}{\theta^2}
\end{align}

This makes the Cramer-Rao lower bound $\boxed{\frac{\theta^2}{n}}$.

In problem~5.4.18, we determined the estimator's variance was~$\frac{\theta^2}{n(n+1)}$ which is less than the Cramer-Rao lower bound.  Note that Theorem~5.5.1 specifics that the lower bound only applies when the support does not depend on~$\theta$ as it does in this case.  Therefore, the lower bound does not apply.

  \newpage
  \begin{problem}
  \probNum{5.5.5} Let $X$ have the pdf ${f_{X}(k;\theta) = \frac{(\theta - 1)^{k-1}}{\theta^{k}}}$, ${k=1,2,3,\ldots}$, ${\theta > 1}$, which is geometric (${p = 1 / \theta}$). For this pdf, ${E(X) = \theta}$ and ${\var{X} = \theta (\theta - 1)}$ (see Theorem~4.4.1).  Is the statistic~$\bar{X}$ efficient?
\end{problem}

\begin{align*}
  \ln f_X &= (k-1)\ln(\theta - 1) - k \ln(\theta) \\
  \frac{\partial \ln f_X}{\partial X} &= \frac{k-1}{\theta - 1} - \frac{k}{\theta} \\
  \frac{\partial^2 \ln f_X}{\partial X^2} &= -\frac{k-1}{\left(\theta - 1\right)^2} + \frac{k}{\theta^{2}}
\end{align*}

Given the provided expectation, we see that the Cramer-Rao lower bound is:
\begin{align}
  -\frac{\theta-1}{\left(\theta - 1\right)^2} + \frac{\theta}{\theta^{2}} &= \frac{-1}{\theta - 1} + \frac{1}{\theta} \\
                                                                          &= \frac{-\theta}{\theta(\theta - 1)} + \frac{\theta - 1}{\theta(\theta - 1)} \\
                                                                          &= \frac{-1}{\theta(\theta - 1)}
\end{align}

This makes the Cramer-Rao lower bound~$\frac{\theta(\theta - 1)}{n}$.  Given the above variance, the sample mean is~an efficient statistic since its variance is equivalent to the Cramer-Rao lower bound.

  % \newpage
  % \begin{problem}
  \probNum{5.6.3} If $\hat{\theta}$~is sufficient for~$\theta$, show that any one-to-one function of~$\hat{\theta}$ is also sufficient for~$\theta$.
\end{problem}

Define ${\hat{\theta}' \defeq f\left(\hat{\theta}\right)}$ where $f$~is one\=/to\=/one.

Since $\hat{\theta}$ is sufficient, it is possible to partition the likelihood function as:
\begin{equation}
  L(\theta) = \prod_{i=1}^{n} p_{X}(k_i;\theta) = p_{\hat{\theta}}(\theta_e;\theta)b(k_1,\ldots,k_n)\text{.}
\end{equation}

Since $f$ is one-to-one, its inverse $f^{-1}$, exists even if it is piecewise.  That means there is an equivalent formulation with substitution:
\begin{equation}
  L(\theta) =  p_{\hat{\theta}'}(f^{-1}(\theta_e);\theta)b(k_1,\ldots,k_n)\text{,}
\end{equation}
which by definition makes $\hat{\theta}'$ sufficient.

  % \newpage
  % \begin{problem}
  \probNum{5.6.4} Show that ${\hat{\sigma}^2 = \sum_{i=1}^{n} Y_i^2}$ is sufficient for $\sigma^2$ if ${Y_1,\ldots,Y_n}$ is a random sample from a normal pdf with ${\mu = 0}$.
\end{problem}

\begin{align}
  L(\theta) &= \left(2\pi\sigma^2\right)^{-\frac{n}{2}} e^{-\frac{\sum_{i=1}^{n} (x_i - \mu)^2}{2\sigma^2}} \\
            &= \left(2\pi\sigma^2\right)^{-\frac{n}{2}} e^{-\frac{\sum_{i=1}^{n} x_{i}^{2}}{2\sigma^2}} & \mu = 0 \\
            &= \underbrace{\left(2\pi\sigma^2\right)^{-\frac{n}{2}} e^{-\frac{\hat{\sigma}^2}{2\sigma^2}}}_{p_{\hat{\sigma}^2}(\sigma^{2}_e;\sigma^2)} \underbrace{1}_{b(Y_1,\ldots,Y_n)}
\end{align}

  % \newpage
  % \begin{problem}
  \probNum{5.6.7} Suppose a random sample of size~$n$ is drawn from the~pdf
  \begin{equation*}
    f_{Y}(y;\theta) = e^{-(y-\theta)}\text{,  } \theta \leq y
  \end{equation*}
\end{problem}

\begin{subproblem}
  Show that ${\hat{\theta} = Y_{\min}}$ is sufficient for the threshold parameter~$\theta$.
\end{subproblem}
\begin{align}
  L(\theta) &= \prod_{i=1}^{n} e^{-(y_i - \theta)} \mathbbm{1}\sbrack{y_i \geq \theta} \label{eq:HW04:P04:Indicator}\\
            &= \underbrace{\bigg(e^{n\theta} \mathbbm{1}\sbrack{Y_{\min} \geq \theta} \bigg)}_{p_{\hat{\theta}}(Y_{\min};\theta)} \underbrace{\left( e^{-\sum_{i=1}^{n} y_i}  \right)}_{b(Y_1,\ldots,Y_{n})}
\end{align}

\begin{subproblem}
  Show that $Y_{\max}$ is not sufficient for~$\theta$.
\end{subproblem}

Returning to Eq.~\ref{eq:HW04:P04:Indicator}, it is clear that the indicator function~$\mathbbm{1}\sbrack{\ldots}$ terms must be a component in~$p_{\hat{\theta}}$ since each indicator contains~$\theta$ and~$y_i$.  However, it is impossible to formulate the indicators in terms of~$Y_{\max}$ as it was possible for~${Y_{\min}}$.  Hence it is not sufficient by Theorem~5.6.1.

  % \newpage
  % \begin{problem}
  \probNum{5.7.01} How large a sample must be taken from a normal pdf where ${\expect{Y} = 18}$ in order to guarantee that ${\hat{\mu}_n = \bar{Y}_n = \frac{1}{n} \sum_{i=1}^{n} Y_i}$ has a 90\%~probability of lying somewhere in the interval~${\sbrack{16,20}}$?  Assume that ${\sigma = 5.0}$.
\end{problem}

To have 90\% confidence, each of the tails should have 5\%~of the probability mass.  Using the $Z$\=/table, this corresponds to ${Z \approx 1.645}$.

\begin{align}
  Z \approx 1.645 &= \frac{X - \mu}{\sigma} \\
                  &= \frac{2\sqrt{n}}{5.0} \\
                n &= \ceil{16.91} \\
                  &= \boxed{17} \text{.}
\end{align}



  % \newpage
  % \begin{problem}
  Let ${Y_1,Y_2,\ldots,Y_n}$ be a random sample of size~$n$ from a normal pdf having~${\mu = 0}$.  Show that ${S^2 = \frac{1}{n} \sum_{i=1}^{n} Y_{i}^2}$ is a consistent estimator for ${\sigma^{2} = \text{Var}(Y)}$.
\end{problem}

\begin{align}
  \expect{S^2} &= \expect{\frac{1}{n}\sum_{i=1}^{n} Y_{i}^{2}} \\
               &= \frac{1}{n} \sum_{i=1}^{n} \expect{Y_{i}^{2}} \\
               &= \frac{1}{n} \sum_{i=1}^{n} \left( \sigma^2 - \mu^2 \right) \\
               &= \frac{n\sigma^{2}}{n} & \mu = 0\\
               &= \boxed{\sigma^2}
\end{align}

  % \newpage
  % \begin{problem}
  Let ${Y_1,Y_2,\ldots,Y_n}$ be a random sample of size~$n$ from from the exponential pdf, ${f_Y(y;\lambda) = \lambda e^{-\lambda y} \text{, } y > 0}$.
\end{problem}

\begin{subproblem}
  Show that ${\hat{\lambda}_n = Y_1}$ is not consistent for~$\lambda$.
\end{subproblem}

\begin{subproblem}
  Show that ${\hat{\lambda}_n = \sum_{i=1}^{n} Y_i}$ is not a consistent for~$\lambda$.
\end{subproblem}

\end{document}
