\documentclass{report}

\newcommand{\name}{Zayd Hammoudeh}
\newcommand{\course}{MATH562}
\newcommand{\assnName}{Homework~\#7}
\newcommand{\dueDate}{March~6,~2020}

\usepackage[margin=1in]{geometry}
\usepackage[skip=4pt]{caption}      % ``skip'' sets the spacing between the figure and the caption.
\usepackage{tikz}
\usetikzlibrary{arrows.meta,decorations.markings,shadows,positioning,calc}
\usepackage{pgfplots}               % Needed for plotting
\pgfplotsset{compat=newest}
\usepgfplotslibrary{fillbetween}    % Allow for highlighting under a curve
\usepackage{amsmath}                % Allows for piecewise functions using the ``cases'' construct
\usepackage{bbm}                    % Enables \mathbbm{1}
\usepackage{siunitx}                % Allows for ``S'' alignment in table to align by decimal point

\usepackage[obeyspaces,spaces]{url} % Used for typesetting with the ``path'' command
\usepackage[hidelinks]{hyperref}    % Make the cross references clickable hyperlinks
\usepackage[bottom]{footmisc}       % Prevents the table going below the footnote
\usepackage{nccmath}                % Needed in the workaround for the ``aligncustom'' environment
\usepackage{amssymb}                % Used for black QED symbol
\usepackage{bm}                     % Allows for bolding math symbols.
\usepackage{tabto}                  % Allows to tab to certain point on a line
\usepackage{float}
\usepackage{subcaption}             % Allows use of the ``subfigure'' environment
\usepackage{enumerate}              % Allow enumeration other than just numbers

\usepackage[noend]{algpseudocode}
\usepackage[Algorithm,ruled]{algorithm}
\algnewcommand\algorithmicforeach{\textbf{for each}}
\algdef{S}[FOR]{ForEach}[1]{\algorithmicforeach\ #1\ \algorithmicdo}

%---------------------------------------------------%
%     Define Distances Used for Document Margins    %
%---------------------------------------------------%

\newcommand{\hangindentdistance}{1cm}
\newcommand{\defaultleftmargin}{0.25in}
\newcommand{\questionleftmargin}{-.5in}

\setlength{\parskip}{1em}
\setlength{\oddsidemargin}{\defaultleftmargin}

%---------------------------------------------------%
%      Configure the Document Header and Footer     %
%---------------------------------------------------%

% Set up page formatting
\usepackage{todonotes}
\usepackage{fancyhdr}                   % Used for every page footer and title.
\pagestyle{fancy}
\fancyhf{}                              % Clears both the header and footer
\renewcommand{\headrulewidth}{0pt}      % Eliminates line at the top of the page.
\fancyfoot[LO]{\course\ -- \assnName}   % Left
\fancyfoot[CO]{\thepage}                % Center
\fancyfoot[RO]{\name}                   % Right

%---------------------------------------------------%
%           Define the Title Page Entries           %
%---------------------------------------------------%

\title{\textbf{\course\ -- \assnName}}
\author{\name}

%---------------------------------------------------%
% Define the Environments for the Problem Inclusion %
%---------------------------------------------------%

\usepackage{scrextend}
\newcounter{problemCount}
\setcounter{problemCount}{0}  % Reset the subproblem counter

\newcounter{subProbCount}[problemCount]   % Reset subProbCount any time problemCount changes.
\renewcommand{\thesubProbCount}{\alph{subProbCount}}  % Make it so the counter is referenced as a number

\newenvironment{problemshell}{
  \begin{addmargin}[\questionleftmargin]{0em}
    \par%
    \medskip
    \leftskip=0pt\rightskip=0pt%
    \setlength{\parindent}{0pt}
    \bfseries
  }
  {
    \par\medskip
  \end{addmargin}
}
\newenvironment{problem}
{%
  \refstepcounter{problemCount} % Increment the subproblem counter.  This must be before the exercise to ensure proper numbering of claims and lemmas.
  \begin{problemshell}
    \noindent \textit{Exercise~\#\arabic{problemCount}} \\
  }
  {
  \end{problemshell}
  %  \setcounter{subProbCount}{0} % Reset the subproblem counter
}
\newenvironment{subproblem}
{%
  \begin{problemshell}
    \refstepcounter{subProbCount} % Increment the subproblem counter
    \setlength{\leftskip}{\hangindentdistance}
    % Print the subproblem count and offset to the left
    \hspace{-\hangindentdistance}(\alph{subProbCount}) \tabto{0pt}
  }
  {
  \end{problemshell}
}

% Change interline spacing.
\renewcommand{\baselinestretch}{1.1}
\newenvironment{aligncustom}
{ \csname align*\endcsname % Need to do this instead of \begin{align*} because of LaTeX bug.
  \centering
}
{
  \csname endalign*\endcsname
}


%---------------------------------------------------%
% Define the Environments for the Problem Inclusion %
%---------------------------------------------------%

\usepackage{amsthm}       % Allows use of the ``proof'' environment.

% Number lemmas and claims using the problem count
\newtheorem{claim}{Claim}[problemCount]
\newtheorem{lemma}{Lemma}[problemCount]

%---------------------------------------------------%
%       Define commands related to managing         %
%    floats (e.g., images) across multiple pages    %
%---------------------------------------------------%

\usepackage{placeins}     % Allows \FloatBarrier

% Prevent preceding floats going to this page
\newcommand{\floatnewpage}{\FloatBarrier\newpage}

% Add the specified input file and prevent any floated figures/tables going onto the same page as new input
\newcommand{\problemFile}[1]{
  \floatnewpage
  \input{#1}
}

\newcommand{\probNum}[1]{(\textnormal{Problem: #1})}

\newcommand{\etal}{~et~al.}

% Used for including standalone docs
\usepackage{standalone}

% Imported via UltiSnips
% Unbreakable dash:
%  Words hyphened with these dashes can also be broken at other positions than the dash
%    \-/ hyphen
%    \-- en-dash
%    \--- em-dash
%    extdash unbreakable dashes
%
%  No line breaks possible at the hyphen
%    \=/ hyphen
%    \== en-dash
%    \=== em-dash
\usepackage[shortcuts]{extdash}

% Imported via UltiSnips
\usepackage{color}
\newcommand{\colortext}[2]{{\color{#1} #2}}
\newcommand{\red}[1]{\colortext{red}{#1}}
\newcommand{\blue}[1]{\colortext{blue}{#1}}
\newcommand{\green}[1]{\colortext{green}{#1}}

% Imported via UltiSnips
\usepackage{amsmath}
\DeclareMathOperator*{\argmax}{arg\,max}
\DeclareMathOperator*{\argmin}{arg\,min}
\usepackage{amsfonts}  % Used for \mathbb and \mathcal
\usepackage{amssymb}

% Imported via UltiSnips
\usepackage{mathtools} % for "\DeclarePairedDelimiter" macro
% \swapifbranches changes unstarred paired delimiters to starred and
% vice versa.  This means by default, paired delimiters have the star.
\usepackage{etoolbox}
\newcommand\swapifbranches[3]{#1{#3}{#2}}
\makeatletter
\MHInternalSyntaxOn
\patchcmd{\DeclarePairedDelimiter}{\@ifstar}{\swapifbranches\@ifstar}{}{}
\MHInternalSyntaxOff
\makeatother
% Place after swap to ensure swap star
\DeclarePairedDelimiter{\sbrack}{\lbrack}{\rbrack}
\DeclarePairedDelimiter{\floor}{\lfloor}{\rfloor}
\DeclarePairedDelimiter{\ceil}{\lceil}{\rceil}
\DeclarePairedDelimiter{\abs}{\lvert}{\rvert}
\DeclarePairedDelimiter{\norm}{\lVert}{\rVert}
\usepackage{bm}
\DeclarePairedDelimiterX\set[1]\lbrace\rbrace{#1}
\DeclarePairedDelimiterX\setbuild[2]\lbrace\rbrace{#1 \bm: #2}
\newcommand{\setint}[1]{{\sbrack{#1}}}
\newcommand{\func}[3]{{#1:#2\rightarrow#3}}
% \newcommand{\defeq}{\stackrel{\mathclap{\mbox{\tiny def}}}{=}}
\newcommand{\defeq}{\coloneqq}
\newcommand{\fedeq}{\eqqcolon}
\newcommand{\expect}[1]{\mathbb{E}\sbrack{#1}}
% Expectation with the subscript defining the distribution
\newcommand{\expectS}[2]{\mathbb{E}_{#1}\sbrack{#2}}

% Allow numbering in align*
\newcommand{\numberthis}{\addtocounter{equation}{1}\tag{\theequation}}

\newcommand{\ints}{\mathbb{Z}}
\newcommand{\nats}{\mathbb{N}}
\newcommand{\real}{\mathbb{R}}
\newcommand{\realnn}{\real_{{\geq}0}}  % Set of non-negative real numbers

\newcommand{\iidsim}{\stackrel{\mathclap{\mbox{\tiny i.i.d.}}}{\sim}}

\newcommand{\normaldist}[2]{{\mathcal{N}\mathopen{}\left(#1,#2\right)\mathclose{}}}

% Imported via UltiSnips
\usepackage{array}  % Provides a way add a \centering command to a p-column
\usepackage{arydshln}  % Introduces hdashline & cdashline
\usepackage{bigdelim}
\usepackage{booktabs}
\usepackage{multirow}
\usepackage{makecell}  % Needed for multirowcell

% % Imported via UltiSnips
\usepackage{amsthm}
\newtheorem{theorem}{Theorem}[section]
% \newtheorem{corollary}{Corollary}[theorem]  % Corollary number derives from theorem
% \newtheorem{lemma}[theorem]{Lemma}  % Lemma and theorem share same counter
% \newtheorem{claim}[theorem]{Claim}  % Same numbering as lemma and theorem
% \newtheorem*{remark}{Remark}
% \newtheorem*{note}{Note}
% \newtheoremstyle{definition}  % <name>
% {3pt}   % <Space above>
% {3pt}   % <Space below>
% % {\itshape}     % <Body font>
% {\normalfont}   % <Body font>
% {}      % <Indent amount>
% {\bfseries} % <Theorem head font>
% {:}     % <Punctuation after theorem head>
% {.5em}  % <Space after theorem head>
% {}      % <Theorem head spec (can be left empty, meaning `normal')>
% \theoremstyle{definition}
% \newtheorem{definition}{Def.}[section]

% % Imported via UltiSnips
% \usepackage[noend]{algpseudocode}
\usepackage[Algorithm,ruled]{algorithm}
% \algnewcommand\algorithmicforeach{\textbf{for each}}
% \algdef{S}[FOR]{ForEach}[1]{\algorithmicforeach\ #1\ \algorithmicdo}
% \newcommand{\algin}[1]{\hspace*{\algorithmicindent} \textbf{Input} #1\\}
% \newcommand{\algin}[1]{\textbf{Input} #1}
% \newcommand{\algout}[1]{\hspace*{\algorithmicindent} \textbf{Output} #1}

% Imported via UltiSnips
\usepackage{tikz}
\usetikzlibrary{arrows,decorations.markings,shadows,positioning,calc,backgrounds,shapes}

\usepackage{pgfplots}
\pgfplotsset{compat=1.13}
\usepackage{pgfplotstable}
% \usepackage{subcaption}  % Cannot be used with subfigure

% Handle empty parameters
\usepackage{xifthen}
\newcommand{\ifempty}[3]{%
  \ifthenelse{\isempty{#1}}{#2}{#3}%
}


\newcommand{\thhat}[1]{\hat{\theta}_{#1}}
\newcommand{\var}[1]{\text{Var}\left(#1\right)}

\begin{document}
  \maketitle

  \noindent
  \textbf{Name}: \name\\
  \textbf{Course}: \course\\
  \textbf{Assignment}: \assnName\\
  \textbf{Due Date}: \dueDate

  \noindent
  \textbf{Other Student Discussions}: I discussed the problems in this homework with the following student(s) below.  All write-ups were prepared independently.
  \vspace{-1em}
  \begin{itemize}
    \item Viet Lai
  \end{itemize}

  \newpage
  \begin{problem}
  \probNum{7.4.7}~Cell phones emit radio frequency energy that is absorbed by the body when the phone is next to the ear and may be harmful.  The table in the next column gives the absorption rate for a sample of twenty high-radio cell phones.  Construct a 90\%~confidence interval for the true average cell phone absorption rate.
\end{problem}

The number of samples is ${n=20}$ so there are 19~degrees of freedom.  The mean is~1.4255.  The sample standard deviation is:
\begin{equation}\label{eq:SampleStdDev}
  S = \sqrt{\frac{\sum_{i=1}^{n} (X_i - \bar{X})}{n - 1}}\text{.}
\end{equation}
\noindent
${S = 0.056}$. For~${\alpha = 0.9}$, ${T_{\alpha / 2} = 1.7291}$.

The confidence interval is ${1.4255 \pm \frac{T \cdot S}{\sqrt{n}}}$.  The final range is ${[1.404,1.447]}$.

  \newpage
  \begin{problem}
  \probNum{7.4.8}~The following table lists the typical cost of repairing the bumper of a moderately priced midsize car damaged by a corner collision at 3~mph.  Use these observations to construct a 95\%~confidence interval for~$\mu$, the true average repair cost for all such automobiles with similar damage.  The sample standard deviation for these data is ${s = \$369.02}$.
\end{problem}

${n = 16}$. A confidence interval is two sided; with ${\alpha = 0.95}$, ${t_{0.025,15} = 2.131}$.  The sample mean is ${\bar{\mu} = 1516}$.

Therefore, the confidence interval is:
\begin{align}
  \bar{\mu} \pm \frac{T \cdot s}{\sqrt{n}} &= 1516 \pm \frac{2.131 \cdot 369.02}{\sqrt{16}} \\
                                           &= 1516 \pm 196.6 \text{.}
\end{align}

The range is therefore: $\sbrack{1319.4,1712.6}$.

  \newpage
  \begin{problem}
  \probNum{7.4.11}~In a nongeriatric population, platelet counts ranging from~140 to~440 (thousands per mm\textsuperscript{3} of blood) are considered ``normal.'' The following are the platelet counts record for twenty-four female nursing home residents~(180).

  Use the following sums:

  \begin{align}
    \sum_{i=1}^{24} y_i = 4645 &\text{ and }  &\sum_{i=1}^{24} y_i^{2} = 959,265 \text{.}
  \end{align}

  How does the definition of ``normal'' above compare with the 90\%~confidence interval?
\end{problem}

${\bar{\mu} = 193.5}$. ${n = 24}$.  For ${\alpha = 0.9}$, ${T_{0.05,23} = 1.714}$. The sample standard deviation is:
\begin{align}
  s^{2} &= \frac{\sum_{i=1}^{n} \left(x_i - \bar{x}\right)^2}{n-1} \\
        &= \frac{\sum_{i=1}^{n} \left(x^2_{i} - 2x_i \bar{x} + \bar{x}^2\right)}{n-1} \\
        &= \frac{\sum_{i=1}^{n} \left(x^2_{i}\right) - 2n\bar{x}^2 + n\bar{x}^2}{n-1} \\
        &= \frac{\sum_{i=1}^{n} \left(x^2_{i}\right) - n\bar{x}^2}{n-1} \\
        &= \frac{959,265 - 24 \cdot \left(193.5\right)^2}{23} \\
        &= 2637
\end{align}

The confidence interval is then:
\begin{align}
  \bar{\mu} \pm \frac{T_{\alpha / 2, \text{df}} \cdot s}{\sqrt{n}} &= 193.5 \pm \frac{1.714 \cdot 51.35}{\sqrt{24}} \\
                                                            &= 193.5 \pm 18.0 \text{.}
\end{align}

Therefore the confidence interval is~$\sbrack{175.5,211.5}$.

  \newpage
  \begin{problem}
  If a normally distributed sample of size ${n = 16}$ produces a 95\% confidence interval for~$\mu$ from~44.7 to~49.9, what are the values of~$\bar{y}$ and~$s$?
\end{problem}

$\bar{y}$~is always the center of the confidence interval so it is
\begin{equation}
  \bar{y} = \frac{44.7 + 49.9}{2} = \boxed{47.3}\text{.}
\end{equation}

${T_{1-\alpha/2,15} = 2.131}$.  The sample mean is then:
\begin{align}
  \frac{T_{0.025,15} \cdot s}{\sqrt{16}} &= 2.6 \\
  s &= \boxed{4.88} \text{.}
\end{align}

  \newpage
  \begin{problem}
  \probNum{7.4.17}~Recall the \textit{Bacillus subtilis} data in Question~5.3.2.  Test the null hypothesis that exposure to the enzyme does not affect a worker's respiratory capacity (as measured by the FEV\textsubscript{1}/VC ration).  Use a one side $H_1$ and let ${\alpha = 0.05}$.  Assume that $\sigma$~is not known.
\end{problem}

Using the data on page~306, ${\bar{y} = 0.766}$.  Also, ${s = 0.0859}$.  ${n = 19}$.  ${T_{0.05,18} = 1.7341}$.

The null hypothesis is rejected if
\begin{equation}
  0.766 = \bar{y} < 0.8 - \frac{1.7341 \cdot 0.0859}{\sqrt{19}} = 0.800 - 0.0342 = 0.7658 \text{.}
\end{equation}
There is \textbf{not} rejected.

  \newpage
  \begin{problem}
  \probNum{7.4.19}~MBAs R US advertises that its program increases a person's score on the GMAT by an average of forty points.  As a way of checking the validity of that claim, a consumer watchdog group hired fifteen students to take both the review course and the~GMAT\@.  Prior to starting the course, the fifteen students were given a diagnostic test that predicted how well they would do on the GMAT in the absence of any special training.  The following table gives each student's actual GMAT~score minus his or her predicted score.  Set up and carry out an appropriate hypothesis test.  Use the 0.05~level of significance.
\end{problem}

\noindent
$H_0: \mu = 40$\\
$H_1: \mu < 40$  A one-sided test is appropriate since if the improvement is greater than the advertised~40 points, that is not a concern.

${n = 15}$ so ${T_{0.05,14} = 1.7613}$.  The sample standard deviation is:
\begin{equation}
  s = \sqrt{\frac{\sum_{i=1}^{15} \left(y_{i}^2\right) - n \bar{y}^2}{n-1}} = \sqrt{\frac{20966 - 15 * 37.066^2}{14}} = 5.05\text{.}
\end{equation}

The null hypothesis is rejected if:
\begin{equation}
  T < \mu - \frac{T_{0.05,14} \cdot \sigma}{\sqrt{n}} = 40 - \frac{1.7613 \cdot 5.05}{\sqrt{15}} = 37.7\text{.}
\end{equation}
${\bar{y} = 37.066}$ so the null hypothesis is \underline{rejected}.

  \newpage
  \begin{problem}
  \probNum{7.4.21}~A manufacturer of pipe for laying underground electrical cables is concerned about the pipe's rate of corrosion and whether a special coating may retard that rate. As a way of measuring corrosion, the manufacturer examines a short length of pipe and records the depth of the maximum pit. The manufacturer's tests have shown that in a year's time in the particular kind of soil the manufacturer must deal with, the average depth of the maximum pit in a foot of pipe is 0.0042~inch.  To see whether that average can be reduced, ten pipes are coated with a new plastic and buried in the same soil. After one year, the following maximum pit depths are recorded in inches:

  \begin{center}
    \begin{tabular}{c}
      \toprule
      0.0039 \\
      0.0041 \\
      0.0038 \\
      0.0044 \\
      0.0040 \\
      0.0036 \\
      0.0034 \\
      0.0036 \\
      0.0046 \\
      0.0036 \\\bottomrule
    \end{tabular}
  \end{center}

  \noindent
  Given that the sample standard deviation for these measures is 0.000383~inch, can it be concluded at the ${\alpha= 0.05}$~level of significance that the plastic coating is beneficial?
\end{problem}

${\bar{y} = 0.0039}$, ${n= 10}$, ${T_{0.05,9} = 1.8331}$. The null hypothesis is rejected if:
\begin{equation}
  \bar{y} \leq \mu_{0} - \frac{T_{\alpha,n-1} \cdot s}{\sqrt{n}} = 0.0042 - \frac{1.8331 \cdot 0.000383}{\sqrt{10}} = 0.00398\text{.}
\end{equation}
\noindent
With ${\bar{y} = 0.0039}$, the null hypothesis is \underline{rejected}.

This indicates that the expected mean value for the geriatric population is on the low end of the ``normal'' range.

  \newpage
  \begin{problem}
  \probNum{7.4.27}~On which of the following datasets would you be reluctant to do a $t$\=/test?  Explain?
\end{problem}

I would be reluctant to do a $t$\=/test on~(c). The data does not appear to be normally distributed which is the assumption of a $t$\=/test.  The dataset is also small; a large dataset may even be approximately normal for large~$n$.

  \newpage
  \begin{problem}
  \probNum{7.5.1}~Use the Appendix Table~A.3 to find the following cutoff values and indicate their location on the graph of the approximate chi square distribution.
\end{problem}

\begin{subproblem}
  $\chi_{.95,14}^{2}$
\end{subproblem}

23.685, it is on the far right extreme of the graph.

\begin{subproblem}
  $\chi_{.90,2}^{2}$
\end{subproblem}

4.605, it is on the far right extreme of the graph.

\begin{subproblem}
  $\chi_{.025,9}^{2}$
\end{subproblem}

2.700, it is on the far left extreme of the graph.

  \newpage
  \begin{problem}
  \probNum{7.5.9}~How long sporting events last is quite variable.  This variability can cause problems for TV~broadcasters, since the amount of commercials and commentator blather varies with the length of the event.  As an example of this variability, the table below gives the lengths of a random sample of middle\=/round contents at the Wimbeldon Championships in women's tennis.
\end{problem}

For the data, ${s^2 = 742.38}$.

\begin{subproblem}
  Assume that match lengths are normally distributed. Use Theorem~7.5.1 to construct a 95\%~confidence interval for the standard deviation of match lengths.
\end{subproblem}

\begin{align}
  \left(\sqrt{\frac{(n-1)s^2}{\chi^{2}_{1-\alpha/2,n-1}}},\sqrt{\frac{(n-1)s^2}{\chi^{2}_{\alpha/2,n-1}}}\right) &= \left(\sqrt{\frac{15 * 742.38}{27.488}}, \sqrt{\frac{15 * 742.38}{6.262}}\right) \\
                                                                                                          &= \left(20.13,42.17\right)\text{.}
\end{align}

\begin{subproblem}
  Use the same data to construct the two one-sided 95\%~confidence intervals for~$\sigma$.
\end{subproblem}

The first one-sided confidence interval is:
\begin{align}
  \sigma_0 &\leq \sqrt{\frac{(n-1)  s^2}{\chi_{\alpha,n-1}^{2}}} \\
           &= \sqrt{\frac{15 * 742.38}{7.261}} \\
           &= 39.16 \text{,}
\end{align}
\noindent
making the interval~${(0, 39.16)}$.

The second one-sided confidence interval is:
\begin{align}
  \sigma_0 &\geq \sqrt{\frac{(n-1)  s^2}{\chi_{1-\alpha,n-1}^{2}}} \\
           &= \sqrt{\frac{15 * 742.38}{24.996}} \\
           &= 21.11 \text{,}
\end{align}
\noindent
making the interval~${(21.11,\infty)}$.

  \newpage
  \begin{problem}
  \probNum{7.5.10}~How much interest certificates of deposit~(CDs) pay varies by financial institution and also by length of the investment.  A large sample of national one-year CD offerings in~2009 showed an average interest rate of~1.84 and a standard deviation of~${\sigma  = 0.262}$.  A five\=/year CD ties up an investor's money, so it usually pays a higher rate of interest.  However, higher rates might cause more variability.  The table lists the five\=/year CD rate offerings from ${n = 10}$ banks in the northeast United States.  Find a 95\%~confidence interval for the standard deviation of five\=/year CD rates.  Do these data suggest that interest rates for five\=/year CDs are more variable than those for one\=/year certificates.
\end{problem}

${\chi_{0.025,9}^2 = 2.700}$, ${\chi_{0.975,9}^2 = 19.023}$, ${n =10}$ and ${s^2 = 0.192}$.

The confidence interval is:
\begin{align}
  \left(\sqrt{frac{(n-1)s^2}{\chi_{1-\alpha^2,n-1}}},\sqrt{\frac{(n-1)s^2}{\chi_{1-\alpha^2,n-1}}}\right) &= \left(\sqrt{\frac{9 * 0.192}{19.023}}, \sqrt{\frac{9 * 0.192}{2.700}}\right) \\
                                                                                                          &= \left(0.301,0.603\right)\text{.}
\end{align}
Since ${\sigma = 0.262}$ on one year CDs, data indicates 5~year CDs are more variable.

\end{document}
