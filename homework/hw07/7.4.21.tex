\begin{problem}
  \probNum{7.4.21}~A manufacturer of pip for laying underground electrical cables is concerned about the pipe's rate of corrosion and whether a special coating may retard that rate. As a way of measuring corrosion, the manufacturer examines a short length of pipe and records the depth of the maximum pit. The manufacturer's tests have shown that in a year's time in the particular kind of soil the manufacturer must deal with, the average depth of the maximum pit in a foot of pipe is 0.0042~inch.  To see whether that average can be reduced, ten pipes are coated with a new plastic and buried in the same soil. After one year, the following maximum pit depths are recorded in inches:
  \begin{tabular}{c}
    \toprule
    0.0039 \\
    0.0041 \\
    0.0038 \\
    0.0044 \\
    0.0040 \\
    0.0036 \\
    0.0034 \\
    0.0036 \\
    0.0046 \\
    0.0036 \\\bottomrule
  \end{tabular}
  \noindent
  Given that the sample standard deviation for these measures is 0.000383~inch, can it be concluded at the ${\alpha= 0.05}$~level of significance that the plastic coating is beneficial?
\end{problem}

${\bar{y} = 0.0039}$, ${n= 10}$, ${T_{0.05,9} = 1.8595}$. The null hypothesis is rejected if:
\begin{equation}
  \bar{y} \leq \mu_{0} - \frac{T_{1-\alpha,n-1} \cdot s}{\sqrt{n}} = 0.0042 - \frac{1.8595 \cdot 0.000383}{\sqrt{10}} = 0.00397\text{.}
\end{equation}
\noindent
With ${\bar{y} = 0.0039}$, the null hypothesis is \underline{rejected}.
