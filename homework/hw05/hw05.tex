\documentclass{report}

\newcommand{\name}{Zayd Hammoudeh}
\newcommand{\course}{MATH562}
\newcommand{\assnName}{Homework~\#5}
\newcommand{\dueDate}{February~21,~2020}

\usepackage[margin=1in]{geometry}
\usepackage[skip=4pt]{caption}      % ``skip'' sets the spacing between the figure and the caption.
\usepackage{tikz}
\usetikzlibrary{arrows.meta,decorations.markings,shadows,positioning,calc}
\usepackage{pgfplots}               % Needed for plotting
\pgfplotsset{compat=newest}
\usepgfplotslibrary{fillbetween}    % Allow for highlighting under a curve
\usepackage{amsmath}                % Allows for piecewise functions using the ``cases'' construct
\usepackage{bbm}                    % Enables \mathbbm{1}
\usepackage{siunitx}                % Allows for ``S'' alignment in table to align by decimal point

\usepackage[obeyspaces,spaces]{url} % Used for typesetting with the ``path'' command
\usepackage[hidelinks]{hyperref}    % Make the cross references clickable hyperlinks
\usepackage[bottom]{footmisc}       % Prevents the table going below the footnote
\usepackage{nccmath}                % Needed in the workaround for the ``aligncustom'' environment
\usepackage{amssymb}                % Used for black QED symbol
\usepackage{bm}                     % Allows for bolding math symbols.
\usepackage{tabto}                  % Allows to tab to certain point on a line
\usepackage{float}
\usepackage{subcaption}             % Allows use of the ``subfigure'' environment
\usepackage{enumerate}              % Allow enumeration other than just numbers

\usepackage[noend]{algpseudocode}
\usepackage[Algorithm,ruled]{algorithm}
\algnewcommand\algorithmicforeach{\textbf{for each}}
\algdef{S}[FOR]{ForEach}[1]{\algorithmicforeach\ #1\ \algorithmicdo}

%---------------------------------------------------%
%     Define Distances Used for Document Margins    %
%---------------------------------------------------%

\newcommand{\hangindentdistance}{1cm}
\newcommand{\defaultleftmargin}{0.25in}
\newcommand{\questionleftmargin}{-.5in}

\setlength{\parskip}{1em}
\setlength{\oddsidemargin}{\defaultleftmargin}

%---------------------------------------------------%
%      Configure the Document Header and Footer     %
%---------------------------------------------------%

% Set up page formatting
\usepackage{todonotes}
\usepackage{fancyhdr}                   % Used for every page footer and title.
\pagestyle{fancy}
\fancyhf{}                              % Clears both the header and footer
\renewcommand{\headrulewidth}{0pt}      % Eliminates line at the top of the page.
\fancyfoot[LO]{\course\ -- \assnName}   % Left
\fancyfoot[CO]{\thepage}                % Center
\fancyfoot[RO]{\name}                   % Right

%---------------------------------------------------%
%           Define the Title Page Entries           %
%---------------------------------------------------%

\title{\textbf{\course\ -- \assnName}}
\author{\name}

%---------------------------------------------------%
% Define the Environments for the Problem Inclusion %
%---------------------------------------------------%

\usepackage{scrextend}
\newcounter{problemCount}
\setcounter{problemCount}{0}  % Reset the subproblem counter

\newcounter{subProbCount}[problemCount]   % Reset subProbCount any time problemCount changes.
\renewcommand{\thesubProbCount}{\alph{subProbCount}}  % Make it so the counter is referenced as a number

\newenvironment{problemshell}{
  \begin{addmargin}[\questionleftmargin]{0em}
    \par%
    \medskip
    \leftskip=0pt\rightskip=0pt%
    \setlength{\parindent}{0pt}
    \bfseries
  }
  {
    \par\medskip
  \end{addmargin}
}
\newenvironment{problem}
{%
  \refstepcounter{problemCount} % Increment the subproblem counter.  This must be before the exercise to ensure proper numbering of claims and lemmas.
  \begin{problemshell}
    \noindent \textit{Exercise~\#\arabic{problemCount}} \\
  }
  {
  \end{problemshell}
  %  \setcounter{subProbCount}{0} % Reset the subproblem counter
}
\newenvironment{subproblem}
{%
  \begin{problemshell}
    \refstepcounter{subProbCount} % Increment the subproblem counter
    \setlength{\leftskip}{\hangindentdistance}
    % Print the subproblem count and offset to the left
    \hspace{-\hangindentdistance}(\alph{subProbCount}) \tabto{0pt}
  }
  {
  \end{problemshell}
}

% Change interline spacing.
\renewcommand{\baselinestretch}{1.1}
\newenvironment{aligncustom}
{ \csname align*\endcsname % Need to do this instead of \begin{align*} because of LaTeX bug.
  \centering
}
{
  \csname endalign*\endcsname
}


%---------------------------------------------------%
% Define the Environments for the Problem Inclusion %
%---------------------------------------------------%

\usepackage{amsthm}       % Allows use of the ``proof'' environment.

% Number lemmas and claims using the problem count
\newtheorem{claim}{Claim}[problemCount]
\newtheorem{lemma}{Lemma}[problemCount]

%---------------------------------------------------%
%       Define commands related to managing         %
%    floats (e.g., images) across multiple pages    %
%---------------------------------------------------%

\usepackage{placeins}     % Allows \FloatBarrier

% Prevent preceding floats going to this page
\newcommand{\floatnewpage}{\FloatBarrier\newpage}

% Add the specified input file and prevent any floated figures/tables going onto the same page as new input
\newcommand{\problemFile}[1]{
  \floatnewpage
  \input{#1}
}

\newcommand{\probNum}[1]{(\textnormal{Problem: #1})}

\newcommand{\etal}{~et~al.}

% Used for including standalone docs
\usepackage{standalone}

% Imported via UltiSnips
% Unbreakable dash:
%  Words hyphened with these dashes can also be broken at other positions than the dash
%    \-/ hyphen
%    \-- en-dash
%    \--- em-dash
%    extdash unbreakable dashes
%
%  No line breaks possible at the hyphen
%    \=/ hyphen
%    \== en-dash
%    \=== em-dash
\usepackage[shortcuts]{extdash}

% Imported via UltiSnips
\usepackage{color}
\newcommand{\colortext}[2]{{\color{#1} #2}}
\newcommand{\red}[1]{\colortext{red}{#1}}
\newcommand{\blue}[1]{\colortext{blue}{#1}}
\newcommand{\green}[1]{\colortext{green}{#1}}

% Imported via UltiSnips
\usepackage{amsmath}
\DeclareMathOperator*{\argmax}{arg\,max}
\DeclareMathOperator*{\argmin}{arg\,min}
\usepackage{amsfonts}  % Used for \mathbb and \mathcal
\usepackage{amssymb}

% Imported via UltiSnips
\usepackage{mathtools} % for "\DeclarePairedDelimiter" macro
% \swapifbranches changes unstarred paired delimiters to starred and
% vice versa.  This means by default, paired delimiters have the star.
\usepackage{etoolbox}
\newcommand\swapifbranches[3]{#1{#3}{#2}}
\makeatletter
\MHInternalSyntaxOn
\patchcmd{\DeclarePairedDelimiter}{\@ifstar}{\swapifbranches\@ifstar}{}{}
\MHInternalSyntaxOff
\makeatother
% Place after swap to ensure swap star
\DeclarePairedDelimiter{\sbrack}{\lbrack}{\rbrack}
\DeclarePairedDelimiter{\floor}{\lfloor}{\rfloor}
\DeclarePairedDelimiter{\ceil}{\lceil}{\rceil}
\DeclarePairedDelimiter{\abs}{\lvert}{\rvert}
\DeclarePairedDelimiter{\norm}{\lVert}{\rVert}
\usepackage{bm}
\DeclarePairedDelimiterX\set[1]\lbrace\rbrace{#1}
\DeclarePairedDelimiterX\setbuild[2]\lbrace\rbrace{#1 \bm: #2}
\newcommand{\setint}[1]{{\sbrack{#1}}}
\newcommand{\func}[3]{{#1:#2\rightarrow#3}}
% \newcommand{\defeq}{\stackrel{\mathclap{\mbox{\tiny def}}}{=}}
\newcommand{\defeq}{\coloneqq}
\newcommand{\fedeq}{\eqqcolon}
\newcommand{\expect}[1]{\mathbb{E}\sbrack{#1}}
% Expectation with the subscript defining the distribution
\newcommand{\expectS}[2]{\mathbb{E}_{#1}\sbrack{#2}}

% Allow numbering in align*
\newcommand{\numberthis}{\addtocounter{equation}{1}\tag{\theequation}}

\newcommand{\ints}{\mathbb{Z}}
\newcommand{\nats}{\mathbb{N}}
\newcommand{\real}{\mathbb{R}}
\newcommand{\realnn}{\real_{{\geq}0}}  % Set of non-negative real numbers

\newcommand{\iidsim}{\stackrel{\mathclap{\mbox{\tiny i.i.d.}}}{\sim}}

\newcommand{\normaldist}[2]{{\mathcal{N}\mathopen{}\left(#1,#2\right)\mathclose{}}}

% Imported via UltiSnips
\usepackage{array}  % Provides a way add a \centering command to a p-column
\usepackage{arydshln}  % Introduces hdashline & cdashline
\usepackage{bigdelim}
\usepackage{booktabs}
\usepackage{multirow}
\usepackage{makecell}  % Needed for multirowcell

% % Imported via UltiSnips
\usepackage{amsthm}
\newtheorem{theorem}{Theorem}[section]
% \newtheorem{corollary}{Corollary}[theorem]  % Corollary number derives from theorem
% \newtheorem{lemma}[theorem]{Lemma}  % Lemma and theorem share same counter
% \newtheorem{claim}[theorem]{Claim}  % Same numbering as lemma and theorem
% \newtheorem*{remark}{Remark}
% \newtheorem*{note}{Note}
% \newtheoremstyle{definition}  % <name>
% {3pt}   % <Space above>
% {3pt}   % <Space below>
% % {\itshape}     % <Body font>
% {\normalfont}   % <Body font>
% {}      % <Indent amount>
% {\bfseries} % <Theorem head font>
% {:}     % <Punctuation after theorem head>
% {.5em}  % <Space after theorem head>
% {}      % <Theorem head spec (can be left empty, meaning `normal')>
% \theoremstyle{definition}
% \newtheorem{definition}{Def.}[section]

% % Imported via UltiSnips
% \usepackage[noend]{algpseudocode}
\usepackage[Algorithm,ruled]{algorithm}
% \algnewcommand\algorithmicforeach{\textbf{for each}}
% \algdef{S}[FOR]{ForEach}[1]{\algorithmicforeach\ #1\ \algorithmicdo}
% \newcommand{\algin}[1]{\hspace*{\algorithmicindent} \textbf{Input} #1\\}
% \newcommand{\algin}[1]{\textbf{Input} #1}
% \newcommand{\algout}[1]{\hspace*{\algorithmicindent} \textbf{Output} #1}

% Imported via UltiSnips
\usepackage{tikz}
\usetikzlibrary{arrows,decorations.markings,shadows,positioning,calc,backgrounds,shapes}

\usepackage{pgfplots}
\pgfplotsset{compat=1.13}
\usepackage{pgfplotstable}
% \usepackage{subcaption}  % Cannot be used with subfigure

% Handle empty parameters
\usepackage{xifthen}
\newcommand{\ifempty}[3]{%
  \ifthenelse{\isempty{#1}}{#2}{#3}%
}


\newcommand{\thhat}[1]{\hat{\theta}_{#1}}
\newcommand{\var}[1]{\text{Var}\left(#1\right)}

\begin{document}
  \maketitle

  \noindent
  \textbf{Name}: \name\\
  \textbf{Course}: \course\\
  \textbf{Assignment}: \assnName\\
  \textbf{Due Date}: \dueDate

  % \noindent
  % \textbf{Other Student Discussions}: I discussed the problems in this homework with the following student(s) below.  All write-ups were prepared independently.
  % \vspace{-1em}
  % \begin{itemize}
  %   \item Viet Lai
  % \end{itemize}

  \newpage
  \begin{problem}
  \probNum{6.2.1}~State the decision rule that would be used to test the following hypotheses.  Evaluate the appropriate test statistic and state your conclusion.
\end{problem}

\begin{subproblem}
  $H_0$: ${\mu = 120}$ versus $H_1$: ${\mu < 120}$; ${\bar{y} = 114.2}$, ${n = 25}$, ${\sigma = 18}$, ${\alpha = 0.08}$
\end{subproblem}

\noindent
Reject~$H_0$ if ${\frac{\mu - \bar{y}}{\sigma / \sqrt{n}} = \frac{120 - 114.2}{18 / \sqrt{25}} > Z_{\alpha} = 1.405}$

\noindent
${Z = 1.61}$; \underline{Rejected}

\begin{subproblem}
  $H_0$: ${\mu = 42.9}$ versus $H_1$: ${\mu \ne 42.9}$; ${\bar{y} = 45.1}$, ${n = 16}$, ${\sigma = 3.2}$, ${\alpha = 0.01}$
\end{subproblem}

\noindent
Reject~$H_0$ if ${\frac{\abs{\mu - \bar{y}}}{\sigma / \sqrt{n}} = \frac{\abs{45.1 - 42.9}}{3.2 / \sqrt{16}} > Z_{\alpha/2} = 2.575}$

\noindent
${Z = 2.75}$; \underline{Rejected}

\begin{subproblem}
  $H_0$: ${\mu = 14.2}$ versus $H_1$: ${\mu > 14.2}$; ${\bar{y} = 15.8}$, ${n = 9}$, ${\sigma = 4.1}$, ${\alpha = 0.13}$
\end{subproblem}

\noindent
Reject~$H_0$ if ${\frac{\mu - \bar{y}}{\sigma / \sqrt{n}} = \frac{15.8 - 14.2}{4.1 / \sqrt{9}} > Z_{\alpha} = 1.123}$

\noindent
${Z = 1.170}$; \underline{Rejected}

  \newpage
  \begin{problem}
  An herbalist is experimenting with juices extracted from berries and roots that may have the ability to affect the Stanford-Binet IQ scores of students afflicted with mild cases of attention deficit disorder~(ADD).  A random sample of twenty-two children diagnosed with the condition have been drinking Berry Smart daily for two months. Past experience suggests that children with ADD~score an average of~95 on the IQ test with a standard deviation of~15. If the data are to be analyzed using the ${\alpha =0.06}$ level of significance, what values of ${\bar{y}}$~would cause ${H_0}$ to be rejected.  Assume that $H_1$ is two-sided.
\end{problem}

\noindent
${Z_{\alpha / 2} = -1.88}$. The null hypothesis of no affect is rejected when:
\begin{equation*}
  \frac{\abs{\bar{y} - 95}}{15 / \sqrt{22}} > 1.88\text{.}
\end{equation*}

Therefore, ${\bar{y} \leq 89.0}$ or ${\bar{y} \geq 101.0}$.

  \newpage
  \begin{problem}
  \probNum{6.2.3}
\end{problem}

\begin{subproblem}
  Suppose $H_0$: ${\mu = \mu_0}$ is rejected in favor of $H_1$: ${\mu = \mu_0}$ at the ${\alpha = 0.05}$~level of significance.  Would $H_0$~necessarily be rejected at the ${\alpha = 0.01}$~level of significance?
\end{subproblem}

\noindent
No.  ${\alpha = 0.01}$~is a stricter requirement so it will catch five~times less cases than ${\alpha = 0.05}$.

\begin{subproblem}
  Suppose $H_0$: ${\mu = \mu_0}$ is rejected in favor of $H_1$: ${\mu = \mu_0}$ at the ${\alpha = 0.01}$~level of significance.  Would $H_0$~necessarily be rejected at the ${\alpha = 0.05}$~level of significance?
\end{subproblem}

Yes.  ${\alpha = 0.05}$~is a superset of~${\alpha = 0.01}$.

  \newpage
  \begin{problem}
  \probNum{6.2.10}~As a class research project, Rosaura wants to see whether the stress of final exams elevates the blood pressures of freshmen women.  When they are not under any untoward duress, healthy eighteen-year-old women have systolic blood pressures that average 120mm~Hg with a standard deviation of 12mm~Hg.  If Rosaura finds that the average blood pressure for the fifty women in Statistics~101 on the day of the final exam is~125.2, what should she conclude.  Set up and test an appropriate hypothesis.
\end{problem}

\noindent
Since Rosaura is specifically testing for \textit{elevated} blood pressures, we are performing only a one-sided test with the null hypothesis being that testing does not raise student blood pressure.  The default value is ${\mu_0 = 120}$.  We will use ~${\alpha=0.05}$ as is common in my field.  I do not know what is common in her field.

\noindent
The hypothesis is rejected if:
\begin{align*}
  \frac{\bar{y} - \mu}{\sigma / \sqrt{n}} = \frac{125.2 - 120}{12 / \sqrt{50}} &> Z_{\alpha} = 1.645 \\
  3.06 &> 1.645\text{.}
\end{align*}

\noindent
The null hypothesis is rejected.

  \newpage
  \begin{problem}
  \probNum{6.3.2}~Efforts to find a genetic explanation for why certain people are right-handed and others left-handed have been largely unsuccessful. Reliable data are difficult to find because of environmental factors that also a child's ``handedness.''  To avoid that complication, researchers often study the analogous problem of ``pawedness'' in animals, where both genotypes and the environment can be partially controlled. In one such experiment, mice were put into a cage having a feeding tube that was equally accessible from the right or left.  Each mouse was then carefully watched over a number of feedings.  If it used its right paw more than half the time to activate the tube, it was defined to be ``right-pawed.'' Observations of this sort showed that 67\%~of mice belong to strain~A/J are right-pawed. A similar protocol was followed on a sample of thirty-five mice belonging to strain~A/HeJ. Of those thirty-five, a total of eighteen were eventually classified as right-pawed. Test whether proportion of right-pawed mice found in the A/HeJ~sample was significantly different from what was known about the A/H~strain.  Use a two-side alternative and let 0.05~be the probability associated with the critical region.
\end{problem}

\noindent
The assumed standard deviation is found from the Bernoulli distribution with~${p=0.67}$. The standard deviation is:
\begin{equation*}
  \sigma = \sqrt{p(1-p)} = \sqrt{0.67 * 0.33} \approx 0.470\text{.}
\end{equation*}

\noindent
The A/HeJ~sample will be rejected if:
\begin{equation*}
  \frac{\abs{\bar{y} - \mu}}{\sigma / \sqrt{n}} = \frac{\abs{\frac{18}{35} - 0.67}}{0.470 / \sqrt{35}} \geq Z_{\alpha /2} = 1.96 \text{.}
\end{equation*}

\noindent
${Z = 1.960 \geq 1.96}$ so the hypothesis is rejected.

  \newpage
  \begin{problem}
  \probNum{6.3.4}~Suppose $H_0$: ${p=0.45}$ is to be tested against $H_1$: ${p > 0.45}$ at the ${\alpha = 0.14}$ level of significance where ${p = P(\text{th trial ends in success})}$. If the sample size is two hundred, what is the smallest number of successes that will cause $H_0$~to be rejected?
\end{problem}

\noindent
${Z_{\alpha} = 1.08}$.  The standard deviation is:
\begin{equation*}
  \sigma = \sqrt{p(1-p)} = \sqrt{0.45 * 0.55} \approx 0.497\text{.}
\end{equation*}

\noindent
To reject, the observed probability of success must be
\begin{align*}
  \bar{p} &\geq y + Z_{\alpha} * \frac{\sigma}{\sqrt{n}} \\
          &= 0.45 + 1.08 \frac{0.497}{\sqrt{200}} \\
          &\approx 0.4880 \text{.}
\end{align*}
Therefore,
\begin{align*}
  n_{\text{smallest}} &= \ceil{0.4880 * 200} \\
                      &= \ceil{97.59} = 98\text{.}
\end{align*}

  \newpage
  \begin{problem}
  \probNum{6.3.8}~The following is a Minitab printout of the binomial pdf ${p_{X}(k) = \binom{9}{k} (0.6)^k (0.4)^{9-k}}$, ${k=0,1,\ldots,9}$.  Suppose that $H_0$: ${p=0.6}$ is to be tested against ${H_1}$: ${p > 0.6}$, and we wish the level of significance to be \textit{exactly}~${0.05}$.  Use Theorem~2.4.1 to combine two different critical regions into a single randomized decision rule for which~${\alpha = 0.05}$.
\end{problem}

If performing ${n=9}$ trials and seek a decision rule with ${\alpha = 0.05}$ \textit{exactly}, a hybrid decision rule is required.
\begin{itemize}
  \item \textbf{Rule~\#1}: $k=9$.  This occurs with probability $\sim0.01$.
  \item \textbf{Rule~\#2}: ${k \geq 8}$.  This occurs with probability $\sim0.07$.
\end{itemize}
\noindent
These two rules can be assigned a weight such that:
\begin{align*}
  0.07x + 0.01(1 - x) &= 0.05 \\
  0.06x &= 0.04 \\
  \boxed{x = \frac{2}{3}}\text{.}
\end{align*}
\noindent
Therefore, rule~\#1 is applied two-thirds of the time and rule~\#2 is applied one-third of the time.

  \newpage
  \begin{problem}
  \probNum{6.4.1}~Recall the ``Math for the Twenty-First Century'' hypothesis test done in Example~6.2.1. Calculate the power of that test when the true mean is~500.
\end{problem}

The eighty-six Bayview students averaged~502.  The national average was~494 with standard deviation of~124. ${\alpha = 0.05}$.

\begin{align*}
  \alpha &= \Pr\sbrack{\text{Reject } H_0 \vert H_1 \text{ is true}} \\
         &= \Pr\sbrack{\bar{Y} \geq \bar{y}* \vert \mu = 500} \\
         &= \Pr\sbrack{\frac{\bar{Y} - 500}{124 /\sqrt{86}} > \frac{\bar{y}* - 500}{124 /\sqrt{86}}} \\
         &= \Pr\sbrack{Z > \frac{\bar{y}* - 500}{124 /\sqrt{86}}} \\
         &= 0.05
\end{align*}

\noindent
If ${\alpha = 0.05}$, then ${Z = 1.645}$. Therefore, ${\bar{y}* = 122}$.

  \newpage
  \begin{problem}
  \probNum{6.4.3}~For the decision rule found in Question~6.2.2 to test $H_0$: ${\mu = 95}$ versus $H_1$: ${\mu \ne 95}$ at the ${\alpha = 0.06}$~level of significance, calculate ${1 - \beta}$ when~${\mu = 90}$.
\end{problem}

In 6.2.2, ${\sigma = 15}$ and ${n =22}$.
\begin{equation}
  \frac{\abs{\bar{Y} - 95}}{15 / \sqrt{22}} \geq Z_{\alpha/2} = 1.88
\end{equation}
\noindent
Therefore, ${\bar{Y} \leq 89.0}$ and ${\bar{Y} \geq 101.0}$.

\noindent
\begin{align*}
  1 - \beta &= \Pr\sbrack{\text{Reject } H_0 \vert H_1 \text{ is true}} \\
            &= \Pr\sbrack{\bar{Y} \leq 89.0 \vert \mu = 90} + \Pr\sbrack{\bar{Y} \geq 101.0 \vert \mu = 90} \\
            &= \Pr\sbrack{Z \leq \frac{-1}{15 / \sqrt{22}}} + \Pr\sbrack{Z \geq \frac{11}{15 / \sqrt{22}}} \\
            &= \Pr\sbrack{Z \leq -0.31} + \Pr\sbrack{Z \geq 3.44} \\
            &= 0.3783 + 0.0003 = \boxed{0.3786}\text{.}
\end{align*}

  \newpage
  \begin{problem}
  If $H_0$: ${\mu = 200}$ is to be tested against $H_1$: ${\mu < 200}$ at the ${\alpha = 0.10}$~level of significance based on a random sample of size~$n$ from a normal distribution where ${\sigma = 15.0}$, what is the smallest value for~$n$ that will make the power equal to at least~0.75 when~${\mu = 197}$?
\end{problem}

If ${\alpha = 0.10}$, then~${Z = 1.28}$.  For a power of~0.75, then ${Z=0.675}$.
\begin{align*}
  \frac{200 - 1.28 *\frac{15}{\sqrt{n}} - 197}{15 / \sqrt{n}} &\geq 0.675
  3 \sqrt(n) &\geq ( 0.675 + 1.28) * 15
  n \geq 95.05 \text{.}
\end{align*}

The sample size must be a whole number larger so~$\boxed{n = 96}$.

  \newpage
  \begin{problem}
  \probNum{6.4.13}~Suppose that a random sample of size~5 is drawn from a uniform pdf:
  \begin{equation}
    f_{Y}(y;\theta) = \begin{cases}
                        \frac{1}{\theta} & 0 < y < \theta \\
                        0 & \text{Otherwise}
                      \end{cases}\text{.}
  \end{equation}
  \noindent
  We wish to test
  \begin{center}
    $H_0$: $\theta = 2$ \\
    \text{versus} \\
    $H_1$: $\theta > 2$ \\
  \end{center}
  \noindent
  by rejecting the null hypothesis if~${y_{\max} \geq k}$. Find the value of~$k$ that makes the probability of committing a Type~I error equal to~0.05.
\end{problem}

A \textit{Type~1 error} is a rejection of~$H_0$ despite it being true.

\noindent
The CDF of $Y_{\max}$~is
\begin{equation*}
  F(y) = \frac{y^5}{\theta ^5}\text{.}
\end{equation*}
With ${F(y) = 0.95}$ and ${\theta = 2}$, the rejection point is ${\boxed{y = 1.980}}$.

  \newpage
  \begin{problem}
  \probNum{6.4.15}~A series of $n$~Bernoulli trials is to be observed as data for testing.
  \begin{center}
    $H_0$: $p = \frac{1}{2}$ \\
    \text{versus} \\
    $H_1$: $p > \frac{1}{2}$ \\
  \end{center}
  \noindent
  The null hypothesis will be rejected if~$k$, the observed number of successes, equals~$n$.  For what value of $p$~will the probability of committing a Type~II error equal~0.05.
\end{problem}

A \textit{Type~II} error is when null hypothesis~$H_0$ is not rejected although $H_1$~is true.

For $p$~to be rejected, then~${k \ne n}$.  Therefore:
\begin{align*}
  \beta = \Pr\sbrack{H_0 \text{ is accepted} \vert H_1 \text{ is true}} &= 1 - \Pr\sbrack{k = n} \\
            0.05 &= 1 - p^n \\
            0.95 &= p^n \\
            \boxed{0.95^{\frac{1}{n}} = p}
\end{align*}

\end{document}
