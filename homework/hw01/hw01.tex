\documentclass{report}

\newcommand{\name}{Zayd Hammoudeh}
\newcommand{\course}{MATH562}
\newcommand{\assnName}{Homework~\#1}
\newcommand{\dueDate}{January~17,~2020}

\usepackage[margin=1in]{geometry}
\usepackage[skip=4pt]{caption}      % ``skip'' sets the spacing between the figure and the caption.
\usepackage{tikz}
\usetikzlibrary{arrows.meta,decorations.markings,shadows,positioning,calc}
\usepackage{pgfplots}               % Needed for plotting
\pgfplotsset{compat=newest}
\usepgfplotslibrary{fillbetween}    % Allow for highlighting under a curve
\usepackage{amsmath}                % Allows for piecewise functions using the ``cases'' construct
\usepackage{siunitx}                % Allows for ``S'' alignment in table to align by decimal point

\usepackage[obeyspaces,spaces]{url} % Used for typesetting with the ``path'' command
\usepackage[hidelinks]{hyperref}    % Make the cross references clickable hyperlinks
\usepackage[bottom]{footmisc}       % Prevents the table going below the footnote
\usepackage{nccmath}                % Needed in the workaround for the ``aligncustom'' environment
\usepackage{amssymb}                % Used for black QED symbol
\usepackage{bm}                     % Allows for bolding math symbols.
\usepackage{tabto}                  % Allows to tab to certain point on a line
\usepackage{float}
\usepackage{subcaption}             % Allows use of the ``subfigure'' environment
\usepackage{enumerate}              % Allow enumeration other than just numbers

\usepackage[noend]{algpseudocode}
\usepackage[Algorithm,ruled]{algorithm}
\algnewcommand\algorithmicforeach{\textbf{for each}}
\algdef{S}[FOR]{ForEach}[1]{\algorithmicforeach\ #1\ \algorithmicdo}

\usepackage{mathtools} % for "\DeclarePairedDelimiter" macro
\DeclarePairedDelimiter{\floor}{\lfloor}{\rfloor}
\DeclarePairedDelimiter{\ceil}{\lceil}{\rceil}
\DeclarePairedDelimiter{\abs}{\lvert}{\rvert}

%---------------------------------------------------%
%     Define Distances Used for Document Margins    %
%---------------------------------------------------%

\newcommand{\hangindentdistance}{1cm}
\newcommand{\defaultleftmargin}{0.25in}
\newcommand{\questionleftmargin}{-.5in}

\setlength{\parskip}{1em}
\setlength{\oddsidemargin}{\defaultleftmargin}

%---------------------------------------------------%
%      Configure the Document Header and Footer     %
%---------------------------------------------------%

% Set up page formatting
\usepackage{todonotes}
\usepackage{fancyhdr}                   % Used for every page footer and title.
\pagestyle{fancy}
\fancyhf{}                              % Clears both the header and footer
\renewcommand{\headrulewidth}{0pt}      % Eliminates line at the top of the page.
\fancyfoot[LO]{\course\ -- \assnName}   % Left
\fancyfoot[CO]{\thepage}                % Center
\fancyfoot[RO]{\name}                   % Right

%---------------------------------------------------%
%           Define the Title Page Entries           %
%---------------------------------------------------%

\title{\textbf{\course\ -- \assnName}}
\author{\name}

%---------------------------------------------------%
% Define the Environments for the Problem Inclusion %
%---------------------------------------------------%

\usepackage{scrextend}
\newcounter{problemCount}
\setcounter{problemCount}{0}  % Reset the subproblem counter

\newcounter{subProbCount}[problemCount]   % Reset subProbCount any time problemCount changes.
\renewcommand{\thesubProbCount}{\alph{subProbCount}}  % Make it so the counter is referenced as a number

\newenvironment{problemshell}{
  \begin{addmargin}[\questionleftmargin]{0em}
    \par%
    \medskip
    \leftskip=0pt\rightskip=0pt%
    \setlength{\parindent}{0pt}
    \bfseries
  }
  {
    \par\medskip
  \end{addmargin}
}
\newenvironment{problem}
{%
  \refstepcounter{problemCount} % Increment the subproblem counter.  This must be before the exercise to ensure proper numbering of claims and lemmas.
  \begin{problemshell}
    \noindent \textit{Exercise~\#\arabic{problemCount}} \\
  }
  {
  \end{problemshell}
  %  \setcounter{subProbCount}{0} % Reset the subproblem counter
}
\newenvironment{subproblem}
{%
  \begin{problemshell}
    \refstepcounter{subProbCount} % Increment the subproblem counter
    \setlength{\leftskip}{\hangindentdistance}
    % Print the subproblem count and offset to the left
    \hspace{-\hangindentdistance}(\alph{subProbCount}) \tabto{0pt}
  }
  {
  \end{problemshell}
}

% Change interline spacing.
\renewcommand{\baselinestretch}{1.1}
\newenvironment{aligncustom}
{ \csname align*\endcsname % Need to do this instead of \begin{align*} because of LaTeX bug.
  \centering
}
{
  \csname endalign*\endcsname
}

%---------------------------------------------------%
%      Define commands for coloring the text.       %
%---------------------------------------------------%

\newcommand{\red}[1]{{\color{red} #1}}
\newcommand{\blue}[1]{{\color{blue} #1}}
\newcommand{\green}[1]{{\color{green} #1}}

%---------------------------------------------------%
% Define the Environments for the Problem Inclusion %
%---------------------------------------------------%

\usepackage{amsthm}       % Allows use of the ``proof'' environment.

% Number lemmas and claims using the problem count
\newtheorem{claim}{Claim}[problemCount]
\newtheorem{lemma}{Lemma}[problemCount]

%---------------------------------------------------%
%       Define commands related to managing         %
%    floats (e.g., images) across multiple pages    %
%---------------------------------------------------%

\usepackage{placeins}     % Allows \FloatBarrier

% Prevent preceding floats going to this page
\newcommand{\floatnewpage}{\FloatBarrier\newpage}

% Add the specified input file and prevent any floated figures/tables going onto the same page as new input
\newcommand{\problemFile}[1]{
  \floatnewpage
  \input{#1}
}



\begin{document}
  \maketitle

  \noindent
  \textbf{Name}: \name\\
  \textbf{Course}: \course\\
  \textbf{Assignment}: \assnName\\
  \textbf{Due Date}: \dueDate

  \noindent
  \textbf{Other Student Discussions}: I discussed the problems in this homework with the following student(s) below.  All write-ups were prepared independently.
  \vspace{-1em}
  \begin{itemize}
    \item <Enter Other Student Name>
  \end{itemize}

  \floatnewpage
  \begin{problem}
  (\textnormal{Problem~5.2.2}) The number of red chips and white chips in an urn is unknown, but the \textit{proportion},~$p$, of reds is either~$\frac{1}{3}$ or~$\frac{1}{2}$. A sample size of~5, drawn with replacement, yields the sequence red, white, white, red, and white.  What is the maximum likelihood estimate for~$p$?
\end{problem}

If ${p=\frac{1}{3}}$, then the sample's likelihood is:

\begin{equation}\label{eq:P01:1s3Likelihood}
  L\left(\frac{1}{3}\right) = \left(\frac{1}{3}\right)^2 \left( \frac{2}{3} \right)^{3} = \frac{8}{243} \approx 0.0329\text{.}
\end{equation}

\noindent
If ${p=\frac{1}{2}}$, then the sample's likelihood is:

\begin{equation}\label{eq:P01:1s2Likelihood}
  L\left(\frac{1}{2}\right) = \left(\frac{1}{2}\right)^2 \left( \frac{1}{2} \right)^{3} = \frac{1}{32} \approx 0.0313\text{.}
\end{equation}

\noindent
Since ${0.0329 > 0.0313}$, the maximum likelihood estimate of~$p$ is~$\boxed{\frac{1}{3}}$.

  \floatnewpage
  \begin{problem}
  (\textnormal{Problem~5.2.3}) Use the sample~${Y_1 = 8.2}$, ${Y_2 = 9.1}$, ${Y_3 = 10.6}$, and~${Y_4 = 4.9}$ to calculate the maximum likelihood estimate for~$\lambda$ in the exponential pdf:

  \begin{equation}\label{eq:P02:Exponential}
    f_{Y}(y;\lambda) = \lambda e^{-\lambda y}, \hspace{0.1cm} y \geq 0 \text{.}
  \end{equation}
\end{problem}

Given some vector~$\vec{Y}$ of dimension~$n$, the likelihood is

\begin{equation}\label{eq:P02:Likelihood}
  L_{n}(\lambda) = \lambda^{n}\exp\left(-\lambda\sum_{i=1}^{n} y_{i}\right)\text{.}
\end{equation}

\noindent
The log likelihood of the aforementioned sequence is

\begin{equation}\label{eq:P02:LogLikelihood}
  \ell(\lambda) = \ln\left(L(\lambda)\right) = n\ln(\lambda) - \lambda \sum_{i=1}^{n} y_i\text{.}
\end{equation}

\noindent
Its derivative is:

\begin{equation}\label{eq:P02:LogLikelihood:Derivative}
  \ell'(\lambda) = \frac{4}{\lambda} - \sum_{i=1}^{4} y_i = \frac{4}{\lambda} - 32.8  \text{.}
\end{equation}

\noindent
Setting the above equal to zero, we see that maximum likelihood estimate is

\begin{align}
  \frac{4}{\lambda_{\text{e}}} - 32.8 &= 0 \\
  \lambda_{\text{e}} &= \frac{4}{32.8} \\
                     &\approx \boxed{0.1220} \text{.}
\end{align}

  \floatnewpage
  \begin{problem}
  (\textnormal{Problem 5.2.4}) Suppose a random sample of size~$n$ is drawn from the probability model

  \begin{equation}\label{eq:P03:Distr}
    p_{X}(k;\theta) = \frac{\theta^{2k} e^{-\theta^2}}{k!}, \hspace{0.1cm}k=0,1,2,\ldots
  \end{equation}

  \noindent
  Find a formula for the maximum likelihood estimator,~$\hat{\theta}$.
\end{problem}

For an i.i.d.\ sample of size~$n$, the likelihood is:

\begin{equation}\label{eq:P03:Likelihood}
  L_{n}(\theta) = \frac{e^{-n\theta^{2}} \theta^{\left(2\sum_{i=1}^{n} k_i\right)}}{\prod_{i=1}^{n} k_i !}\text{.}
\end{equation}

\noindent
The log likelihood is

\begin{equation}\label{eq:P03:LogLikelihood}
  \ln\left(L_{n}(\theta)\right) = -n\theta^{2} + {\left(2\sum_{i=1}^{n} k_i\right)} \ln\left(\theta\right) - \sum_{i=1}^{n} \ln\left(k_{i}!\right) \text{.}
\end{equation}

\noindent
Taking the derivative with respect to~$\theta$ yields:

\begin{equation}\label{eq:P03:LogLikelihood:Deriv}
  \ell'(\theta) = -2n\theta + \frac{2\sum_{i=1}^{n} k_i}{\theta}\text{.}
\end{equation}

\noindent
Setting equal to zero yields:

\begin{align}
  \hat{\theta}^{2} &= \frac{\sum_{i=1}^{n} k_i}{n} \\
  \hat{\theta} &= \boxed{\sqrt{\frac{\sum_{i=1}^{n} k_i}{n}}} \text{.}
\end{align}

  \floatnewpage
  \begin{problem}
  (\textnormal{Problem~5.2.7}) An engineer is creating a project schedule program and recognizes that the tasks making up the project are not always completed on time.  However, the completion proportion tends to be fairly high.  To reflect this condition, he uses the pdf

  \begin{equation}\label{eq:P04}
    f_{Y}(y;\theta)=\theta y^{\theta - 1}, \hspace{0.1cm} 0 \leq y \leq 1, \hspace{0.1cm} 0 < \theta
  \end{equation}

  \noindent
  where $y$~is the proportion of the task completed.  Suppose that in his previous project, the proportions of tasks completed were~0.77, 0.82, 0.92, 0.94, and~0.98.  Estimate~$\theta$.
\end{problem}

For a set of~$n$ proportions, the likelihood is

\begin{equation}\label{eq:P04:Likelihood}
  L_{n}(\theta) = \theta^{n} \prod_{i=1}^{n} y_{i}^{\theta - 1} \text{.}
\end{equation}

\noindent
The log likelihood is

\begin{equation}\label{eq:P04:LogLikelihood}
  \ell_{n}(\theta) = n\ln(\theta) + (\theta - 1) \sum_{i=1}^{n} \ln y_i \text{.}
\end{equation}

\noindent
The log likelihood's derivative is:

\begin{equation}\label{eq:P04:LogLikelihood:Deriv}
  \ell_{n}'(\theta) = \frac{n}{\theta} + \sum_{i=1}^{n} \ln y_i \text{.}
\end{equation}

\noindent
Setting this to zero yields

\begin{align}
  \frac{n}{\theta} &= \sum_{i=1}^{n} \ln\left(\frac{1}{y_i} \right) \\
  \theta &= \frac{n}{\sum_{i=1}^{n} 1 / y_i} \\
         &\approx \boxed{7.996} \text{.}
\end{align}

  \floatnewpage
  \begin{problem}
  (\textnormal{5.2.10})
\end{problem}

\begin{subproblem}\label{P05:A}
  Based on the random sample ${Y_1=6.3}$, ${Y_2 = 1.8}$, ${Y_3 = 14.2}$, and ${Y_4 = 7.6}$, use the method of maximum likelihood to estimate the parameter~$\theta$ in the uniform pdf

  \begin{equation}
    f_{Y}(y;\theta) = \frac{1}{\theta}, \hspace{0.1cm} 0 \leq y \leq \theta \text{.}
  \end{equation}
\end{subproblem}

This is the same as the example done in class on January~8, 2020 where ${\hat{\theta} = \max\set{Y_1,\ldots,Y_{n}}}$.  Therefore, $\boxed{{\hat{\theta} = 14.2}}$.

\begin{subproblem}
  Suppose the random sample in part~$(\ref{P05:A})$ represents the two parameter uniform pdf

  \begin{equation}\label{eq:P05:b}
    f_{Y}(y;\theta_1,\theta_2) = \frac{1}{\theta_2 - \theta_1}, \hspace{0.1cm} \theta_{1} \leq y \leq \theta_{2} \text{.}
  \end{equation}

  \noindent
  Find the maximum likelihood estimates for~$\theta_1$ and~$\theta_{2}$.
\end{subproblem}

Similar to part~(\ref{P05:A}), the likelihood for an $n$\-/dimensional sample is

\begin{equation}
  L_{n}(\theta_{1},\theta_{2}) =  \begin{cases}
                                    \frac{1}{(\theta_{2} - \theta_{1})^{n}} & \theta_{1} \leq \min\set{Y_1,\ldots,Y_n} \wedge \theta_2 \geq \max\set{Y_1,\ldots,Y_n} \\
                                    0 & \text{Otherwise}
                                  \end{cases}\text{.}
\end{equation}

\noindent
For a fixed ${\theta_{1} \leq \min\set{Y_{1},\ldots,Y_{n}}}$, the value of~$\theta_{2}$ that maximizes the likelihood is~${\max\set{Y_1,\ldots,Y_n}}$ (since it keeps the denominator as small as possible).  Similarly, the denominator is minimized when ${\theta_{1} = \min\set{Y_{1},\ldots,Y_{n}}}$ for fixed~${\theta_{2} \geq \max\set{Y_{1},\ldots,Y_{n}}}$.

Given the above, the MLE is $\boxed{{\hat{\theta}_{1}=1.8}}$ and $\boxed{{\hat{\theta}_{2} = 14.2}}$.

  \floatnewpage
  \begin{problem}
  (\textnormal{Problem~5.2.12}) A random sample of size~$n$ is taken from the pdf

  \begin{equation}\label{eq:P06}
    f_{Y}(y;\theta) = \frac{2y}{\theta^2}, \hspace{0.1cm} 0 \leq y \leq \theta\text{.}
  \end{equation}
  Find an expression for~$\hat{\theta}$, the maximum likelihood estimator for~$\theta$.
\end{problem}

Given some sample~${Y_1,\ldots,Y_n}$, the likelihood is

\begin{equation}\label{eq:P06:Likelihood}
  L_n(\theta) = \begin{cases}
    \frac{2^n \prod_{i=1}^n y_i}{\theta^{2n}} & \theta \geq \max\set{Y_1,\ldots,Y_n} \\
                  0 & \text{Otherwise}\text{.}
                \end{cases}
\end{equation}

\noindent
For that fixed sample, the likelihood is maximized when the denominator is minimized.  This occurs when ${\theta = \pm\max\set{Y_1,\ldots,Y_n}}$.  Since the preimage is constrained by ${0 \leq y \leq \theta}$, $\theta$ cannot be negative.  Therefore, $\boxed{{\hat{\theta} = \max\set{Y_1,\ldots,Y_n}}}$

% \noindent
% The log likelihood is

% \begin{equation}\label{eq:P06:LogLikelihood}
%   \ell_n(\theta) = n \ln 2 + \sum_{i=1}^n \left( \ln y_i \right) - 2n \ln\theta \text{.}
% \end{equation}

  \floatnewpage
  \begin{problem}
  (\textnormal{Problem~5.2.14}) For the negative binomial pdf:

  \begin{equation}\label{eq:P07}
    p_{X}(k;p,r) =  \left(%
                      \begin{array}{c}
                        k + r - 1 \\
                        k
                      \end{array}
                    \right) (1-p)^{k} p^{r}\text{,}
  \end{equation}

  \noindent
  find the maximum likelihood estimator for~$p$ if~$r$ is known.
\end{problem}

\end{document}
